%%%%%%%%%%%%%%%%%%%%%%%%%%%%%%%%%%%%%%%%%%%%%%%%%%%%%%%%%%%%%%%%%%%%%%%%%%%%%%%
% Memorial para concurso público de Professor Doutor na USP.
%
% Formatação inspirada em:
% * https://tug.org/pracjourn/2008-1/mori/mori.pdf
% * https://github.com/santisoler/phd-thesis
% * https://github.com/compgeolab/dissertation-template
%%%%%%%%%%%%%%%%%%%%%%%%%%%%%%%%%%%%%%%%%%%%%%%%%%%%%%%%%%%%%%%%%%%%%%%%%%%%%%%

%%%%%%%%%%%%%%%%%%%%%%%%%%%%%%%%%%%%%%%%%%%%%%%%%%%%%%%%%%%%%%%%%%%%%%%%%%%%%%%
% Set a class and import packages
\documentclass[12pt,a4paper,oneside]{book}

% Variables
\newcommand{\Year}{2025}
\newcommand{\Author}{Leonardo Uieda}
\newcommand{\Title}{Memorial circunstanciado}
\newcommand{\TitlePDF}{Memorial circunstanciado para concurso visando obtenção do título de Livre Docente no IAG - Universidade de São Paulo, Edital ATAc-IAG/005/2025}
\newcommand{\MemorialDOI}{10.6084/m9.figshare.28737800}
\newcommand{\Email}{uieda@usp.br}
\newcommand{\ORCID}{0000-0001-6123-9515}
\newcommand{\ResearcherID}{G-3258-2012}
\newcommand{\GoogleScholar}{qfmPrUEAAAAJ}
\newcommand{\Lattes}{8939551682050504}

% Variables for easier typing of some names
\newcommand{\USP}{Universidade de São Paulo}
\newcommand{\IAG}{Instituto de Astronomia, Geofísica e Ciências Atmosféricas}
\newcommand{\UoL}{University of Liverpool}
\newcommand{\ON}{Observatório Nacional}
\newcommand{\UHM}{University of Hawai`i at M\={a}noa}
\newcommand{\UERJ}{Universidade do Estado do Rio de Janeiro}
\newcommand{\Trieste}{Università degli studi di Trieste}

% Names for citing coauthors
\newcommand{\Me}{\textbf{Uieda, L}}
\newcommand{\Val}{Barbosa, VCF}
\newcommand{\Bi}{Oliveira Jr, VC}
\newcommand{\Paul}{Wessel, P}
\newcommand{\Joaquim}{Luis, J}
\newcommand{\Remko}{Scharroo, R}
\newcommand{\Florian}{Wobbe, F}
\newcommand{\Walter}{Smith, WHF}
\newcommand{\Dongdong}{Tian, D}
\newcommand{\Bridget}{Smith-Konter, B}
\newcommand{\Eric}{Xu, X}
\newcommand{\David}{Sandwell, DT}
\newcommand{\Carla}{Braitenberg, C}
\newcommand{\Naomi}{Ussami, N}
\newcommand{\Manoel}{D'Agrella-Filho, MS}
\newcommand{\JB}{Silva, JBC}
\newcommand{\Dai}{Sales, DP}
\newcommand{\Figura}{Melo, FF}
\newcommand{\Dio}{Carlos, DU}
\newcommand{\BragaVale}{Braga, MA}
\newcommand{\YLi}{Li, Y}
\newcommand{\Angeli}{Angeli, G}
\newcommand{\Peres}{Peres, G}
\newcommand{\Everton}{Bomfim, EP}
\newcommand{\Eder}{Molina, E}
\newcommand{\Gomes}{Gomes, AAS}
\newcommand{\Santiago}{Soler, SR}
\newcommand{\Agustina}{Pesce, A}
\newcommand{\Gimenez}{Gimenez, ME}
\newcommand{\Kristoffer}{Hallam, KAT}
\newcommand{\Guangdong}{Zhao, G}
\newcommand{\Bo}{Chen, B}
\newcommand{\JLiu}{Liu, J}
\newcommand{\LChen}{Chen, L}
\newcommand{\RGuo}{Guo, R}
\newcommand{\MKaban}{Kaban, MK}
\newcommand{\Lindsey}{Heagy, LJ}
\newcommand{\Lion}{Krischer, L}
\newcommand{\Rene}{Gassmoeller, R}
\newcommand{\Bane}{Sullivan, CB}
\newcommand{\Jens}{Klump, JF}
\newcommand{\LBarba}{Barba, LA}
\newcommand{\JBazan}{Bazan, J}
\newcommand{\JBrown}{Brown, J}
\newcommand{\RGuimera}{Guimera, RV}
\newcommand{\MGymrek}{Gymrek, M}
\newcommand{\AHanna}{Alex Hanna}
\newcommand{\KHuff}{Huff, KD}
\newcommand{\DKatz}{Katz, DS}
\newcommand{\CMadan}{Madan, CR}
\newcommand{\KMoerman}{Moerman, KM}
\newcommand{\KNiemeyer}{Niemeyer, KE}
\newcommand{\JPoulson}{Poulson, JL}
\newcommand{\PPrins}{Prins, P}
\newcommand{\KRam}{Ram, K}
\newcommand{\ARokem}{Rokem, A}
\newcommand{\Arfon}{Smith, AM}
\newcommand{\GThiruvathukal}{Thiruvathukal, GK}
\newcommand{\KThyng}{Thyng, KM}
\newcommand{\BWilson}{Wilson, BE}
\newcommand{\Yehudi}{Yehudi, Y}
\newcommand{\Remi}{Rampin, R}
\newcommand{\Hugo}{van Kemenade, H}
\newcommand{\MattTurk}{Turk, M}
\newcommand{\Shapero}{Shapero, D}
\newcommand{\Anderson}{Banihirwe, A}
\newcommand{\Leeman}{Leeman, J}
\newcommand{\JEbbing}{Ebbing, J}
\newcommand{\AGuy}{Guy, A}
\newcommand{\JFarquharson}{Farquharson, J}
\newcommand{\AKushnir}{Kushnir, A}
\newcommand{\FWadsworth}{Wadsworth, F}
\newcommand{\LPerozzi}{Perozzi, L}
\newcommand{\MWieczorek}{Wieczorek, MA}
\newcommand{\LLi}{Li, L}
\newcommand{\Ricardo}{Trindade, RIF}
\newcommand{\Yago}{Castro, YM}
\newcommand{\Arthur}{Macêdo, AS}
\newcommand{\India}{Uppal, I}
\newcommand{\Gelson}{Souza-Junior, GF}
\newcommand{\Roger}{Fu, RR}
\newcommand{\Janine}{Carmo, J}
\newcommand{\Ualisson}{Bellon, UD}
\newcommand{\Wyn}{Williams, W}
\newcommand{\Muxworthy}{Muxworthy, AR}
\newcommand{\Les}{Nagy, L}

% Links to webpages I use often
\newcommand{\SantiagoLink}{\href{https://www.santisoler.com/}{Santiago R. Soler}}
\newcommand{\VanderleiLink}{\href{https://www.pinga-lab.org/people/oliveira-jr.html}{Vanderlei C. Oliveira Jr.}}
\newcommand{\SandwellLink}{\href{https://topex.ucsd.edu/sandwell/}{David Sandwell}}
\newcommand{\ValeriaLink}{\href{https://www.pinga-lab.org/people/barbosa.html}{Valéria C. F. Barbosa}}
\newcommand{\PaulLink}{\href{https://www.soest.hawaii.edu/pwessel/}{Paul Wessel}}
\newcommand{\IndiaLink}{\href{https://www.compgeolab.org/team/\#indiauppal}{India Uppal}}
\newcommand{\GelsonLink}{\href{https://www.compgeolab.org/team/\#Souza-junior}{Gelson Ferreira de Souza Junior}}
\newcommand{\YagoLink}{\href{https://www.compgeolab.org/team/\#YagoMCastro}{Yago Moreira Castro}}
\newcommand{\EllenLink}{\href{https://www.compgeolab.org/team/\#fernandesellen}{Ellen Fernandes Marcos}}
\newcommand{\GabrielLink}{\href{https://www.compgeolab.org/team/\#chagas-gabriel}{Gabriel Aparecido das Chagas Silva}}
\newcommand{\ArthurLink}{\href{https://www.compgeolab.org/team/\#arthursmacedo}{Arthur Siqueira Macêdo}}
\newcommand{\ErosLink}{\href{https://www.compgeolab.org/team/\#ErosKerouak}{Eros Kerouak Cordeiro Pereira}}
\newcommand{\SpirosLink}{\href{https://www.yorku.ca/spiros/spiros.html}{Spiros Pagiatakis}}
\newcommand{\GMTLink}{\href{https://www.generic-mapping-tools.org}{Generic Mapping Tools}}
\newcommand{\CompGeoLabLink}{\href{https://www.compgeolab.org/}{Computer-Oriented Geoscience Lab}}
\newcommand{\SwungLink}{\href{https://softwareunderground.org/}{Software Underground}}
\newcommand{\FatiandoLink}{\href{https://www.fatiando.org}{Fatiando a Terra}}
\newcommand{\PyGMTLink}{\href{https://www.pygmt.org}{PyGMT}}
\newcommand{\SSILink}{\href{https://software.ac.uk/}{Software Sustainability Institute}}
\newcommand{\JOSSLink}{\href{https://joss.theoj.org/}{Journal of Open Source Software}}
\newcommand{\PyOSciLink}{\href{https://www.pyopensci.org/}{pyOpenSci}}
\newcommand{\EarthArXivLink}{\href{https://eartharxiv.org/}{EarthArXiv}}
\newcommand{\RBRLink}{\href{https://www.reprodutibilidade.org/}{Rede Brasileira de Reprodutibilidade}}

% Import packages
\usepackage[utf8]{inputenc}
\usepackage[T1]{fontenc}
\usepackage[brazil]{babel}
\usepackage{geometry}
\usepackage{graphicx}
\usepackage{amssymb}
\usepackage{amsmath}
\usepackage{hyperref}
% create fancy headers
\usepackage{fancyhdr}
% commands for managing dates and its formats
\usepackage{datetime2}
% improved urls with proper hyphenation
\usepackage{xurl}
% Control over enumerate and itemize
\usepackage{enumitem}
% Tweak the look of captions
\usepackage{caption}
% To control the style of section titles
\usepackage{titlesec}
% Add the bibliography to the table of contents
\usepackage[nottoc,chapter]{tocbibind}
\usepackage[round,authoryear,sort]{natbib}
% show dois as links on references
\usepackage{doi}
% Icon and fonts (requires using xelatex or luatex)
\usepackage{fontawesome5}
\usepackage{academicons}
\usepackage{fontspec}
% \usepackage[mono]{notomath}
% \usepackage{mathpazo}
\usepackage[sfdefault]{atkinson}
% To make everything neater
\usepackage{microtype}
% To make fancy text boxes
\usepackage{xcolor}
\usepackage[framemethod=default]{mdframed}
% For fancy and multipage tables
\usepackage{tabularx}
\usepackage{ltablex}
% To define custom environments
\usepackage{environ}
\usepackage{setspace}
% Reference sections by name
\usepackage{nameref}
% Better handling of footnotes inside summary boxes
\usepackage{footmisc}
% To get the number of pages in the document
\usepackage{lastpage}
%%%%%%%%%%%%%%%%%%%%%%%%%%%%%%%%%%%%%%%%%%%%%%%%%%%%%%%%%%%%%%%%%%%%%%%%%%%%%%%

%%%%%%%%%%%%%%%%%%%%%%%%%%%%%%%%%%%%%%%%%%%%%%%%%%%%%%%%%%%%%%%%%%%%%%%%%%%%%%%
% Configuration of the document

\geometry{%
  left=20mm,
  right=20mm,
  top=20mm,
  bottom=15mm,
  headsep=5mm,
  headheight=5mm,
  footskip=10mm,
  includehead=true,
  includefoot=true
}

% Increase the line spacing
\SetSinglespace{1.2}
\onehalfspacing

% Remove spacing between enumerate/itemize items
\setlist{nosep}

% Padding between the first figure and the chapter title
\newcommand{\HeroFigPad}{\vspace{-1cm}}

% Padding before the software logo figures
\newcommand{\SoftwareFigPad}{\vspace{-0.3cm}}

% Add a link to a DOI
\newcommand{\DOI}[1]{\url{https://doi.org/#1}}

% Add a link to a GitHub repository
\newcommand{\GitHub}[1]{\faGithub{} Código: \url{https://github.com/#1}}

% Add a link to a YouTube or Vimeo video
\newcommand{\YouTube}[1]{\faYoutube{} Vídeo: \url{https://youtu.be/#1}}
\newcommand{\Vimeo}[1]{\faVimeo{} Vídeo: \url{https://https://vimeo.com/#1}}

% Add a link to a supplementary data
\newcommand{\Data}[1]{\faChartBar{} Dados: \url{https://doi.org/#1}}

% Add a link to a preprint
\newcommand{\Preprint}[1]{\faLockOpen{} Preprint: \url{https://doi.org/#1}}

% Make a Unicode bullet symbol
\newcommand{\Bullet}{•\enspace}

% Define custom colors
\definecolor{lu_gray}{gray}{0.98}
\definecolor{lu_darkgray}{gray}{0.3}
\definecolor{lu_mediumgray}{gray}{0.5}
\definecolor{lu_blue}{RGB}{32, 96, 194}
\definecolor{lu_lightblue}{RGB}{238, 245, 250}
\definecolor{lu_yellow}{RGB}{255, 193, 7}
\definecolor{lu_lightyellow}{RGB}{255, 249, 230}

% Customize how Chapter headings are displayed
\titleformat{\chapter}[display]{\normalfont}{\large Capítulo \thechapter}{0pt}{\huge}[\titlerule]
\titlespacing*{\chapter}{0pt}{-40pt}{40pt}

% Set the spacing between bibliography entries (requires natbib)
\setlength{\bibsep}{0pt}

% Configure captions
\captionsetup{labelfont=bf,font={small,color=lu_darkgray},skip=0pt}

% Define a fancy text box
\mdfdefinestyle{summarybox}{%
  leftline=true,
  rightline=false,
  topline=false,
  bottomline=false,
  linewidth=4pt,
  linecolor=lu_blue,
  frametitlefont=\bfseries\color{black}\small,
  frametitlebackgroundcolor=lu_lightblue,
  frametitleaboveskip=7pt,
  frametitlebelowskip=7pt,
  frametitlerule=true,
  frametitlerulewidth=1pt,
  backgroundcolor=lu_gray,
  innertopmargin=7pt,
  innerbottommargin=10pt,
  innerleftmargin=15pt,
  innerrightmargin=15pt,
  skipbelow=5pt,
  skipabove=0pt,
}
\newmdenv[style=summarybox]{summarybox}
\mdfdefinestyle{subsummarybox}{%
  leftline=true,
  rightline=false,
  topline=false,
  bottomline=false,
  linewidth=4pt,
  linecolor=lu_yellow,
  frametitlefont=\bfseries\color{black}\small,
  frametitlebackgroundcolor=lu_lightyellow,
  frametitleaboveskip=7pt,
  frametitlebelowskip=7pt,
  frametitlerule=true,
  frametitlerulewidth=1pt,
  backgroundcolor=lu_gray,
  innertopmargin=7pt,
  innerbottommargin=10pt,
  innerleftmargin=15pt,
  innerrightmargin=15pt,
  skipbelow=5pt,
  skipabove=0pt,
}
\newmdenv[style=subsummarybox]{subsummarybox}

% Define something like an fa-ul and a date list
\NewEnviron{fa-ul}{%
  \vspace{-0.4cm}
  \small
  \renewcommand{\arraystretch}{1.25}
  \begin{tabularx}{\linewidth}{@{}p{0.05\linewidth}@{}@{}p{0.95\linewidth}@{}}
    \BODY
  \end{tabularx}%
}
\NewEnviron{datelist}{%
  \vspace{-0.4cm}
  \small
  \renewcommand{\arraystretch}{1.25}
  \begin{tabularx}{\linewidth}{@{}p{0.15\linewidth}@{}@{}p{0.85\linewidth}@{}}
    \BODY
  \end{tabularx}%
}
\NewEnviron{paperlist}{%
  \vspace{-0.4cm}
  \small
  \renewcommand{\arraystretch}{1.25}
  \begin{tabularx}{\linewidth}{@{}p{0.08\linewidth}@{}@{}p{0.92\linewidth}@{}}
    \BODY
  \end{tabularx}%
}
\NewEnviron{courselist}{%
  \vspace{-0.4cm}
  \small
  \renewcommand{\arraystretch}{1.25}
  \begin{tabularx}{\linewidth}{@{}p{0.15\linewidth}@{}@{}p{0.85\linewidth}@{}}
    \BODY
  \end{tabularx}
}

% Define a fancy enumerate that has a title
\NewEnviron{fancyenum}[2]{%
  \vspace{0.25cm}
  \noindent#1\quad\textbf{#2}:
  \vspace{0.25cm}
  \begin{enumerate}
    \BODY
  \end{enumerate}
}

% Configure hyperref and add PDF metadata
\hypersetup{
    colorlinks,
    allcolors=lu_blue,
    pdftitle={\TitlePDF},
    pdfauthor={\Author},
    pdftex,
    breaklinks=true,
}

% make urls use the same font as every other text
\urlstyle{same}

% Prevent footnotes from being broken into multiple pages
\interfootnotelinepenalty=10000

% Configure headers and footers
\newcommand{\Separator}{\hspace{3pt}|\hspace{3pt}}
\newcommand{\HeaderFont}{\footnotesize\color{lu_darkgray}}
\fancyhf{}
\lhead{%
    \HeaderFont{}
    \nouppercase\leftmark{}
    \Separator{}
    \Title{}
}
\chead{}
\rhead{%
    \HeaderFont{}
    \Author{}
    \Separator{}
    \thepage\space de\space \pageref*{LastPage}
}
\cfoot{}
\renewcommand{\headrulewidth}{0.3pt}
%%%%%%%%%%%%%%%%%%%%%%%%%%%%%%%%%%%%%%%%%%%%%%%%%%%%%%%%%%%%%%%%%%%%%%%%%%%%%%%

%%%%%%%%%%%%%%%%%%%%%%%%%%%%%%%%%%%%%%%%%%%%%%%%%%%%%%%%%%%%%%%%%%%%%%%%%%%%%%%
\begin{document}

\pagestyle{plain}
\frontmatter


%==============================================================================
\begin{titlepage}
  \begin{center}
    \includegraphics[height=1.5cm]{images/usp.png}
    \hspace{1cm}
    \includegraphics[height=1.5cm]{images/iag.png}
    \vspace{1cm}

    UNIVERSIDADE DE SÃO PAULO

    INSTITUTO DE ASTRONOMIA, GEOFÍSICA E CIÊNCIAS ATMOSFÉRICAS
    \vspace{5cm}

    \textbf{\huge \MakeUppercase{\Title}}
    \vspace{2cm}

    {\Large \Author}
    \vspace{5cm}

    {\small
      Apresentado para concurso títulos e provas visando a obtenção do

      título de Livre Docente junto ao Departamento de Geofísica do

      Instituto de Astronomia, Geofísica e Ciências Atmosféricas da

      Universidade de São Paulo.
      \vspace{1cm}

      Edital ATAc-IAG/005/2025
    }
    \vfill

    \Year{}
  \end{center}
\end{titlepage}
%==============================================================================

{\small

\vspace*{\fill}

\noindent
\textit{\TitlePDF{}}.
\\[0.2cm]
\textcopyright{} Copyright \Year{} \Author{}.
\\[0.2cm]
Última modificação em \today.
doi:\href{https://doi.org/\MemorialDOI}{\MemorialDOI}.

\vspace{2.5cm}

\noindent
\textbf{\LARGE \faCreativeCommons{} \faCreativeCommonsBy{}}
\\
Disponível sob a
\textbf{Licença Creative Commons Atribuição 4.0 Internacional}.
\\
\url{https://creativecommons.org/licenses/by/4.0/deed.pt-br}

\vspace{0.25cm}

\noindent
Você tem o direito de:

\begin{description}[labelindent=0.5cm]
    \item[Compartilhar ---]{
        Copiar e redistribuir o material em qualquer suporte ou formato para
        qualquer fim, mesmo que comercial.
    }
    \item[Adaptar ---]{
        Remixar, transformar, e criar a partir do material para qualquer fim,
        mesmo que comercial.
    }
\end{description}

\vspace{0.25cm}

\noindent
De acordo com os termos seguintes:

\begin{description}[labelindent=0.5cm]
    \item[Atribuição ---]{
         Você deve dar o crédito apropriado, prover um link para a licença
         e indicar se mudanças foram feitas. Você deve fazê-lo em qualquer
         circunstância razoável, mas de nenhuma maneira que sugira que
         o licenciante apoia você ou o seu uso.
    }
    \item[Sem restrições adicionais ---]{
        Você não pode aplicar termos jurídicos ou medidas de caráter
        tecnológico que restrinjam legalmente outros de fazerem algo que
        a licença permita.
}
\end{description}

\vspace{1.5cm}

}


%==============================================================================
\chapter*{Resumo}

Possuo Bacharelado em Geofísica pela \USP{} e Mestrado
e Doutorado em Geofísica pelo \ON{}.
Ao longo da minha formação e carreira, passei por seis instituições de ensino
superior em quarto países diferentes.
Trabalhei como Professor Assistente na \UERJ{},
Professor Visitante na \UHM{}, \textit{Lecturer} (equivalente a
Professor Doutor) na \UoL{} e
atualmente sou Professor Doutor no Departamento de Geofísica do \IAG{} da
\USP{}.
Ministrei 14 disciplinas diferentes a nível de graduação,
3 a nível de pós-graduação e 18 cursos de curta
duração abrangendo uma gama de tópicos da geofísica, geologia e programação.
Sou autor de 18 artigos científicos que agregam mais de 3700
citações\footnote{Segundo Google Scholar em 2025-04-06: \url{https://scholar.google.com/citations?user=qfmPrUEAAAAJ}}.
Atuo na área de ciência aberta e reprodutibilidade desde meu primeiro artigo
publicado em 2012 durante meu Mestrado.
Desenvolvo diversos projetos de software livre para ciência, dentre eles
\FatiandoLink{}, \href{https://tesseroids.leouieda.com}{Tesseroids}
e \GMTLink{},
que podem chegar a ter centenas de milhares de downloads
mensais\footnote{Por exemplo, o software
\href{https://github.com/fatiando/pooch}{Pooch} que é parte do Fatiando a
Terra: \url{https://pypistats.org/packages/pooch}}.
Sou reconhecido por minha expertise em Geociências, ciência aberta e
desenvolvimento de software livre, tendo servido como editor do
\JOSSLink{} e na
coordenação das organizações
\EarthArXivLink{},
\PyOSciLink{},
\SwungLink{}
e \RBRLink.
Em 2020, fundei o \CompGeoLabLink{}, um grupo de pesquisa com foco no
desenvolvimento metodológico em métodos potenciais, geofísica computacional
e software livre.

Este memorial apresenta minha formação e atuação profissional, incluindo
reflexões sobre os fatores que me trouxeram até onde estou e as lições que
aprendi ao longo do caminho.
Além disso, o memorial também relata meus planos futuros para meu
desenvolvimento profissional e a direção na qual pretendo guiar meu grupo de
pesquisa.

\begin{figure}[tb]
  \begin{center}
    \includegraphics[width=\textwidth]{images/timeline.pdf}
  \end{center}
  \caption*{
    \textbf{Linha do tempo} (fora de escala) resumindo minha trajetória
    acadêmica, desde o início do meu curso de Bacharelado em Geofísica na
    \USP{} em 2004 até meu retorno à \USP{} em 2023.
  }
\end{figure}

%==============================================================================
\tableofcontents

\mainmatter
\pagestyle{fancy}

%==============================================================================
\chapter{Introdução}

\begin{figure}[h]
  \HeroFigPad
  \begin{center}
    \includegraphics[width=\textwidth]{images/1997-06-ithaca-creek.jpg}
  \end{center}
  \caption{
    Minha mãe mostrando para mim e minha irmã caçula o lado inferior de uma
    pedra em um riacho, provavelmente contendo invertebrados aquáticos.
    Foto de Junho de 1997, tirada no estado de Nova York, E.U.A., durante o
    pós-doutorado de meus pais na
    \href{https://www.cornell.edu/}{Cornell University}.
  }
  \label{fig_riacho}
\end{figure}
\begin{summarybox}[frametitle=\faInfoCircle{}\quad Informações para contato]
  \begin{fa-ul}
    \faEnvelope & Email: \href{mailto:\Email}{\Email} \\
    \aiOrcid & ORCID: \href{https://orcid.org/\ORCID}{\ORCID} \\
    \aiLattes & Currículo Lattes: \url{https://lattes.cnpq.br/\Lattes} \\
    \aiPublonsSquare & ResearcherID: \href{https://www.webofscience.com/wos/author/rid/\ResearcherID}{\ResearcherID} \\
    \faUser & Página pessoal: \url{https://www.leouieda.com} \\
    \faUsers & Grupo de pesquisa: \url{https://www.compgeolab.org}
  \end{fa-ul}
\end{summarybox}

Este memorial apresenta uma análise reflexiva sobre os principais temas de
minha carreira acadêmica: minha formação, minhas linhas de pesquisa, minhas
atividades de ensino e extensão e meus esforços para tornar a ciência feita em
nossa disciplina mais aberta, reprodutível e acessível para uso das comunidades
científica, acadêmica e empresarial.
Ao buscar a fonte de vários dos princípios que me guiam hoje em dia, percebi
que o ponto mais adequado para começar seria com uma análise das influências
que tive durante minha criação.

\section{Influências durante a infância e a adolescência}

Meu primeiro contato com a ciência foi através de meus pais,
\href{https://orcid.org/0000-0002-6078-1342}{Virginia Sanches Uieda} e
\href{https://orcid.org/0000-0002-4177-3339}{Wilson Uieda},
ambos professores aposentados do Instituto de Biociências da Universidade
Estadual Paulista Júlio de Mesquita Filho (UNESP) de Botucatu, São Paulo.
Eles rotineiramente incluíam minhas duas irmãs e eu em suas atividades como
docentes da UNESP, o que nos proporcionou oportunidades de aprendizagem únicas
e que foram particularmente influentes na minha formação.
Tenho memórias marcantes de coletar peixes e invertebrados aquáticos com minha
mãe (figura~\ref{fig_riacho}), fotografar morcegos do gênero
\href{https://pt.wikipedia.org/wiki/Artibeus}{\textit{Artibeus}} se alimentando
dos frutos do chapéu-de-sol com meu pai, observar minha mãe corrigindo provas
de zoologia de vertebrados e tentar acertar mais questões que seus alunos,
alimentar os morcegos
\href{https://pt.wikipedia.org/wiki/Desmodus_rotundus}{\textit{Desmodus rotundus}}
que meu pai mantinha em cativeiro com cubos de sangue bovino congelado nos
finais de semana e acompanhar minha mãe na disciplina de campo sobre cetáceos
onde pudemos interagir diretamente com botos-cinza
(\href{https://pt.wikipedia.org/wiki/Sotalia_guianensis}{\textit{Sotalia guianensis}})
em seu habitat natural.

A curiosidade, a dedicação e a ética dos meus pais formaram a base da minha
posição a respeito da ciência e do que significa ser um educador de qualidade.
Essa base e todo o apoio que recebi de meus pais foram fundamentais para
alcançar tudo o que consegui até hoje (i.e., o conteúdo deste memorial).

\section{Reflexão sobre vantagens e privilégios}

Este memorial representa todas as minhas conquistas ao longo da minha carreira.
Dedicação, esforço e talento (i.e., mérito) foram certamente importantes para
meu sucesso profissional.
Porém, seria muito ingênuo de minha parte assumir que esses foram os únicos
fatores que influenciaram minha trajetória.
Por isso, acho importante refletir sobre as vantagens e privilégios que tive
sobre meus contemporâneos para dar contexto ao resto do memorial.

Primeiramente, sou homem, heterossexual, cisgênero e de etnia mista branca
europeia e norte asiática.
A junção desses fatores significa que, por nenhum mérito próprio, tive que
superar um número consideravelmente menor de barreiras ao longo de minha
carreira que outras pessoas.
Fui criado por pais dedicados e com imenso suporte de toda minha família
estendida.
Minha família é de classe média alta e tive acesso a educação privada em boas
escolas.
Ao contrário de alguns dos meus colegas do curso de graduação, não tive que
trabalhar para me sustentar durante meu curso de graduação, podendo me dedicar
exclusivamente aos estudos\footnote{E, é claro, às festas e outras atividades
culturais que enriquecem a experiência universitária.}.

Ter pais acadêmicos, em particular, me conferiu diversas vantagens.
Antes mesmo de ingressar no ensino superior, eu já sabia sobre o estilo de
trabalho, a trajetória para se chegar ao cargo de Professor Doutor, o balanço
entre ensino, pesquisa e extensão, os tipos de cargos administrativos que
existem, entre outros.
Mas talvez a vantagem mais importante que meus pais deram foi a oportunidade
de morar no exterior quando criança.
Entre Agosto de 1996 e Dezembro de 1997, meus pais fizeram um pós-doutorado
na \href{https://www.cornell.edu/}{Cornell University}, E.U.A., levando junto
toda a família.
Por isso, cursei o quinto e sexto ano do ensino fundamental nos Estados Unidos
e aprendi a ler, escrever e falar inglês fluentemente.
Somente percebi o quanto esse único fator (fluência na língua inglesa) me foi
vantajoso após ingressar no curso de Bacharelado em Geofísica da Universidade
de São Paulo (seção~\ref{sec_usp}).
Eu era capaz de ler livros e artigos em inglês em menos tempo que meus colegas,
me comunicava com pesquisadores estrangeiros naturalmente durante meu trabalho
de conclusão de curso e creio que minha fluência na língua foi um fator
importante para conseguir o intercâmbio com a York University, Canadá
(seção~\ref{sec_york}).

A sorte é outro fator que foi muito importante em diversas etapas da minha
carreira.
Minha decisão de prestar o vestibular da USP para o curso de Geofísica dependeu
de minha irmã mais velha encontrar aleatoriamente um aluno de Geofísica no
``bandeijão'' da USP que lhe contou sobre o curso.
Como eu estava indeciso sobre minhas escolhas de carreira, selecionei Geofísica
como minha primeira opção por conselho de minha irmã sem saber exatamente do
que se tratava o curso.
Ter entrado no curso de Geofísica na USP no ano de 2004, em particular, foi
extremamente oportuno.
A turma da Geofísica de 2004 é simplesmente excepcional.
O apoio da turma foi muito importante, tanto para superar momentos desafiadores
quanto para elevar cada um de nós a alcançar além do que achávamos possível.
Além disso, pude usufruir desse suporte ainda na pós-graduação no \ON{} (seção~\ref{sec_on}), tanto por conta de vários membros da turma
estarem trabalhando no Rio de Janeiro, quanto por ter meu amigo
\href{https://www.pinga-lab.org/people/oliveira-jr.html}{Vanderlei C. Oliveira Jr.}
comigo na pós-graduação (Vanderlei é atualmente Pesquisador Titular do
\ON{}).
Também tive muita sorte no meu acesso a mentores excelentes:
Manoel S. D'Agrella Filho, Ricardo I. F. Trindade e Naomi Ussami
durante a graduação, Valéria C. F. Barbosa e Carla Braitenberg durante a
pós-graduação e Paul Wessel durante o pós-doutorado.

Todos os fatores descritos acima me proporcionaram acesso diferenciado a
oportunidades e vantagens para conquistá-las.
Porém, um fator que considero de mérito próprio é que tive a perspicácia para
identificar essas oportunidades quando elas se apresentaram, a confiança para
aplicar e a perseverança para usufruir ao máximo de minhas conquistas.

\section{A estrutura deste memorial}

Identificar uma estrutura coerente  para este memorial que minimizasse a
sobreposição de informação entre os capítulos foi uma tarefa desafiadora.
A minha formação, atividades de ensino e pesquisa e, principalmente, minha
atuação na área de software livre estão todas intrinsecamente ligadas.
A estrutura que concebi começa pela minha formação acadêmica no
capítulo~\ref{cap_formacao} e atuação profissional no
capítulo~\ref{cap_atuacao}.
Em seguida, dividi minhas atividades acadêmicas
entre ciência aberta (capítulo~\ref{cap_cienciaaberta}),
linhas de pesquisa (capítulo~\ref{cap_pesquisa})
e ensino e extensão (capítulo~\ref{cap_ensino}).
Algumas informações estão necessariamente repetidas entre alguns capítulos,
por exemplo o software \href{https://www.fatiando.org}{Fatiando a Terra}
é discutido em quase todos os capítulos em diferentes contextos.
Finalmente, apresento considerações finais no capítulo~\ref{cap_conclusao}.


%==============================================================================
\chapter{Formação Acadêmica}
\label{cap_formacao}

\begin{figure}[h]
  \HeroFigPad
  \begin{center}
    \includegraphics[width=\textwidth]{images/vassouras-geomag-observation-2012.jpg}
  \end{center}
  \caption{
    Realizando medidas da direção do campo geomagnético no observatório de
    Vassouras, Rio de Janeiro, em 2012. A atividade foi parte de uma disciplina
    de instrumentação geofísica que cursei durante a pós-graduação do
    \ON{}.
  }
\end{figure}
\begin{summarybox}[frametitle=\faInfoCircle{}\quad Resumo da formação acadêmica]
  \begin{datelist}
    \textbf{2004--2009} & \textbf{Bacharelado em Geofísica --- \USP{}} \\
    \textbf{2010--2011} & \textbf{Mestrado em Geofísica --- \ON{}} \\
    \textbf{2011--2016} & \textbf{Doutorado em Geofísica --- \ON{}}
  \end{datelist}
  \hrule
  \begin{datelist}
    2008--2009 & Intercâmbio Internacional --- York University \\
    2018 & Instructor Training --- The Carpentries \\
    2020--2022 & Postgraduate Certificate in Academic Practice --- \UoL
  \end{datelist}
\end{summarybox}

Este capítulo relata a minha formação acadêmica, do Bacharelado ao Doutorado,
refletindo sobre os fatores que influenciaram minhas linhas de pesquisa e o
rumo que tomei durante minha carreira.

\section{\USP{}}
\label{sec_usp}

\begin{subsummarybox}[frametitle=\faGraduationCap{}\quad Bacharelado em Geofísica]
  \begin{fa-ul}
    \faUniversity & \USP{} \\
    \faCalendar & Fevereiro 2004 -- Novembro 2009 \\
    \faUser & Orientadora: Naomi Ussami\\
    \faInfoCircle & Trabalho de conclusão: Cálculo do tensor gradiente
    gravimétrico utilizando tesseroides (\DOI{10.6084/m9.figshare.963547})
  \end{fa-ul}
\end{subsummarybox}

Ingressei no curso de Bacharelado em Geofísica da \USP{} em
2004.
Já no primeiro semestre, o curso desafiou diversos de meus preconceitos sobre
os assuntos abordados.
Uma das experiências mais marcantes foi a disciplina MAC0115 ``Introdução à
Computação para Ciências Exatas e Tecnologia''.
Minha expectativa era aprender sobre funções avançadas de softwares como o
Microsoft Office, talvez aprender sobre algum programa específico para a
geofísica.
Jamais havia imaginado que como parte do meu curso de Geofísica eu aprenderia
como criar meus próprios programas, mas foi exatamente isso que aprendemos
nessa disciplina que foi ministrada de maneira excepcional.
Minha carreira com certeza teria tomado um rumo completamente diferente se
minha primeira experiência com a programação não houvesse sido tão positiva.
Aprendi os conceitos básicos da linguagem de programação C e, junto com meu
amigo \href{https://www.linkedin.com/in/balancin/}{Lucas Balancin}, resolvi
todos os exercícios fornecidos para estudo da disciplina.
Porém, não alcancei um nível suficientemente avançado para enxergar aplicações
imediatas da programação nas demais disciplinas do curso.

Busquei aprender mais sobre a programação através da disciplina optativa
AGG0204 ``Computação para Geofísicos''.
Durante a disciplina, desenvolvi aplicações diretas da programação à geofísica
como o cálculo do efeito gravitacional de um prisma poligonal 2D.
Essas aplicações me mostraram o enorme poder da programação no aprendizado de
conceitos complexos da geofísica e da matemática.
Ao criar uma implementação computacional de um método, fui levado a considerar
detalhes e a elaborar perguntas que me passariam despercebidas ao estudar
somente pela teoria.
Além disso, também fui capaz de explorar as possibilidades e os limites de uma
teoria de forma dinâmica e independente.

Nos anos seguintes continuei a estudar programação por conta própria nas horas
vagas e a aplicar à geofísica o que estava aprendendo.
Aprendi como programar nas linguagens Java, C++ e Python (por recomendação do
então aluno de mestrado \href{https://www.linkedin.com/in/fspaolo/}{Fernando Paolo}).
Implementei a Transformada Discreta de Fourier\footnote{Disponível em
\url{https://github.com/leouieda/dft-in-c}} para estudar para a disciplina
AGG0330 ``Processamento de Sinais Digitais''.
Utilizei uma implementação do método \textit{Ant Colony Optimization}
\citep{Socha2008}, que fiz por curiosidade própria, para realizar uma inversão
de velocidades de grupo de ondas
Love\footnote{Disponível em \url{https://github.com/leouieda/love-aco-inv}}
como meu projeto para a disciplina AGG0305 ``Teoria de Ondas Sísmicas e
Estrutura da Terra''.
Cursei a disciplina optativa MAC0122 ``Princípios de Desenvolvimento de
Algoritmos'' onde aprendi os conceitos de estruturas de dados e recursividade
que possibilitaram alguns dos avanços que obtive em \citet{Uieda2016}
(seção~\ref{sec_tesseroids}).

O curso também me forneceu treinamento excepcional em quase todos os métodos de
geofísica.
Tivemos experiências de campo e utilizamos uma ampla variedade de equipamentos
geofísicos.
A junção da base teórica sólida com essa experiência prática foi
extremamente motivante para alunos como eu, que estavam indecisos sobre suas
carreiras e sobre qual rumo seguir após a graduação.

Refletindo agora, mais de 20 anos após ingressar na USP, percebo o quão sólida
foi a base que adquiri durante a graduação. Utilizo os conceitos que aprendi
nas disciplinas de computação, álgebra linear, física e métodos potenciais
diariamente. Tendo passado por cinco outras instituições no Brasil e no
exterior, reconheço o quão raro é um curso preparar tão bem seus alunos.
Por isso, sou muito grato a todos os meus professores e ao país por me dar
acesso a essa educação de forma gratuita (outra raridade, principalmente no
exterior).

\subsection{Iniciação científica: Paleomagnetismo}
\label{sec_ic_paleomag}

Durante meu segundo ano de graduação, iniciei um projeto de iniciação
científica com o Professor Manoel Souza D'Agrella Filho.
O objetivo do trabalho era obter um paleo-pólo geomagnético para um conjunto
de diques de idade cambriana da região de Maravilhas, Paraíba.
O projeto intitulado ``Paleomagnetismo e mineralogia magnética dos diques
cambrianos de Maravilhas e Prata (PB)'' foi apoiado por uma bolsa da
FAPESP\footnote{Mais informações em
\url{https://bv.fapesp.br/pt/bolsas/73578/paleomagnetismo-e-mineralogia-magnetica-dos-diques-cambrianos-de-maravilhas-e-prata-pb}}
por um ano.
O trabalho incluiu uma expedição para amostrar novos diques na região de
Monteiro, Paraíba, liderado pelo Professor Ricardo I. F. Trindade.
Os resultados foram apresentados em um poster no XI Simpósio de Iniciação
Científica do IAG/USP \citep{Uieda2006}.
Essa foi a primeira vez que participei de um projeto de pesquisa e apresentei
um poster.
Sou muito grato ao Manuel e o Ricardo pela oportunidade de aprender mais sobre
o paleomagnetismo e pelas experiências de laboratório e de campo.
Percebi com esse projeto que, embora os resultados e sua interpretação tenham
sido muito interessantes, a rotina de laboratório não era algo que eu
conseguiria manter a longo prazo.
Ao mesmo tempo, estava cada vez mais interessado na computação e modelagem
numérica.
Isso me levou a buscar outra área para continuar minha iniciação científica e
trabalho de conclusão de curso.
Mesmo assim, o paleomagnetismo ainda é um assunto que me interessa muito.
Tanto que, 16 anos depois dessa primeira iniciação científica, retornei
ao assunto com uma nova linha de pesquisa em microscopia magnética em
colaboração com o Ricardo \citep{SouzaJunior2024,SouzaJunior2025}
(seção~\ref{sec_micromag}).

\subsection{Iniciação científica: Gravimetria e computação}
\label{sec_ic_tesseroids}

No final de 2007, durante meu terceiro ano de graduação, me juntei ao grupo da
Professora Naomi Ussami para trabalhar em um projeto que abordava os temas que
mais me interessavam naquele momento: computação, modelagem numérica e
gravimetria.
O projeto intitulado ``Modelagem gravimétrica de corpos tesseroidais'' foi
executado com uma bolsa da
SBGf\footnote{Mais informações em \url{https://sbgf.org.br/programa_ic}}
e em colaboração com a Professora Carla Braitenberg da \Trieste{}, Itália.
Nosso objetivo era desenvolver um software que pudesse calcular campos
gravitacionais causado por segmentos de uma esfera (\textit{tesseroides}).
Esse programa seria utilizado para trabalhar com dados da futura missão de
satélite \href{https://www.esa.int/Enabling_Support/Operations/GOCE}{GOCE},
tanto na fase inicial de avaliação de sua sensibilidade a diferentes estruturas
geológicas quanto na fase de processamento e modelagem dos dados obtidos.
Durante as fases iniciais desse projeto, contei com o auxílio da Dra.
Franziska Wild-Pfeiffer, cujo artigo \citep{WildPfeiffer2008} eu estava
tentando reproduzir.
Apresentei meus resultados iniciais no XIII Simpósio de Iniciação Científica do
IAG/USP \citep{Uieda2008}.
No final de 2009, concluí o Bacharelado defendendo o trabalho de conclusão de
curso intitulado ``Cálculo do tensor gradiente gravimétrico utilizando
tesseroides''\footnote{Disponível em \url{https://doi.org/10.6084/m9.figshare.963547}}.

Este trabalho marcou a primeira versão do software Tesseroids
(seção~\ref{sec_tesseroids}), desenvolvido inicialmente na linguagem Python,
e o início de uma linha de pesquisa que abrangeu minha pós-graduação e primeira
coorientação de um aluno de Doutorado (seção~\ref{sec_modelagemdireta}).

\section{York University}
\label{sec_york}

\begin{subsummarybox}[frametitle=\faPlane{}\quad Intercâmbio internacional]
  \begin{fa-ul}
    \faUniversity & York University, Canadá\\
    \faCalendar & Agosto 2008 -- Maio 2009
  \end{fa-ul}
\end{subsummarybox}

Tive a vontade de fazer um intercâmbio no exterior desde o início do curso de
graduação.
Rotineiramente vasculhava as diversas oportunidades divulgadas pela
universidade por uma que oferecesse cursos de Ciências da Terra.
Uma das primeiras que encontrei foi a \href{https://www.yorku.ca/}{York University},
Canadá, cujo curso de Ciências da Terra oferecia diversas disciplinas que
complementariam minha formação na USP, principalmente na área de geodésia.
Me inscrevi no processo seletivo interno da USP para concorrer a uma única vaga
que estava sendo ofertada para alunos de todos os cursos da universidade.
Felizmente fui selecionado e me mudei para Toronto, Canadá, em Agosto de 2008.

Tive uma surpresa ao chegar na York e me apresentar na secretaria de graduação:
o curso de Ciências da Terra havia sido descontinuado no ano anterior por causa
do baixo número de alunos inscritos.
Aparentemente, a página online do curso não havia sido atualizada e por isso
eu baseei meu plano de estudos para o ano em curso inexistente.
Por sorte, a maioria das disciplinas que eu havia escolhido cursar ainda
seriam oferecidas como parte de outros cursos.
Os meus estudos acabaram não sendo tão afetados mas minha experiência não foi
como eu esperava por não ter uma turma de alunos de geociências cursando as
mesmas disciplinas, como era o caso na USP.

Durante minha estadia na York, aprendi sobre sistemas geográficos de
coordenadas, posicionamento, ajustes de redes geodésicas, geodésia física e
levantamentos gravimétricos de alta precisão.
Um destaque dessa experiência foram as aulas do Professor \SpirosLink{}.
Suas aulas de geodésia e matemática forneceram a clareza que me faltava nos
conceitos de anomalias da gravidade e a solução prática de problemas inversos
em geofísica.
Foram as aulas do Prof. Spiros que me proporcionaram a base matemática
necessária para criar o método de \textit{Inversão de Euler} \citep{Uieda2025}.

Meu tempo em Toronto foi excelente para meu crescimento pessoal, cultural e
acadêmico.
Fiz amizade com pessoas de todos os cantos do planeta (Europeus, Asiáticos,
Canadenses) com os quais mantenho contato até hoje.
O conhecimento que adquiri nas disciplinas me possibilitaram começar a
trabalhar diretamente no meu projeto de Mestrado pois já possuía grande parte
da base teórica e experiência prática computacional necessária.
Por isso, fui capaz de desenvolver um método novo em pouco tempo.

\section{\ON{}}
\label{sec_on}

\begin{subsummarybox}[frametitle=\faGraduationCap{}\quad Mestrado em Geofísica]
  \begin{fa-ul}
    \faUniversity & \ON{} \\
    \faCalendar & Fevereiro de 2010 -- Outubro de 2011 \\
    \faUser & Orientadora:  Valéria C. F. Barbosa\\
    \faInfoCircle & Dissertação: Robust 3D gravity gradient inversion by
    planting anomalous densities (\DOI{10.6084/m9.figshare.16882300})
  \end{fa-ul}
\end{subsummarybox}
\begin{subsummarybox}[frametitle=\faGraduationCap{}\quad Doutorado em Geofísica]
  \begin{fa-ul}
    \faUniversity & \ON{} \\
    \faCalendar & Novembro de 2011 -- Abril de 2016 \\
    \faUser & Orientadora:  Valéria C. F. Barbosa\\
    \faInfoCircle & Tese Modelagem direta e inversão de campos gravitacionais em
    coordenadas esféricas (\DOI{10.6084/m9.figshare.16883689}) \\
    \faTrophy & Ganhador do Prêmio SBGf de Melhor Tese de Doutorado (2015--2017)\footnotemark
  \end{fa-ul}
\end{subsummarybox}
\footnotetext{Mais informações em \url{https://sbgf.org.br/premiacoes}}

Minha ida para o Canadá durante a graduação fez com que eu atrasasse minha
formatura em um ano.
Ao retornar, comecei a explorar as opções do que fazer após terminar a
graduação.
Após conversar com meus amigos que já estavam formados e trabalhando em
empresas voltadas à indústria do petróleo no Rio de Janeiro, cheguei à
conclusão de que ainda gostaria de continuar meus estudos e expandir minhas
atividades de pesquisa.
Minha experiência no Canadá me mostrou o quão benéfico é a exposição a uma
diversidade de formas de pensamento que se obtém em diferentes instituições.
Por isso, após cinco anos na USP, decidi que estava na hora de buscar uma
pós-graduação em outra instituição no Brasil.

O \ON{} (ON) já havia despertado meu interesse após uma visita
que fizemos à instituição em 2007 durante uma viagem de nossa turma de
graduação para participar do Congresso Internacional da Sociedade Brasileira de
Geofísica.
Além disso, meu amigo e colega de turma
\href{https://www.pinga-lab.org/people/oliveira-jr.html}{Vanderlei C. Oliveira Jr.}
já havia se formado e estava cursando o Mestrado em Geofísica do Observatório
Nacional (ON) sob supervisão da Professora
\href{https://www.pinga-lab.org/people/barbosa.html}{Valéria C. F. Barbosa}.
Após uma visita ao Rio de Janeiro em 2009, o Vanderlei me convenceu (sem muito
esforço) a me inscrever no Mestrado do ON ao terminar a graduação na USP.
Ele também convenceu a Valéria a me orientar, o que considero ser um dos
maiores favores que um amigo jamais me fez.
Sou eternamente grato ao Vanderlei pela recomendação e à Valéria por aceitar me
orientar.

O ambiente da pós-graduação do ON era extremamente produtivo e estimulante.
As salas misturavam alunos dos diversos grupos de pesquisa da astronomia e
geofísica, facilitando o intercâmbio de ideias entre os alunos.
Por exemplo, aprendi muito sobre o processamento de dados sísmicos e de GPR
ajudando meu amigo e colega de sala
\href{https://www.linkedin.com/in/saulo-siqueira-martins-78770878/}{Saulo Siqueira Martins}
(atualmente Professor de Geofísica da Universidade Federal do Pará)
a utilizar o software \href{https://www.reproducibility.org/}{Madagascar}.
Esse conhecimento foi extremamente útil nas minhas atividades de ensino na
\UERJ{} (capítulo~\ref{cap_ensino}).

A pós-graduação também me forneceu diversas oportunidades de frequentar
congressos internacionais com financiamento da CAPES e de projetos da Valéria.
Essas participações me ajudaram a estabelecer contatos e criar uma rede de
apoio e colaboração internacional.
Por exemplo, os contatos que fiz no congresso
\href{https://conference.scipy.org/scipy2014/}{Scipy} de 2013 e 2014 levaram a
minha participação na diretoria do
\href{https://softwareunderground.org/}{Software Underground}, a organização
de seções em congressos e colaborações com os desenvolvedores do software
\href{https://simpeg.xyz/}{SimPEG}.
O incentivo e a liberdade de escolher meus temas de pesquisa dados pela Valéria
sempre me motivaram a dar o melhor de mim.
Não exagero quando afirmo que conhecer a Valéria foi o acontecimento mais
influente na minha carreira.

Durante a pós-graduação, continuei a perseguir meu interesse na programação,
no software livre e na ciência aberta.
Aprendi como usar o sistema de controle de versão
\href{https://git-scm.com/}{git} e a plataforma
\href{https://github.com}{GitHub} e como criar páginas da internet com HTML e
CSS.
Continuei o desenvolvimento do software \href{https://tesseroids.leouieda.com/}{Tesseroids}
e criei o projeto \href{https://www.fatiando.org}{Fatiando a Terra}
(seção~\ref{sec_fatiando}) junto com alguns colegas da graduação, incluindo o
Vanderlei.
O investimento inicial que fiz na qualidade do código do Fatiando me
permitiu terminar meu projeto de Mestrado em apenas 18 meses,
concluir minha tese de Doutorado enquanto já trabalhava como Professor
Assistente na UERJ (seção~\ref{sec_uerj}) e elaborar aulas interativas sobre
geofísica para meus alunos de geologia.

Em meados de 2013, eu, o Vanderlei e a Valéria iniciamos o grupo de
\href{https://www.pinga-lab.org/}{\textbf{P}roblemas \textbf{In}versos em \textbf{G}eofísic\textbf{a}}
(PINGA).
A conta do grupo no GitHub\footnote{Disponível em \url{https://github.com/pinga-lab}}
agrega os repositórios com o código fonte para reproduzir as publicações do
grupo.
O grupo também conta com uma página na internet\footnote{Página do grupo PINGA: \url{https://www.pinga-lab.org}},
feita em grande parte por mim\footnote{Sou o maior contribuidor em termos de
linhas de código geradas:
\url{https://github.com/pinga-lab/website/graphs/contributors}.},
onde divulgamos as teses, artigos, projetos e integrantes do grupo.

\subsection{Mestrado}

Meu projeto de mestrado era adaptar o método desenvolvido
pela Valéria e seu ex-aluno
\href{https://www.researchgate.net/profile/Fernando-Dias-8}{Fernando Silva Dias}
\citep{SilvaDias2009} para inverter dados de gradiente da gravidade.
Na época, esse tipo de dado estava começando a ser utilizado na área de
recursos minerais mas ainda havia uma falta de métodos de inversão 3D para sua
interpretação.
O projeto estava atrelado ao projeto de Doutorado do aluno
\href{https://www.linkedin.com/in/dionisio-uendro-carlos-093671225/}{Dionisio Uendro Carlos},
que iria utilizar o método desenvolvido por mim para interpretar dados
fornecidos pela empresa \href{https://vale.com/}{Vale}.

A abordagem que eu preferia (e prefiro até hoje) para compreender um assunto
novo é fazer por conta própria a implementação computacional de todos os
conceitos básicos e reproduzir resultados existentes.
Logo, comecei meu Mestrado implementando novamente as rotinas básicas
necessárias para realizar a inversão: o método de modelagem direta de
\citet{Nagy2000}, a geração de dados sintéticos, a solução de problemas
inversos lineares com regularização e a visualização em 3D dos modelos.
Esse código formou a base do projeto
\href{https://www.fatiando.org}{Fatiando a Terra} e ainda sobrevive em partes
de sua encarnação atual (seção~\ref{sec_fatiando}).
Minha vontade era encontrar uma abordagem nova, ao invés de simplesmente seguir
o que já havia sido feito em \citet{SilvaDias2009}.
Sendo uma orientadora consciente, a Valéria corretamente me deu somente até o
final de meu primeiro ano para explorar diferentes opções.
Caso não fosse capaz de desenvolver um método novo, combinamos que eu faria o
projeto inicialmente proposto.
Tendo esse prazo em mente, trabalhei incessantemente durante o ano de 2010 para
desenvolver uma abordagem nova de inversão.

Minha grande descoberta veio quando me deparei com o trabalho de
\citet{Rene1986}.
Este trabalho relativamente desconhecido propôs um método de inversão 2D de
dados de gravidade pouco convencional.
Seu método adiciona elementos iterativamente à solução em torno de ``sementes''
e evita a solução de sistemas lineares, um dos grandes empecilhos
computacionais para a inversão 3D.
Porém, esse trabalho não explorou completamente as vantagens que o conceito de
construir a solução iterativamente possibilitava.
Baseado nas ideias de \citet{Rene1986}, criei um método capaz de inverter de
maneira conjunta dados de gravimetria tradicional e gradiometria gravimétrica
em três dimensões.
Adicionei diversas inovações ao método para torná-lo viável a modelos da ordem
de milhões de elementos e melhor controlar a forma do modelo final.
Essas inovações são resultado direto do meu interesse pela computação, podendo
ser rastreadas às disciplinas que cursei ainda na graduação.
O resultado foi publicado em meu primeiro artigo \citep{Uieda2012}, que formou minha
dissertação e foi apresentado nos congressos internacionais da
Society of Exploration Geophysicists,
European Association of Geoscientists and Engineers,
e Sociedade Brasileira de Geofísica
(seção~\ref{sec_planting}).
Esse artigo também foi meu primeiro experimento em ciência aberta.
Todo o código para produzir os resultados e figuras do artigo foi publicado
em um repositório do GitHub\footnote{Disponível em
\url{https://github.com/pinga-lab/paper-planting-densities}} e o material
suplementar foi publicado no
\href{https://figshare.com/}{figshare}\footnote{Disponíveis em
\url{https://doi.org/10.6084/m9.figshare.91574} e
\url{https://doi.org/10.6084/m9.figshare.91469}}.

Terminei meu mestrado em Outubro de 2011 (quatro meses adiantado) e ingressei
no Doutorado em Geofísica do \ON{}, ainda sob supervisão da
Valéria, logo em seguida.

\subsection{Viagem para Trieste}
\label{sec_triste_carla}

Em 2011, ainda no Mestrado, fui convidado pela Professora Carla Braitenberg
para passar um mês na \Trieste{}, Itália, para continuar
o desenvolvimento do software Tesseroids.
Passei Fevereiro de 2011 trabalhando com ela em uma nova versão do software
escrito em linguagem C.
Na época, produzir um software numérico em Python que pudesse alcançar a
performance de programas escritos em C não era uma tarefa fácil.
Por isso, decidimos que a melhor alternativa seria reescrever o software em C.
Essa nova versão mais eficiente do programa seria necessária para o
processamento de dados do satélite GOCE que o grupo de Trieste almejava fazer.
Durante minha estadia em Trieste, reescrevi o software na linguagem C, criei
uma página para a documentação\footnote{Disponível em \url{https://tesseroids.leouieda.com}}
e desenvolvi um algoritmo de discretização adaptativa para combater o problema
de estabilidade numérica do método\footnote{O \textit{commit} 0af974f
introduziu a discretização adaptativa de tesseroides em 11 de Fevereiro de
2011:
\url{https://github.com/leouieda/tesseroids/commit/0af974f26a15f98f1072ccc6c4ebf29588863f51}}.

\subsection{Doutorado}
\label{sec_doutorado}

Meu projeto de Doutorado era desenvolver métodos para inversão de dados de
gravidade 3D em uma aproximação esférica da Terra, combinando assim os temas
do meu trabalho de conclusão de curso de graduação e dissertação de Mestrado.
A aproximação esférica é necessária para a modelagem em escala continental e
global.
Também decidimos que o desenvolvimento dos softwares Tesseroids e Fatiando a
Terra seriam parte dos objetivos principais da tese.
Esses programas seriam os principais ``produtos'' gerados pelo meu Doutorado
para a comunidade científica.

Os dois primeiros anos do meu Doutorado foram dedicados ao desenvolvimento dos
programas e à colaborações com outros membros do recém-formado
\href{https://www.pinga-lab.org}{PINGA}.
Participei da concepção, execução e escrita dos trabalhos
\citet{OliveiraJr2013}, \citet{Melo2013}, \citet{Carlos2014},
\citet{OliveiraJr2015} e \citet{Carlos2016}.
Expandi a gama de funções disponíveis no Fatiando a Terra\footnote{Ver lista
de mudanças nas versões v0.1 e v0.2 em \url{https://legacy.fatiando.org/changelog.html}}
e apresentei meus trabalhos em diversos congressos internacionais.

No final de 2013, me inscrevi e fui aprovado no concurso público para a vaga de
Professor Assistente no Departamento de Geologia Aplicada da Faculdade de
Geologia da \UERJ{} (seção~\ref{sec_uerj}).
Entre 2014 e 2016, exerci minhas tarefas de docente da UERJ enquanto terminava
os trabalhos \citet{Uieda2016} e \citet{Uieda2017}.
Esses dois anos foram muito desafiadores, principalmente no período de
adaptação ao meu novo cargo de Professor.
Graças ao investimos que havia feito no Fatiando a Terra nos quatro anos
anteriores, fui capaz de desenvolver, aplicar e publicar o método descrito em
\citet{Uieda2017} durante o pouco tempo vago que tive em 2015 e
2016\footnote{O primeiro \textit{commit} do repositório do GitHub do artigo é
de Março de 2015: \url{https://github.com/pinga-lab/paper-moho-inversion-tesseroids/commit/edd0e33a200bd1946be0020a38d1d362d93f2c36}}.
Em Abril de 2016, defendi minha tese de Doutorado intitulada ``Modelagem direta
e inversão de campos gravitacionais em coordenadas esféricas'', composta pelos
trabalhos \citet{Uieda2013}, \citet{Uieda2016} e \citet{Uieda2017}.
Fui ganhador do Prêmio SBGf de Melhor Tese de Doutorado
(2015--2017)\footnote{Mais informações em \url{https://sbgf.org.br/premiacoes}}
e esses três artigos estão entre meus trabalhos com maior número de
citações\footnote{Segundo o Google Scholar em 2025-04-06:
\url{https://scholar.google.com/citations?user=qfmPrUEAAAAJ&hl=en}}.


\section{Formação complementar em pedagogia}

Minha formação no Bacharelado, Mestrado e Doutorado me prepararam bem para uma
carreira de pesquisa.
Porém, senti que ainda havia lacunas no meu treinamento, principalmente na área
de ensino.
Busquei preencher essas lacunas através dos cursos complementares em técnicas
práticas de ensino e teoria pedagógica descritos abaixo.

\subsection{Software Carpentry}
\label{sec_swcarpentry}

\begin{subsummarybox}[frametitle=\faGraduationCap{}\quad The Carpentries Instructor Training]
  \begin{fa-ul}
    \faUniversity & \href{https://carpentries.org/}{The Carpentries} \\
    \faCalendar & 9--10 de Julho de 2018\\
    \faInfoCircle & Habilitação para organizar e ministrar os cursos
    \textit{Software Carpentry}, \textit{Data Carpentry} e
    \textit{Library Carpentry}, incluindo treinamento em pedagogia e práticas
    de ensino de programação e ciência de dados
  \end{fa-ul}
\end{subsummarybox}

Me deparei com o \href{https://software-carpentry.org/}{Software Carpentry}
em 2008 durante meu intercâmbio na York University.
Na época, a organização consistia de uma página na internet com informações
para treinamento de cientistas em técnicas de engenharia de
software\footnote{Infelizmente, a versão do material de 2008 só está disponível
no repositório \url{https://github.com/swcarpentry/v3}}.
Esse material abriu meus olhos para o mundo da engenharia de software que ia
muito além das disciplinas de programação que cursei na USP durante minha
graduação.
Passei grande parte do meu tempo livre durante os meses de inverno no Canadá
imerso no Software Carpentry, aprendendo sobre o sistema de controle de versão
\href{https://subversion.apache.org/}{subversion} (precursor do
\href{https://git-scm.com/}{git}), testes unitários, programação defensiva,
automatização com o \href{https://www.gnu.org/software/make/}{GNU Make},
expressões regulares, programação em
\href{https://www.gnu.org/software/bash/}{bash}, entre outros.
Busquei aplicar esses conceitos novos imediatamente, tanto para as tarefas das
disciplinas que estava cursando quanto para meu trabalho de conclusão de curso
e para o programa Tesseroids.
Utilizo todas a lições que aprendi com o Software Carpentry diariamente na
minha pesquisa, ensino e até mesmo para escrever esse memorial (que está
armazenado em um repositório privado no GitHub e utiliza o Make para compilação
do código \LaTeX{}).

Atualmente, o Software Carpentry é parte da organização sem fins lucrativos
\href{https://carpentries.org/}{The Carpentries}, que promove
internacionalmente cursos de curta duração em engenharia de software para
cientistas.
Os cursos são ministrados, e frequentemente organizados, voluntariamente por
instrutores credenciados.
Em 2018, realizei o curso de habilitação de instrutores do The Carpentries e me
tornei um instrutor
credenciado\footnote{Mais informações em \url{https://carpentries.org/instructors/\#leouieda}}.
O curso cobre técnicas para ensino de programação baseadas em evidências da
literatura pedagógica \citep[resumidas em][]{Brown2018}.
A habilitação me permite organizar e ministrar cursos oficiais do The
Carpentries.

Utilizo as técnicas aprendidas tanto em minhas aulas de programação em Python
como nas aulas de geofísica que possuem uma componente computacional
(capítulo~\ref{cap_ensino}), que são a grande maioria das aulas que dou
atualmente.
A experiência que tive com o uso eficaz de tecnologias para ensino virtual que
foram utilizadas nas etapas finais do curso (Zoom, Google Docs, etc.) foram
extremamente valiosas durante a transição para o ensino online causada pela
pandemia de COVID em 2020 e 2021.


\subsection{Pedagogia no ensino superior}
\label{sec_pgcap}

\begin{subsummarybox}[frametitle=\faGraduationCap{}\quad Postgraduate Certificate in Academic Practice]
  \begin{fa-ul}
    \faUniversity & \UoL{} \\
    \faCalendar & Novembro de 2020 -- Maio de 2022 \\
    \faInfoCircle & Curso de pós-graduação em pedagogia no ensino superior que
    me confere o título de \textit{Fellow of the Higher Education
    Academy} (número de referência PR242069)
  \end{fa-ul}
\end{subsummarybox}

Durante meu segundo ano como Lecturer na \UoL{}, realizei o curso de
pós-graduação
Postgraduate Certificate in Academic Practice (PGCAP) oferecido pela Faculty
of Humanities and Social Sciences da universidade.
A conclusão do PGCAP em 2022 me conferiu o título de \textit{Fellow of the
Higher Education Academy}\footnote{Mais informações em
\url{https://www.advance-he.ac.uk/fellowship/fellowship}},
que é necessário para progressão na carreira acadêmica nas instituições da
Inglaterra.
O curso foi divido em duas partes:
a primeira composta de aulas sobre teoria pedagógica aplicada ao ensino
superior e a segunda composta de um projeto de pesquisa ou revisão em
pedagogia.

Meu projeto para a segunda parte do curso foi uma revisão bibliográfica sobre
a técnica de observação por pares aplicada ao ensino superior
\citep{Cosh1998,Fletcher2018,OKeeffe2021}.
Observação por pares se refere a diversas técnicas que envolvem professores
assistirem e revisarem aulas de outros professores.
Resolvi abordar esse tema após realizar uma sessão de observação por pares
durante a primeira parte do curso.
De todas as atividades que fizemos no PGCAP, essa foi a que mais me beneficiou
e me pareceu ter o maior potencial para difundir boas práticas pedagógicas
entre os professores.
Minha revisão bibliográfica e demais reflexões e notas do curso estão
disponíveis em \citet{PGCAP}\footnote{Recomendo utilizar
a versão em HTML: \url{https://www.leouieda.com/pgcap}}.


%==============================================================================
\chapter{Atuação Profissional}
\label{cap_atuacao}

\begin{figure}[h]
  \HeroFigPad
  \begin{center}
    \includegraphics[width=\textwidth]{images/liverpool-gdsl.jpg}
  \end{center}
  \caption{
    Foto de uma apresentação que fiz para o \textit{Geographic Data Science
    Lab} da \UoL{} em Março de 2020. O propósito da palestra
    foi me apresentar para o grupo pouco após minha chegada em Liverpool e
    tentar estabelecer temas para colaborações futuras.
  }
\end{figure}
\begin{summarybox}[frametitle=\faInfoCircle{}\quad Resumo da atuação profissional]
  \renewcommand{\thempfootnote}{$\dagger$}
  \begin{datelist}
    \textbf{2014--2018} & \textbf{Professor Assistente} --- \UERJ \\
    \textbf{2017--2019} & \textbf{Pesquisador Visitante} --- \UHM, E.U.A. \\
    \textbf{2019--2023} & \textbf{Lecturer} (\textit{Professor Doutor}) --- \UoL{}, Reino Unido \\
    \textbf{2023--atual} & \textbf{Professor Doutor} --- \USP{}
  \end{datelist}
  \hrule
  \begin{datelist}
    2019--2022 & Topic Editor --- Journal of Open Source Software \\
    2019--atual & Advisory Council Member --- EarthArXiv \\
    2020--atual & Fellow --- Software Sustainability Institute \\
    2022--atual & Board Member --- Software Underground \\
    2022--2024 & Advisory Committee Member --- pyOpenSci \\
    2024--atual & Embaixador --- Rede Brasileira de Reprodutibilidade
  \end{datelist}
\end{summarybox}

Este capítulo relata minha atuação profissional, tanto como funcionário em
instituições de ensino superior, quanto como voluntário em posições de
liderança em organizações sem fins lucrativos que servem a comunidade
científica.
Os relatos abaixo se referem somente à atividades institucionais e experiências
pessoais.
Minhas linhas de pesquisa serão discutidas no capítulo~\ref{cap_pesquisa}
e minhas atividades de ensino, orientações e extensão serão discutidas no
capítulo~\ref{cap_ensino}.


\section{\UERJ}
\label{sec_uerj}

\begin{subsummarybox}[frametitle=\faUniversity{}\quad Vínculo institucional]
  \begin{fa-ul}
    \faUser & Professor Assistente \\
    \faMapMarker & Departamento de Geologia Aplicada --- Faculdade de Geologia \\
    \faCalendar & Fevereiro 2014 -- Janeiro 2018\footnotemark{} \\
    \faTrophy & Paraninfo da turma de formandos da Geologia (ano de ingresso 2012)
  \end{fa-ul}
\end{subsummarybox}
\footnotetext{Afastado entre Fevereiro de 2017 e Janeiro de 2018 para trabalhar na \UHM{}}
\begin{subsummarybox}[frametitle=\faList{}\quad Atividades institucionais]
  \begin{datelist}
    2014--2017 & Coordenador: Laboratório de Geofísica de Exploração (LAGEX) \\
    2014--2017 & Coordenador: Projeto Qualitec para contratação de um bolsista de nível superior para atuar no LAGEX \\
    2015--2017 & \textit{Faculty Advisor}: Capítulo Estudantil da Society of Exploration Geophysicists (\textit{UERJ Geophysical Society}) \\
    2015 & Representante docente titular da sub-comissão eleitoral da Faculdade de Geologia
  \end{datelist}
\end{subsummarybox}

No final de 2013, durante meu segundo ano do Doutorado, surgiu a oportunidade
de prestar o concurso público para cargo de Professor Assistente na \UERJ{}
(UERJ).
Somente era necessário o título de Mestre e o concurso era para a área de
Geofísica.
Por recomendação da minha orientadora Valéria C. F. Barbosa, decidi prestar o
concurso pois seria uma excelente oportunidade para iniciar uma carreira
acadêmica antes mesmo de terminar meu Doutorado.
Felizmente, fui aprovado em primeiro lugar no concurso e tomei posse do cargo
de Professor Assistente na UERJ em Fevereiro de 2014.
De início, assumi a posição de coordenador do Laboratório de Geofísica de
Exploração (LAGEX) e fui responsável por duas novas disciplinas de geofísica
do Bacharelado em Geologia e outras disciplinas do Bacharelado em Oceanografia
(seção~\ref{sec_ensino_grad}).

Na coordenação do LAGEX, liderei nossa aplicação para uma chamada de projetos
interna da UERJ (QUALITEC) que forneceria financiamento para a
contratação de um bolsista de nível superior por quatro anos.
Nossa aplicação foi bem sucedida e no final de 2014 nomeei o
\href{https://www.linkedin.com/in/victorxalmeida/}{Victor Thadeu Xavier de Almeida}
para assumir a bolsa.
O Victor era responsável por manter os computadores GNU/Linux do LAGEX,
por auxiliar no ensino de disciplinas de graduação que utilizavam o
laboratório e também por contribuir com o desenvolvimento do Fatiando a Terra.
Ter o Victor no LAGEX durante minha estadia foi excelente e elevou minhas
contribuições de ensino, pesquisa e desenvolvimento do Fatiando.

Ainda em 2014, trabalhei com os alunos
\href{https://www.linkedin.com/in/caroline-adolphsson-61723137/}{Caroline Adolphsson Nascimento}
e \href{https://www.linkedin.com/in/gustavo-pereira-780839111/}{Gustavo do Couto Ramos Pereira}
para fundar um capítulo estudantil da
\href{https://seg.org}{Society of Exploration Geophysicists} (SEG) na UERJ.
Os capítulos da SEG proporcionam diversas oportunidades de desenvolvimento
profissional para os alunos através do financiamento de sua participação no
congresso anual nos E.U.A., campeonatos regionais e ciclos de palestras
internacionais.
O capítulo, denominado ``State University of Rio De Janeiro Geophysical
Society'' foi fundado oficialmente em Janeiro de 2015.

Após terminar meu doutorado em 2016, comecei a cogitar pedir um afastamento de
um ano para fazer um pós-doutorado fora do país.
Isso foi motivado em partes pelo meu cansaço após dois intensos anos
trabalhando em período integral enquanto terminava o Doutorado, mas também em
parte porque minha parceira (e atual esposa)
\href{https://www.acarolcolombo.com/}{Ana Caroline Colombo} iria passar um ano
na \href{https://www.stonybrook.edu/}{Stony Brook University} nos Estados
Unidos como parte de seu doutorado.
A oportunidade de continuarmos no mesmo país surgiu na forma do cargo de
Pesquisador Visitante na
\href{https://www.hawaii.edu/}{\UHM{}} para trabalhar com o Professor Paul
Wessel e a equipe do \GMTLink{} (GMT),
um dos projetos de software livre de maior impacto na geofísica.
Essa oportunidade era muito boa para ser passada e então pedi meu afastamento
da UERJ por um ano a partir de Fevereiro de 2017.

Uma escolha muito mais desafiadora se apresentou em Janeiro de 2018 quando meu
afastamento chegaria ao fim.
Meu envolvimento no GMT estava sendo proveitoso e havia financiamento para me
manter no cargo por mais um ano e meio, com a possibilidade de conseguirmos
mais recursos no futuro.
Ao mesmo tempo, as condições financeiras e sociais no Brasil continuaram a
piorar, principalmente no Rio de Janeiro.
A UERJ se encontrava em grave situação financeira, causando o atraso no
pagamento dos servidores.
A escolha entre a certeza do meu cargo na UERJ e a incerteza de uma posição
temporária nos Estados Unidos não foi fácil.
Por fim, decidi que a melhor escolha para mim e para minha família naquela fase
da nossa vida seria tentar a sorte no exterior e pedir exoneração do cargo da
UERJ.

Minha experiência na UERJ foi positiva e muito educativa.
Avancei minhas linhas de pesquisa e fiz amizades com outros professores e
servidores da Faculdade de Geologia.
Também foi na UERJ que eu tive confirmação de que é na interação com os alunos,
tanto no papel de professor quanto de mentor, onde encontro a maior satisfação
profissional.
Meus esforços foram reconhecidos pelos alunos pois tive a honra de ser
escolhido como paraninfo da turma de formandos da Geologia em 2016 (ano de
ingresso 2012), que foi a primeira turma a qual dei aulas de geofísica.


\section{\UHM}
\label{sec_hawaii}

\begin{subsummarybox}[frametitle=\faUniversity{}\quad Vínculo institucional]
  \begin{fa-ul}
    \faUser & Pesquisador Visitante \\
    \faMapMarker & Department of Earth Sciences --- School of Ocean and Earth Science and Technology\\
    \faCalendar & Fevereiro 2017 -- Agosto 2019
  \end{fa-ul}
\end{subsummarybox}

Comecei a contemplar a possibilidade de fazer um pós-doutorado no exterior
após defender minha tese de doutorado em meados de 2016.
Na busca por oportunidades de financiamento, me inscrevi em todas as listas de
email e classificados que pude encontrar\footnote{Até escrevi um artigo no
meu blog com uma lista desses recursos:
\url{https://www.leouieda.com/blog/job-sites.html}}.
Foi assim que me deparei com um email do Professor
\href{https://www.soest.hawaii.edu/pwessel/}{Paul Wessel} divulgando uma
posição para desenvolver uma ponte entre o software
\GMTLink (GMT)
e a linguagem de programação Python.
Tanto o Paul quanto o GMT são mundialmente famosos e o meu perfil se encaixava
perfeitamente na descrição das qualificações necessárias para a vaga.
Após uma entrevista por vídeo conferência com o Paul e os outros
desenvolvedores do GMT, fui informado de que havia sido selecionado para a
vaga.
Em Fevereiro de 2017 me mudei do Rio de Janeiro para Honolulu, E.U.A., para
começar essa nova etapa.

Conhecer e trabalhar com o Paul foi o destaque da minha estadia na
\UHM{} (UH).
Aprendi muito com ele, não somente sobre desenvolvimento de software mas sobre
como o sistema acadêmico americano funciona, como escrever projetos para
agências de fomento, como ser um líder que eleva as pessoas ao meu redor e como
ser humilde e reconhecer todos os fatores externos que possibilitaram meu
sucesso.
O jeito descontraído, bem humorado e energético do Paul é contagiante.
Sua paixão e brilhantes contribuições para a ciência são fruto de uma vida
fazendo exatamente o que mais gosta \citep{Wessel2024}.

Minha experiência na UH foi diversa, incluindo participações em congressos e
até uma experiência de três dias no navio científico
\href{https://www.soest.hawaii.edu/soestwp/tech/watercraft/kilo-moana/}{R/V Kilo Moana}.
Criei uma rede de colaborares nos E.U.A. através do Paul, principalmente com
o grupo do Professor \href{https://topex.ucsd.edu/sandwell/}{David Sandwell}
do Scripps Institution of Oceanography.
Esse grupo desenvolve o software \href{https://github.com/gmtsar}{GMTSAR} para
processamento de dados de Synthetic Aperture Radar (SAR) e a geração de
interferogramas com a técnica InSAR.
Íamos ao Scripps anualmente para trabalhar com o grupo no software e ajudar a
ministrar o curso de GMT e GMTSAR que era promovido pela organização
\href{https://www.unavco.org/}{UNAVCO} (seção~\ref{sec_workshops}).
Minhas contribuições para o GMT serão discutidas mais adiante na
seção~\ref{sec_gmt}.

O tempo que passei em Honolulu foi inesquecível.
Porém, quando comecei a avaliar as opções para permanecer a longo prazo na UH
ou outra instituição do país, percebi que a carreira acadêmica nos E.U.A. era
excessivamente estressante e incerta.
Como eu e minha esposa sentíamos que ainda não estávamos prontos para retornar
ao Brasil, retomei minha busca por oportunidades de emprego no exterior que
possibilitassem um balanço melhor entre a vida pessoal e profissional.
Foi assim que encontrei um anúncio para uma vaga na área de geofísica na
\UoL{} no Reino Unido.


\section{\UoL{}}
\label{sec_liverpool}

\begin{subsummarybox}[frametitle=\faUniversity{}\quad Vínculo institucional]
  \begin{fa-ul}
    \faUser & Lecturer (\textit{equivalente a Professor Doutor})\\
    \faMapMarker & Department of Earth, Ocean and Ecological Sciences --- School of Environmental Sciences \\
    \faCalendar & Agosto 2019 -- Julho 2023
  \end{fa-ul}
\end{subsummarybox}
\begin{subsummarybox}[frametitle=\faList{}\quad Atividades institucionais]
  \begin{datelist}
    2020--2022 & Comissão para avaliação do website do departamento\\
    2020--2023 & Early Career Academic (ECA) Representative --- Earth Sciences\\
    2022--2023 & Coordenador de curso: Bacharelado em Geofísica e Mestrado em Geologia e Geofísica
  \end{datelist}
\end{subsummarybox}

Com meu financiamento para me manter nos E.U.A. chegando ao fim e um desejo de
continuar no exterior por mais tempo, retomei minha busca por novas
oportunidades de pós-doutorado ou uma posição permanente.
No final de 2018 encontrei a chamada para uma vaga na \UoL{}
de Lecturer (que no Reino Unido é equivalente a Professor Doutor) na área de
geofísica.
Descobri que o curso de geofísica de Liverpool possui uma longa tradição e
que diversos membros do departamento possuem ligações com o Brasil na área de
oceanografia geológica, geomagnetismo e paleomagnetismo.
Por conta desses fatores positivos, apliquei para a vaga e fui chamado para
uma entrevista no início de 2019.
Durante minha primeira viagem a Liverpool, pude confirmar que o departamento
era acolhedor e agradável de se trabalhar.
Mesmo com o \textit{jet lag} severo por conta da diferença de 10 horas entre
Honolulu e Liverpool, fui bem sucedido no processo seletivo e dei início ao meu
cargo de Lecturer em Agosto de 2019.

Ao chegar em Liverpool, fundei o grupo de pesquisa
\href{https://www.compgeolab.org/}{Computer-Oriented Geoscience Lab} (CompGeoLab)
com meu então aluno de doutorado
\SantiagoLink{} e comecei a buscar outros
alunos para se juntarem ao grupo (mais informações sobre orientações na
seção~\ref{sec_orientacao}).
Ministrei um total de sete disciplinas de graduação, incluindo trabalho de
campo de geofísica, programação em Python, sensoriamento remoto e geodinâmica
(seção~\ref{sec_ensino_grad}).
Logo em seguida, assumi o cargo de representante de acadêmicos em início de
carreira (\textit{Early Career Academic Representative}), ou seja, servidores
no nível de Lecturer.
Minha responsabilidades incluíam a organização de eventos para desenvolvimento
profissional, eventos sociais entre departamentos, mentoria de novos Lecturers
e representação da categoria em comissões administrativas da universidade.
Entre 2020 e 2022, participei de uma comissão interna do departamento para
avaliar, organizar e atualizar a página na internet do departamento.
Em 2022, assumi o cargo de coordenador dos cursos de Bacharelado em Geofísica
e Mestrado em Geologia e Geofísica.

Como coordenador, fui responsável por recrutar alunos, participar do processo
selectivo, alocar professores para as disciplinas, revisar a estrutura do
curso, revisar outros cursos da School of Environmental Sciences, organizar
atividades para os calouros, dar apoio aos alunos e lidar com casos
administrativos como transferências de curso, trancamento, etc.
Eu e o coordenador dos cursos de geologia reformulamos a estrutura
dos cursos para modernizá-los e possibilitar mais integração entre as áreas.
Além disso, criamos uma nova especialização em geofísica para o Bacharelado em
Física.
O que mais senti falta durante os dois anos e meio que passei em Honolulu era
o contato direto com os alunos.
Estar novamente em uma posição que me permitiu ensinar e atuar como mentor foi
muito gratificante.
Por isso, eu almejava assumir a posição de coordenador do curso em algum ponto
para ter uma visão mais geral de como os cursos são manejados pela
universidade.
Não esperava que a oportunidade viesse tão cedo (somente por conta de problemas
de saúde do coordenador anterior) mas fiquei contente em assumir a
responsabilidade e poder ter um impacto positivo no curso.

Embora minha experiência em Liverpool tenha sido gratificante e possibilitado
meu crescimento profissional, descobri alguns aspectos da academia na
Inglaterra que a tornam menos atrativa para mim.
Por exemplo, a natureza extremamente comercial do ensino superior e
a dependência exclusiva da conquista de grandes projetos de fomento para
a progressão na carreira.
Somado a isso ainda havia o problema do curso extremamente curto de Bacharelado
de apenas aproximadamente 50 dias de aulas por semestre ao longo de três anos
(comparado com aproximadamente 100 dias por semestre ao longo de quatro ou
cinco anos nas universidades brasileiras).
Este último fator resulta em alunos entrando na pós-graduação com menos
preparo, o que significa que projetos que eu julgaria a nível de doutorado no
Brasil são a nível de pós-doutorado para alunos do Reino Unido.
Também tive minha filha Yara em 2020 e, por conta das restrições em viagens
internacionais, não tive suporte familiar algum.
Percebi então o quanto a proximidade da família contribui para o bem-estar
e o balanço saudável entre o trabalho e a vida pessoal.
Também tive a realização de que a soma de todos esses fatores resultariam,
a longo prazo, em uma produção científica e pedagógica que está muito abaixo
das minhas perspectivas profissionais.
Por isso, decidi que não continuaria na \UoL{} por muito mais tempo.
Mas ainda não sabia qual seria meu próximo passo na carreira.


\section{\USP}
\label{sec_usp_prof}

\begin{subsummarybox}[frametitle=\faUniversity{}\quad Vínculo institucional]
  \begin{fa-ul}
    \faUser & Professor Doutor\\
    \faMapMarker & Departamento de Geofísica --- \IAG{} \\
    \faCalendar & Agosto 2023 -- atual \\
    \faTrophy & Paraninfo da turma de formandos do Bacharelado em Geofísica do 2º semestre de 2024
  \end{fa-ul}
\end{subsummarybox}
\begin{subsummarybox}[frametitle=\faList{}\quad Atividades institucionais]
  \begin{datelist}
    2023--atual & Membro Titular da Comissão de Cooperação Nacional e Internacional\\
    2023--atual & Coordenador do Ciclo de Seminários do Departamento de Geofísica\\
    2023--atual & Representante Titular do IAG no Comitê Gestor do Banco Nacional de Dados Gravimétricos (BNDG) \\
    2023--atual & Membro Suplente da Comissão de Informática\\
    2024--atual & Membro da Comissão de Qualificações do Programa de Geofísica\\
    2024--atual & Membro Suplente da Comissão da Biblioteca\\
    2024--atual & Presidente da Comissão de Cooperação Nacional e Internacional\\
    2025--atual & Representante da categoria MS-3 no Conselho do Departamento de Geofísica
  \end{datelist}
\end{subsummarybox}

No final de 2022, quando estava quase certo de que não continuaria na \UoL{},
recebi uma ligação de meu amigo \VanderleiLink{} do Observatório Nacional me
avisando sobre um concurso público aberto para a área de métodos potenciais no
\IAG{} da \USP{}. Ele até mesmo fez um discurso sobre a importância das
universidades públicas brasileiras para me convencer a me inscrever no
concurso.
Esse incentivo, somado a meu desânimo com as minhas perspectivas na \UoL{}
e com a academia no Reino Unido como um todo, foram motivos mais que
suficientes para me convencer a prestar o concurso e tentar regressar ao Brasil
e à USP após quase sete anos no exterior.
Sou muito grato ao Vanderlei por ter me dado o empurrão que eu precisava para
tomar meu próximo passo.

O ano de 2023 foi um dos mais caóticos e desafiadores da minha vida, somando as
pressões do novo cargo de coordenador do curso de graduação, do concurso da USP,
de novamente atravessar o Oceano Atlântico de mudança (dessa vez com uma
criança de 3 anos de idade) e dos desafios de encontrar um novo lar em São
Paulo e de me adequar ao meu novo cargo de Professor Doutor no IAG. Porém,
também foi um ano de muita felicidade. O acolhimento que recebi de todas as
pessoas do IAG me deu a tranquilidade da certeza de que fiz a escolha certa em
retornar.

Como Professor Doutor do IAG, expandi o \CompGeoLabLink{} com a inclusão de
novos alunos e alunas de pós-graduação e iniciação científica
(seção~\ref{sec_orientacao}), assumi diversas disciplinas a nível de
graduação e pós-graduação (seções~\ref{sec_ensino_grad}
e~\ref{sec_ensino_pos}) e assumi diversos cargos administrativos.
Coordeno os exames de qualificação da pós-graduação em Geofísica junto com
o Prof. Carlos Alberto Mendonça.
Sou o presidente da Comissão de Cooperação Nacional e Internacional (CCNI) do
IAG, na qual estou buscando uma melhor comunicação com estudantes de graduação
para aumentar a procura por intercâmbios e parcerias internacionais.
Como coordenador do ciclo de seminários do Departamento de Geofísica,
implementei mudanças no horário dos seminários e no estilo das palestras,
adicionei uma confraternização antes da palestra e busquei maior engajamento de
estudantes de pós-graduação. Essas mudanças tiveram um impacto positivo no
número de participantes em nosso ciclo de seminários, que estava em declínio
desde a pandemia de COVID.
Também tenho buscado uma maior igualdade de gênero, de temas de pesquisa e de
estágio de carreira na escolha de palestrantes para nossos seminários.
Meu principal objetivo como representante do IAG no Comitê Gestor do
\href{https://www.gov.br/anp/pt-br/assuntos/exploracao-e-producao-de-oleo-e-gas/dados-tecnicos/legislacao-aplicavel/bndg-banco-nacional-de-dados-gravimetricos}{Banco Nacional de Dados Gravimétricos} (BNDG)
é tornar o acesso ao banco de dados mais alinhado com os princípios de dados
FAIR \citep{Wilkinson2016}, que é o padrão internacional para dados abertos.
Mais recentemente, fui eleito como representante (titular) da categoria MS-3 no
Conselho do Departamento de Geofísica junto com meu colega Marcelo Bianchi
(suplente).
É excelente poder participar das atividades do Departamento de Geofísica e me
inteirar de todas as iniciativas e mudanças positivas que tem sido
propostas.


\section{Atuação na Comunidade Científica}
\label{sec_comunidade}

\begin{subsummarybox}[frametitle=\faList{}\quad Resumo das atividades]
  \begin{datelist}
    2019--2022 & Topic Editor --- \href{https://joss.theoj.org/}{Journal of Open Source Software} (ISSN 2475-9066) \\
    2019--atual & Advisory Council Member --- \href{https://eartharxiv.org/}{EarthArXiv} \\
    2020--atual & Fellow --- \href{https://software.ac.uk}{Software Sustainability Institute} \\
    2022--atual & Board Member --- \href{https://softwareunderground.org}{Software Underground} \\
    2022--2024 & Advisory Committee Member --- \href{https://www.pyopensci.org/}{pyOpenSci} \\
    2024--atual & Embaixador --- \href{https://www.reprodutibilidade.org/}{Rede Brasileira de Reprodutibilidade}
  \end{datelist}
\end{subsummarybox}

Além dos vínculos institucionais acima descritos, tenho uma participação
extensa na comunidade científica, principalmente na interseção entre
geociências, infraestrutura digital da ciência (e.g., software livre
científico) e ciência aberta.
Atuei como revisor dos periódicos\footnote{Mais informações em \url{https://www.webofscience.com/wos/author/rid/\ResearcherID}}:
Geophysical Journal International,
Geophysics,
Journal of Geodesy,
Pure and Applied Geophysics,
Journal of Applied Geophysics,
Geophysical Prospecting,
Central European Journal of Geosciences,
Computers and Geosciences
e
Journal of Open Source Software.
Organizei sessões para os congressos internacionais AGU Fall Meeting de 2018 e
2019 e EGU General Assembly de 2021.

\subsection{Journal of Open Source Software}

Em 2019 fui convidado a me juntar ao corpo editorial do
\href{https://joss.theoj.org/}{Journal of Open Source Software}
(JOSS)\footnote{Mais informações em \url{https://joss.theoj.org/about\#editors\_emeritus}}
como Topic Editor na área de geociências.
O JOSS é um periódico que é operado no modelo
\href{https://en.wikipedia.org/wiki/Diamond_open_access}{\textit{diamond open acess}}
onde a publicação é gratuita, os autores retem seus direitos autorais e os
artigos são disponibilizados gratuitamente com uma licença
\href{https://creativecommons.org/licenses/by/4.0/}{Creative Commons Atribution} (CC-BY).
Seu objetivo é fornecer crédito, através de publicações revisadas por pares,
aos cientistas que se dedicam à criação de ferramentas de software livre para
o benefício da comunidade científica.
Necessitei me afastar dessa posição na metade de 2022 quando assumi o cargo de
coordenador dos cursos de geofísica da \UoL{} para dar conta da carga horária
administrativa mais elevada.
Espero poder retornar ao JOSS no futuro.

\subsection{EarthArXiv}

O \href{https://eartharxiv.org/}{EarthArXiv} é um repositório de
\href{https://en.wikipedia.org/wiki/Preprint}{preprints} criado em 2017 e
mantido pela comunidade geocientífica.
Entre 2019 e 2022, servi como membro do
Advisory Council\footnote{Mais informações em \url{https://eartharxiv.github.io/AdvisoryCouncil.html}},
auxiliando na migração do repositório para uma nova plataforma hospedada na
\href{https://cdlib.org/}{California Digital Library} e na avaliação de
submissões antes de serem publicadas.
Retornei ao cargo em 2024 a convite do conselho atual para mais um mandato.

\subsection{Software Sustainability Institute}
\label{sec_ssi}

Em 2020, fui premiado com um \textit{Fellowship} do
\href{https://software.ac.uk/}{Software Sustainability Institute} (SSI) que
inclui a afiliação não remunerada ao instituto e acesso a financiamento para
organizar eventos e atividades relacionadas à missão de aprimorar a criação e
manutenção de software para pesquisa.
Como \textit{Fellow}, eu tenho acesso à rede de contatos do instituto e
participação nos eventos e cursos organizados por eles.
Utilizei meu financiamento para organizar um encontro de geocientistas com
interesse na ciência aberta chamado
\href{https://hackmd.io/@leouieda/uk-geo-code-meetup}{Geo+Code}
(figura~\ref{fig_geocode}),
onde demos início ao desenvolvimento de recursos educacionais abertos para
geofísica aplicada (seção~\ref{sec_openedu}).
O evento contou com a participação de 15 pesquisadores, professores e
profissionais da indústria de nove instituições diferentes do Reino Unido e
Irlanda.

\subsection{Software Underground}
\label{sec_swung}

O \SwungLink{} teve seu início em 2014 como uma lista de
emails\footnote{Ainda disponível em \url{https://groups.google.com/g/softwareunderground}}
para pessoas interessadas em geociências e programação.
Em seguida, a comunidade migrou para a plataforma
\href{https://softwareunderground.org/slack}{Slack} onde cresceu rapidamente,
atualmente contando com mais de 4000 membros.
Em 2020, o Software Underground se tornou uma sociedade profissional sem fins
lucrativos incorporada no Canadá.
Além da plataforma Slack, a nova sociedade organiza eventos online e
presenciais e dá apoio aos projetos de software livre desenvolvidos pela
comunidade (como o \FatiandoLink{}).

Estou envolvido no Software Underground desde seu início.
Ter essa comunidade online ativa foi ainda mais importante durante o isolamento
forçado por conta da pandemia de COVID em 2020 e 2021.
Em 2022 me juntei à diretoria da sociedade como
\textit{Board Member}\footnote{Mais informações em \url{https://softwareunderground.org/board}}.
Meus maiores objetivos como parte da diretoria são estabelecer um mecanismo de
apoio financeiro pra projetos de software livre em geociências e auxiliar na
organização de eventos para a nossa comunidade.


\subsection{pyOpenSci}

A organização \PyOSciLink{} foi fundada em 2019 pela Dra.
\href{https://www.leahwasser.com}{Leah Wasser}.
Baseada no modelo do \href{https://ropensci.org/}{rOpenSci}, a organização tem
como objetivo ajudar cientistas a desenvolverem software livre de qualidade na
linguagem Python.
Conheci a Leah durante um painel sobre dados abertos na AGU Fall Meeting de
2018 e me envolvi nas etapas iniciais do estabelecimento do pyOpenSci através
das sessões de mesa redonda que organizei na AGU Fall Meeting de 2018 e 2019.
Em 2022, me juntei oficialmente ao projeto como
\textit{Advisory Committee Member}\footnote{Mais informações em
\url{https://www.pyopensci.org/our-community/\#emeritus-advisory-council}},
auxiliando na criação das normas para revisão de submissões e no estabelecimento
de parcerias com outras organizações como o Journal of Open Source Software e
o Software Underground.
Minha participação se encerrou oficialmente em 2024 mas continuo interagindo
com a comunidade do pyOpenSci online.


\subsection{Rede Brasileira de Reprodutibilidade}

\begin{figure}[tb!]
  \begin{center}
    \includegraphics[width=\textwidth]{images/rbr-embaixadores-2024-12-04.jpg}
  \end{center}
  \caption{
      Foto dos participantes do evento inaugural do programa de embaixadores da
      \RBRLink{}, realizado no Campus da Praia Vermelha da UFRJ no Rio de
      Janeiro de 2--4 de dezembro de 2024. Participaram do evento a coordenação
      da Rede e os embaixadores e embaixadoras selecionados na chamada de 2024.
  }
  \label{fig_rbr}
\end{figure}

A \RBRLink{} é uma iniciativa multidisciplinar e multi-institucional cujo
objetivo é a ``promoção de práticas de pesquisa transparentes e confiáveis na
comunidade científica brasileira''.
A Rede atua em diversas frentes, aconselhando agências como a CAPES e o CNPq,
para influenciar as políticas de ciência aberta do Brasil.
Em 2024, a Rede deu início a um programa de ``embaixadores'', representantes de
diversas instituições e áreas do conhecimento que são envolvidos com ciência
aberta e reprodutibilidade de estudos científicos.
Apliquei para a posição e fui selecionado para participar do
programa\footnote{Ver a lista de embaixadores em \url{https://www.reprodutibilidade.org/membros}.} começando
em Dezembro de 2024 com um evento inaugural no Rio de Janeiro
(Figura~\ref{fig_rbr}).
Como Embaixador da Rede, realizei o curso sobre reprodutibilidade
``Kit de sobrevivência digital para cientistas''\footnote{Disponível em \url{https://github.com/compgeolab/kit}} na XXVII
Escola de Verão de Geofísica do IAG.
Também planejo dar uma palestra sobre ciência aberta no ciclo de seminários do
Departamento de Geofísica e participar da organização da Semana da Ciência
Aberta da USP em 2025.


\subsection{Participação em bancas}

Tive o prazer de participar de diversas bancas em diferentes instituições ao
longo da minha carreira. Sempre aprendi muito com os trabalhos desses
pesquisadores e pesquisadoras. A participação nas bancas me dá a oportunidade
de ler a fundo trabalhos em diferentes áreas às quais eu normalmente não seria
capaz de dedicar tanto tempo.

\vspace{0.3cm}
\noindent
Concurso público:

\begin{itemize}[leftmargin=1.75cm]
    \item[\bfseries 2023:] Concurso público de Professor Doutor no Departamento
        de Geofísica do IAG/USP na área de Geodinâmica - Edital 024-2023.
        Universidade de São Paulo.
\end{itemize}

\vspace{0.3cm}
\noindent
Doutorado:

\begin{itemize}[leftmargin=1.75cm]
    \item[\bfseries 2023:] Larissa da Silva Piauilino. Tese de Doutorado
        intitulada ``Camadas equivalentes rápidas para o processamento de dados
        de campos potenciais''. Observatório Nacional.
    \item[\bfseries 2022:] Yael Annemiek Engbers. Tese de Doutorado intitulada
        ``Geomagnetic field behaviour in the Miocene and structural
        irregularities in the South Atlantic region''. University of Liverpool.
    \item[\bfseries 2022:] Peter Haas. Tese de Doutorado intitulada ``Linking
        the deep structures of the South American and African cratons by
        satellite gravity data''. Christian Albrechts Universität zu Kiel.
\end{itemize}

\vspace{0.3cm}
\noindent
Mestrado:

\begin{itemize}[leftmargin=1.75cm]
    \item[\bfseries 2016:]
        Natacha Medeiros Rocha. Dissertação de Mestrado intitulada
        ``Optimização da Interpretação Sísmica via aumento de Banda Espectral
        e Inversão Acústica: Aplicação em Caracterização de Reservatórios
        não-convencionais de Hidrocarbonetos''. \UERJ{}.
\end{itemize}

\vspace{0.3cm}
\noindent
Graduação:

\begin{itemize}[leftmargin=1.75cm]
    \item[\bfseries 2024:] Julia Lullis Malvar Fortes. Trabalho de conclusão de
        curso intitulado ``Convecção Térmica no manto terrestre: Soluções
        analíticas e numéricas''. Universidade de São Paulo.
    \item[\bfseries 2024:] Aline Montenegro Araújo. Trabalho de conclusão de
        curso intitulado ``Análise da anisotropia sísmica do manto superior sob
        o Brasil a partir de medições da divisão de ondas XKS''. Universidade
        de São Paulo.
    \item[\bfseries 2016:] Henrique Cavalcanti Pequeno. Trabalho de conclusão
        de curso intitulado ``Superposição de Atributos Sísmicos''. \UERJ{}.
\end{itemize}



%==============================================================================
\chapter{Ciência Aberta}
\label{cap_cienciaaberta}

\begin{figure}[h]
  \HeroFigPad
  \begin{center}
    \includegraphics[width=\textwidth]{images/geopluscode.jpg}
  \end{center}
  \caption{
    Foto do evento \textit{Geo+Code UK} que organizei com meu financiamento
    do \href{https://software.ac.uk/}{Software Sustainability Institute} em
    Novembro de 2022. Durante o evento, demos início à criação de um
    livro texto digital sobre geofísica aplicada que será desenvolvido
    conjuntamente por educadores de diversas instituições do Reino Unido e
    Irlanda.
  }
  \label{fig_geocode}
\end{figure}
\begin{summarybox}[frametitle=\faInfoCircle{}\quad Portfólio de produção em ciência aberta]
  \begin{fa-ul}
    \faUser & Página pessoal: \url{https://www.leouieda.com}
      \\
    \faUsers & Grupo de pesquisa: \url{https://www.compgeolab.org}
      \\
    \faGithub & GitHub: \url{https://github.com/leouieda} \\
    \aiFigshare & figshare: \url{https://figshare.com/authors/Leonardo\_Uieda/97471} \\
    \aiZenodo & Zenodo: \url{https://zenodo.org/communities/compgeolab} \\
    \aiImpactstory & Impactstory: \url{https://impactstory.org/u/0000-0001-6123-9515} \\
    \faYoutube & YouTube: \url{https://youtube.com/LeonardoUieda}
  \end{fa-ul}
\end{summarybox}

Este capítulo relata minhas atividades relacionadas a ciência aberta:
desenvolvimento de software livre, dados abertos, reprodutibilidade e recursos
educacionais abertos.
Essas atividades estão intrinsecamente ligadas às minhas linhas de pesquisa
(capítulo~\ref{cap_pesquisa}) e atividades de ensino
(capítulo~\ref{cap_ensino}).
Porém, decidi dedicar um capítulo a elas pois as considero atividades
complementares e tão importantes quanto publicações e aulas dadas.

\section{Introdução}

Meu primeiro contato com o movimento de \href{https://www.fsf.org/}{software livre}
foi durante meu curso de graduação na \USP{} (seção~\ref{sec_usp}),
onde utilizávamos computadores com o sistema GNU/Linux e o software
\href{https://en.wikipedia.org/wiki/Seismic_Unix}{Seismic Unix} nas nossas aulas.
Fui cativado pelo princípio de garantir a todos a liberdade para modificar e
experimentar com programas e a cultura de se desenvolver produtos para o bem
comum de maneira colaborativa e transparente.
Para mim, esses são os ideais que a ciência representa mas que na prática acabam
não sendo realizados por diversas razões, incluindo a elevada competitividade
e falta de incentivos que dominam a ciência no século XXI.

Desde a elaboração de meu primeiro artigo \citep{Uieda2012}, decidi que iria
sempre buscar atingir esses ideais de transparência e colaboração sem barreiras
em tudo o que faço, mesmo que o resultado disso fosse que meu currículo não
seria bom o suficiente para uma carreira acadêmica.
Felizmente, esse receio inicial não se realizou e percebo hoje as grandes
vantagens em termos de impacto, reputação e oportunidades que essa dedicação me
proporcionou.
Atualmente todos os meus artigos como primeiro autor, e diversos como coautor,
incluem todo o código necessário para reproduzir todos os resultados
apresentados.
Mais que isso, busco utilizar somente dados que estão disponíveis com
licenças abertas (e.g., \href{https://creativecommons.org/licenses/by/4.0/}{CC-BY})
e publicar em acesso aberto para garantir que qualquer pessoa interessada
possa reproduzir meus resultados.
Esses princípios estão descritos de forma mais extensa no manual de
operações\footnote{Disponível em \url{https://www.compgeolab.org/manual}}
do \CompGeoLabLink{}, um documento que criamos para informar novos
colaboradores e membros do grupo sobre nossas expectativas em relação à ciência
aberta.
Essa abordagem se estende ao material didático que desenvolvo para minhas aulas
e quase todos os outros aspectos da minha atuação profissional, incluindo
figuras ilustrativas e apresentações em formato oral e pôster.
Todo esse material pode ser encontrado nas diversas plataformas listadas no
``Portfólio de produção em ciência aberta'' acima.

\vspace{0.5cm}
\begin{subsummarybox}[frametitle=\faInfoCircle{}\quad Apresentações sobre ciência aberta]
  \begin{paperlist}
    2025 &
      \Me.
      How lab handbooks can help shape research culture in your team,
      \emph{Prosper PI Network}.
      \Vimeo{1052328486}.
      \\
    2022 &
      \Me.
      Getting started with Open Science,
      \emph{SPIN SPIN-ITN: Seismological Parameters and Instrumentation}.
      \GitHub{leouieda/2022-05-06-spin-open-science}.
      \\
    2021 &
      \Me, \Santiago.
      Python-based workflows for small-to-medium sized data: what works, what
      doesn't, and what can be improved,
      \emph{AGU Fall Meeting}. \GitHub{compgeolab/agu2021}.
      \\
    ~ &
      \Me.
      Academia e software livre: Desafios e oportunidades no Brasil e no exterior,
      \emph{National Observatory's SEG and EAGE Student Chapter},
      Rio de Janeiro, Brazil.
      \GitHub{leouieda/2021-07-22-on}.
      \YouTube{r2x-DN6laj8}.
      \\
    2020 &
      \Me.
      Geophysical research powered by open-source,
      \emph{Departamento de Geofísica, IAG, \USP{}}.
      \GitHub{leouieda/2020-06-18-usp}.
      \YouTube{VqI8BX1Yg54}.
      \\
    ~ &
      \Me.
      Geophysical research powered by open-source,
      \emph{Christian Albrechts Universität zu Kiel},
      Kiel, Germany.
      \GitHub{leouieda/2020-07-01-kiel}.
      \\
    ~ &
      \Me.
      Geophysical research powered by open-source,
      \emph{Technische Universität Bergakademie Freiberg}.
      \GitHub{leouieda/2020-06-04-freiberg}.
      \\
    ~ &
      \Me.
      Geophysical research powered by open-source,
      \emph{Geographic Data Science Lab, \UoL{}}.
      \GitHub{leouieda/liverpool-gdsl-2020}.
      \\
    2019 &
      \Me.
      Building the foundations for open-source geophysics,
      \emph{\UoL{}}.
      \DOI{10.6084/m9.figshare.10255832}.
      \\
    2017 &
      \Me, \Paul.
      Nurturing reliable and robust open-source scientific software,
      \emph{AGU Fall Meeting}.
      \YouTube{0GO4ZZ5Ry6M}.
  \end{paperlist}
\end{subsummarybox}


\section{Software livre}
\label{sec_software}

O termo \textit{software livre} se refere a programas de computador que
respeitam as liberdades de seus usuários de acessar, reutilizar e modificar o
seu código fonte.
Essas liberdades são geralmente garantidas pelo uso de licenças aprovadas pela
\href{https://opensource.org/}{Open Source Initiative} (OSI).
Desde a graduação, estou envolvido na produção de software livre para uso na
ciência.
Sou o criador dos programas
\href{https://tesseroids.leouieda.com}{Tesseroids},
\href{https://www.fatiando.org}{Fatiando a Terra},
\href{https://www.pygmt.org}{PyGMT} e
\href{https://www.compgeolab.org/xlandsat}{xlandsat}.
Além disso, contribuo com o desenvolvimento de outros projetos de software
livre\footnote{Como pode ser observado pela minha atividade no GitHub:
\url{https://github.com/leouieda}}, principalmente na linguagem Python.

Todos esses projetos são utilizados na minha pesquisa
(capítulo~\ref{cap_pesquisa}) e ensino (capítulo~\ref{cap_ensino}).
Para promover a sinergia entre essas atividades e o desenvolvimento dos
softwares, adotamos a seguinte abordagem no
\href{https://www.compgeolab.org/}{CompGeoLab}:

\begin{itemize}
  \item Todas as inovações metodológicas resultantes da nossa pesquisa devem
    ser incluídas em algum software livre, seja um dos que desenvolvemos
    internamente ou projetos desenvolvidos pela comunidade científica.
  \item Código que é de caráter inovador poderá ser desenvolvido privadamente
    até o momento da publicação do artigo/tese/dissertação. Após a publicação,
    o código deve ser integrado a um projeto de software livre.
  \item Código desenvolvido ao longo da pesquisa que não é de caráter
    inovador (e.g., funções baseadas em trabalhos já publicados ou rotinas
    básicas) deve ser incluído em algum software livre imediatamente.
  \item O software desenvolvido pelo CompGeoLab deve ser distribuído com uma
    licença que facilite seu uso pelo setor privado (e.g.,
    \href{https://opensource.org/licenses/BSD-3-Clause}{BSD} ou
    \href{https://opensource.org/licenses/MIT}{MIT}).
\end{itemize}

Essas regras visam maximizar o impacto de nossa pesquisa, possibilitando o
usufruto de nossas inovações sem restrições para toda a comunidade científica e
o setor privado.
Até o momento, essa abordagem também se mostrou muito vantajosa para a minha
carreira e para as de meus colaboradores.
As publicações que são acompanhadas pelas ferramentas computacionais
\citep[e.g.,][]{Uieda2016,Uieda2017} costumam ser mais citadas que minhas
outras publicações\footnote{Segundo dados da plataforma Google Scholar
\url{https://scholar.google.com/citations?user=qfmPrUEAAAAJ} (acessado em
13/04/2025)}.

Apresento a seguir um resumo da minha produção relacionada a software livre.

\subsection{Tesseroids}
\label{sec_tesseroids}

\begin{figure}[h]
  \SoftwareFigPad
  \begin{center}
    \includegraphics[width=\textwidth]{images/tesseroids.jpg}
  \end{center}
  \caption{Logo do software Tesseroids. A maçã caindo da letra ``T''
  é uma alusão à lei da gravitação de Newton.}
\end{figure}
\begin{summarybox}[frametitle=\faInfoCircle{}\quad Informações sobre o projeto]
  \begin{fa-ul}
    \faLink & Página principal: \url{https://tesseroids.leouieda.com}
    \\
    \faGithub & Código: \url{https://github.com/leouieda/tesseroids}
    \\
    \faGavel & Licença: \href{https://github.com/leouieda/tesseroids/blob/master/LICENSE.txt}{BSD 3-clause}
    \\
    \aiGoogleScholarSquare & 221 citações no \href{https://scholar.google.com/citations?view\_op=view\_citation\&hl=en\&user=qfmPrUEAAAAJ\&citation\_for\_view=qfmPrUEAAAAJ:AXPGKjj\_ei8C}{Google Scholar}\footnotemark{} (acessado em 13/04/2025)
  \end{fa-ul}
\end{summarybox}
\footnotetext{Citações ao trabalho \citet{Uieda2016}.}
\begin{subsummarybox}[frametitle=\faFilePdf{}\quad Artigos publicados]
  \begin{paperlist}
    2016 &
      \Me, \Val, \Carla.
      Tesseroids: Forward modeling gravitational fields in spherical coordinates,
      \emph{Geophysics}, \DOI{10.1190/geo2015-0204.1}.
      \GitHub{pinga-lab/paper-tesseroids}.
  \end{paperlist}
\end{subsummarybox}

O Tesseroids foi meu primeiro projeto de software, o tema do meu
trabalho de conclusão de curso de graduação (seção~\ref{sec_ic_tesseroids})
e um dos capítulos da minha tese de doutorado (seção~\ref{sec_doutorado}).
A primeira versão do software foi feita na linguagem C na forma de programas
de linha de comando.
Cada programa era capaz de ler a geometria dos tesseroides e calcular o
potencial gravitacional ou uma de suas primeiras ou segundas derivadas
espaciais nos pontos especificados pelo usuário.
Como estava aprendendo a linguagem Python, resolvi reescrever o código ainda
durante minha iniciação científica para poder utilizar as diversas bibliotecas
disponíveis na linguagem e evitar o trabalho de compilar o código para
diferentes sistemas operacionais.
Porém, a versão do código em Python era consideravelmente mais lenta que a
versão original em C.
Isso era devido à minha limitação como programador, não às limitações da
linguagem e ferramentas disponíveis na época.
Em 2011, durante meu mestrado, fui convidado pela Professora Carla Braitenberg
para passar um mês na \Trieste{} reescrevendo o software na linguagem C como no
modelo que havia feito inicialmente.
Essa última versão do Tesseroids iniciada em Trieste possui diversas vantagens
sobre as anteriores:

\begin{enumerate}
  \item Execução mais rápida por conta do código optimizado em C.
  \item Documentação na forma de uma página na internet, que aprendi como gerar
    através do meu trabalho no Fatiando a Terra (seção~\ref{sec_fatiando}).
  \item Programas para calcular o feito de prismas retangulares retos em
    coordenadas esféricas, usados para avaliar os resultados obtidos com
    tesseroides.
  \item Programas para auxiliar na geração de modelos topográficos.
  \item Testes unitários para verificar o funcionamento correto do código de
    forma automática\footnote{Testes unitários disponíveis em \url{https://github.com/leouieda/tesseroids/tree/master/test}}.
  \item Distribuição de versões compiladas do código para as plataformas Linux
    e Windows em 32 e 64 bits.
  \item Utilização do serviço \href{https://travis-ci.org/github/leouieda/tesseroids}{TravisCI} de
    \href{https://pt.wikipedia.org/wiki/Integra%C3%A7%C3%A3o_cont%C3%ADnua}{integração contínua}
    para executar os testes unitários automaticamente cada vez que uma mudança
    é feita no código.
  \item Algoritmo de discretização adaptativa dos tesseroides para garantir
    acurácia melhor que 0.1\% dos resultados \citep{Uieda2016}.
\end{enumerate}

Essa última versão do Tesseroids foi descrita no artigo \citet{Uieda2016}, que
é um dos meus trabalhos mais citados\footnote{Segundo a plataforma Google
Scholar
\url{https://scholar.google.com/citations?view_op=view_citation&hl=en&user=qfmPrUEAAAAJ&citation_for_view=qfmPrUEAAAAJ:AXPGKjj_ei8C}
(acessado em 13/04/2025)}.
O desenvolvimento do Tesseroids foi interrompido em 2017 para que eu pudesse me
dedicar mais ao Fatiando a Terra e ao meu novo trabalho na \UHM{}
(seção~\ref{sec_hawaii}).
Porém, o método desenvolvido em \citet{Uieda2016} e aprimorado em
\citet{Soler2019} foi implementado no Fatiando a Terra.
A versão atual do código Python para a modelagem com tesseroides, desenvolvido
pelo meu ex-aluno de doutorado \SantiagoLink{} (seção~\ref{sec_orientacao})
para a biblioteca \href{https://www.fatiando.org/harmonica/}{Harmonica} (parte
do Fatiando a Terra), é mais rápida que a versão em C.


\subsection{Fatiando a Terra}
\label{sec_fatiando}

\begin{figure}[h]
  \SoftwareFigPad
  \begin{center}
    \includegraphics[width=\textwidth]{images/fatiando.jpg}
  \end{center}
  \caption{
    Logo do projeto Fatiando a Terra (meio da figura) e os logos dos softwares
    que atualmente fazem parte do projeto: Pooch, Verde, Harmonica e Boule (da
    esquerda para a direita).
}
\end{figure}
\begin{summarybox}[frametitle=\faInfoCircle{}\quad Informações sobre o projeto]
  \begin{fa-ul}
    \faLink & Página principal: \url{https://www.fatiando.org}
    \\
    \faGithub & Código: \url{https://github.com/fatiando}
    \\
    \faGavel & Licença: \href{https://opensource.org/licenses/BSD-3-Clause}{BSD 3-clause}
    \\
    \aiGoogleScholarSquare & 172 citações no \href{https://scholar.google.com/citations?user=qfmPrUEAAAAJ}{Google Scholar}\footnotemark{} (acessado em 13/03/2025)
  \end{fa-ul}
\end{summarybox}
\footnotetext{Total de citações aos trabalhos \citet{Uieda2013}, \citet{Uieda2018} e \citet{Uieda2020}.}
\begin{subsummarybox}[frametitle=\faFilePdf{}\quad Artigos publicados]
  \begin{paperlist}
    2020 &
      \Me, \Santiago, \Remi, \Hugo, \MattTurk, \Shapero, \Anderson, \Leeman.
      Pooch: A friend to fetch your data files.
      \emph{Journal of Open Source Software}.
      \DOI{10.21105/joss.01943}.
      \GitHub{fatiando/pooch}.
      \\
    2018 &
      \Me. Verde: Processing and gridding spatial data using Green's functions.
      \emph{Journal of Open Source Software}.
      \DOI{10.21105/joss.00957}.
      \GitHub{fatiando/verde}.
  \end{paperlist}
\end{subsummarybox}
\begin{subsummarybox}[frametitle=\faFile{}\quad Trabalhos completos em anais de eventos]
  \begin{paperlist}
    2013 &
      \Me, \Bi, \Val.
      Modeling the Earth with Fatiando a Terra,
      \emph{Proceedings of the 12th Python in Science Conference}.
      \DOI{10.25080/Majora-8b375195-010}.
      \GitHub{leouieda/scipy2013}.
  \end{paperlist}
\end{subsummarybox}
\begin{subsummarybox}[frametitle=\faComment{}\quad Outras apresentações]
  \begin{paperlist}
    2021 &
      \Me, \LLi, \Santiago, \Agustina.
      Design useful tools that do one thing well and work together: rediscovering
      the UNIX philosophy while building the Fatiando a Terra project,
      \emph{AGU Fall Meeting}.
      \GitHub{fatiando/agu2021}.
      \\
    ~ &
      \Me, \Santiago, \Agustina.
      Open-science for gravimetry: tools, challenges, and opportunities,
      \emph{GFZ Helmholtz Centre Potsdam}.
      \GitHub{leouieda/2021-06-22-gfz}.
      \YouTube{z-5dvWfB\_SM}.
      \\
    ~ &
      \Me, \Santiago, \Agustina.
      Fatiando a Terra: Open-source tools for geophysics,
      \emph{Geophysical Society of Houston}.
      \GitHub{fatiando/2021-gsh}.
      \\
    ~ &
      \Me, \Santiago, \Agustina, \LPerozzi, \MWieczorek.
      Harmonica and Boule: Modern Python tools for geophysical gravimetry,
      \emph{EGU General Assembly}.
      \DOI{10.5194/egusphere-egu21-8291}.
      \GitHub{fatiando/egu2021}.
      \\
    2015 &
      \Me.
      Fatiando a Terra: construindo uma base para ensino e pesquisa de geofísica,
      \emph{\USP{}}.
      \DOI{10.6084/m9.figshare.1381870}
      \\
    2014 &
      \Me, \Bi, \Val.
      Using Fatiando a Terra to solve inverse problems in geophysics,
      \emph{Scipy}.
      \DOI{10.6084/m9.figshare.1089987}.
  \end{paperlist}
\end{subsummarybox}

Durante meu curso de graduação, eu e meus colegas
\href{https://www.pinga-lab.org/people/oliveira-jr.html}{Vanderlei C. Oliveira Jr.},
\href{https://www.linkedin.com/in/hbuenos/}{Henrique Bueno dos Santos},
\href{https://www.linkedin.com/in/andr%C3%A9-ferreira-lopes/}{André Lopes Ferreira} e
\href{https://www.linkedin.com/in/josecaparica/}{José Fernando Caparica Jr.}
começamos a planejar o desenvolvimento de um software livre capaz de modelar
todos os tipos de dados geofísicos.
Chamávamos esse projeto ambicioso de ``Fatiando a Terra'' pois nosso objetivo
era modelar a Terra inteira (fatiá-la em polígonos) utilizando todos os dados
disponíveis.
O software seria escrito na linguagem C++ e chegamos até a criar um diagrama
das componentes principais que iríamos implementar\footnote{Esse diagrama ainda
existe no histórico do repositório do GitHub: \url{https://github.com/fatiando/fatiando/blob/10c8ff7c17df53e3e0abd83f1ce8d2a3f6bc57aa/fluxo-simples.pdf}}.
Por razões óbvias, não alcançamos nosso objetivo até o final do nosso curso de
graduação.
Porém, o nome do projeto sobreviveu.
Em 30 de Abril de 2010, no início do
meu mestrado, transformei o Fatiando a Terra em uma biblioteca, chamada
\texttt{fatiando}, escrita na linguagem Python\footnote{O momento exato em que
essa mudança aconteceu está registrado no repositório do GitHub:
\url{https://github.com/fatiando/fatiando/commit/928515b0fcfdccecbc4f661ed2469390ef43ec1d}}.
Meu novo objetivo passou a ser agregar todo o código que estava desenvolvendo
para minha dissertação e para as disciplinas da pós-graduação.
Por conta disso, a biblioteca inclui funções para a modelagem direta e inversão
de diversos métodos geofísicos (e.g., métodos potenciais em 2D e 3D, perfilagem
sísmica vertical, condução de calor geotermal, entre outros).

Uma grande parte do desenvolvimento inicial, incluindo a criação da primeira
versão da página \url{https://www.fatiando.org}, ocorreu em preparo para o
curso ``Tópicos de inversão em geofísica'' que ministrei com o Vanderlei na XVI
Escola de Verão de Geofísica do IAG-USP em 2012 (seção~\ref{sec_workshops}).
O software continuou a crescer durante minha pós-graduação, contando com a
participação de outros 12 desenvolvedores\footnote{Mais informações em
\url{https://github.com/fatiando/fatiando/graphs/contributors}}.
Utilizei o Fatiando como parte integral das minhas aulas de geofísica na
UERJ e contratei o bolsista Victor Thadeu Xavier de Almeida para trabalhar no
desenvolvimento das funções para processamento sísmico (seção~\ref{sec_uerj}).

Em 2016, o aluno \SantiagoLink{} se juntou à equipe de desenvolvimento do
Fatiando como parte de seu projeto de doutorado (seção~\ref{sec_orientacao}).
O Santiago é um programador talentoso e rapidamente aprendeu como participar
do desenvolvimento do Fatiando, criar exemplos para a documentação e atuar como
mentor para novos programadores.
Simultaneamente, o grupo
\href{https://softwareunderground.org/}{Software Underground} estava se
formando (seção~\ref{sec_swung}) e nos conectando com os criadores dos projetos
de software livre
\href{https://simpeg.xyz/}{SimPEG} \citep{Cockett2015},
\href{https://www.pygimli.org}{pyGIMLi} \citep{Rucker2017} e
\href{https://www.gempy.org/}{GemPy} \citep{delaVarga2019},
todos escritos na linguagem Python para modelagem direta e inversão.
Através dessas interações e das conversas semanais que tinha com o Santiago,
percebemos que estava na hora de redefinir os objetivos do Fatiando para
nos alinharmos com esses outros projetos.
Nossa decisão\footnote{Resumida em um artigo publicado no meu blog:
\url{https://www.leouieda.com/blog/future-of-fatiando.html}} foi de
interromper o desenvolvimento da biblioteca \texttt{fatiando} e separar suas
funções em bibliotecas menores com escopos mais bem definidos.
As funções que não estavam sendo utilizadas ou que já existiam em outras
bibliotecas seriam abandonadas.
Essa também seria uma oportunidade para modernizar o nosso código e torná-lo
mais eficiente e fácil de usar.

As novas bibliotecas que são parte do projeto Fatiando a Terra são:

\vspace{0.3cm}
\begin{description}[labelindent=0cm,leftmargin=0.5cm]
    \item[\href{https://www.fatiando.org/verde}{Verde} ---]
        A primeira biblioteca que foi desenvolvida para a nova fase do
        Fatiando. O Verde contém funções e classes para processar e interpolar
        dados distribuídos irregularmente.
    \item[\href{https://www.fatiando.org/harmonica}{Harmonica} ---]
        Nossa biblioteca para processamento e modelagem de dados de métodos
        potenciais. O Harmonica é liderado pelo Santiago e inclui funções para
        modelagem direta, correção topográfica, processamento com fontes
        equivalentes (seção~\ref{sec_eql}) e filtros no domínio da frequência.
    \item[\href{https://www.fatiando.org/boule}{Boule} ---]
        Biblioteca para o cálculo do campo de gravidade gerado por elipsóides
        de referência (i.e., a gravidade normal). As funções e classes do Boule
        eram originalmente parte do Harmonica. O Boule foi criado em
        colaboração com os desenvolvedores do
        \href{https://github.com/SHTOOLS/SHTOOLS}{SHTools}
        \citep{Wieczorek2018} para que pudéssemos utilizar suas funções
        independentemente do Harmonica. O cálculo da gravidade normal em
        qualquer ponto fora do elipsóide é feito através da solução analítica
        de \citet{Lakshmanan1991} e \citet{Li2001}. Logo, a correção de
        ar-livre não é necessária para o cálculo de distúrbios da gravidade.
    \item[\href{https://www.fatiando.org/pooch}{Pooch} ---]
        Uma biblioteca para baixar dados da internet e armazená-lo localmente.
        O Pooch não é diretamente relacionado à geofísica e foi criado em
        colaboração com os desenvolvedores do
        \href{https://github.com/Unidata/MetPy}{MetPy} \citep{May2016}. Durante
        o congresso Scipy de 2018, notamos que diversas bibliotecas em Python,
        incluindo o Verde e o MetPy, possuíam códigos semelhantes para baixar
        dados. Por isso, criamos o Pooch para que todos pudéssemos utilizá-lo
        e eliminar o código repetido.
    \item[\href{https://www.fatiando.org/ensaio}{Ensaio} ---]
        Biblioteca que utiliza o Pooch para baixar os dados abertos que
        utilizamos nos tutoriais e nas documentações dos outros softwares.
    \item[\href{https://www.fatiando.org/choclo}{Choclo} ---]
        A mais recente adição ao Fatiando. O Choclo é desenvolvido e liderado
        pelo Santiago. Essa biblioteca implementa rotinas altamente otimizadas
        para modelagem direta em métodos potenciais. Assim como o Boule,
        o código presente no Choclo era inicialmente parte do Harmonica mas
        está sendo separado para que possa ser usado tanto no Harmonica como no
        SimPEG. Esse trabalho é parte do pós-doutorado que o Santiago está
        fazendo com a Professora \href{https://lindseyjh.ca/}{Lindsey Heagy}
        (uma das criadoras do SimPEG) na University of British Columbia,
        Canadá.
    \item[\href{https://www.fatiando.org/bordado}{Bordado} ---]
        Biblioteca para a geração de coordenadas para malhas regulares
        e perfis. Essas funções faziam parte da biblioteca Verde e foram
        movidas para o Bordado com o objetivo de torná-las mais facilmente
        acessadas por outros projetos. As funções foram todas melhoradas
        e generalizadas para N dimensões.
  \item[\href{https://www.fatiando.org/magali}{Magali} ---]
      Essa é nossa biblioteca para análise e inversão de dados de microscopia
      magnética (seção~\ref{sec_micromag}). Seu desenvolvimento está sendo
      liderado pelo aluno de mestrado \YagoLink{} como parte de sua
      dissertação, com contribuições do aluno de doutorado \GelsonLink{}.
\end{description}
\vspace{0.3cm}

Nossa reestruturação foi acompanhada de um esforço para aumentar o engajamento
e a diversidade de voluntários no projeto.
Eu e o Santiago começamos a orientar e ensinar pessoas interessadas, buscar
ativamente contribuidores em nossas redes sociais e organizar reuniões semanais
para criar uma comunidade em torno do Fatiando.
Nossos esforços foram bem sucedidos e o Fatiando conta hoje em dia com a
participação regular de outras cinco pessoas.

O escopo bem definido de cada uma das bibliotecas (ao invés de ``modelar
toda a Terra'') também contribui para sua adoção pela comunidade científica.
Um exemplo claro de sucesso é o Pooch, que atualmente é utilizado por mais de
500 outros softwares\footnote{Segundo análise do GitHub
\url{https://github.com/fatiando/pooch/network/dependents?dependent_type=PACKAGE} (acessada em 13/04/2025)}.
Como consequência, o Pooch agregou mais de 84 milhões de downloads\footnote{Segundo a página
\url{https://pepy.tech/project/Pooch} (acessada em 13/04/2025)} e 44 pessoas
participaram do seu desenvolvimento\footnote{Segundo a página
\url{https://github.com/fatiando/pooch/graphs/contributors} (acessada em
13/04/2025)}.
Sabemos que as outras bibliotecas também estão sendo utilizadas pela comunidade
pelas mais de 50 mil visualizações anuais das nossas páginas de documentação,
com visitantes originados de todos os continentes (exceto a Antártica)\footnote{Segundo a página
\url{https://plausible.io/fatiando.org} (acessada em 13/04/2025)}.



\subsection{The Generic Mapping Tools}
\label{sec_gmt}

\begin{figure}[h]
  \SoftwareFigPad
  \begin{center}
    \includegraphics[width=\textwidth]{images/gmt.jpg}
  \end{center}
  \caption{Logo do Generic Mapping Tools (GMT), gerado pelo próprio GMT
    utilizando o comando
  \href{https://docs.generic-mapping-tools.org/latest/gmtlogo.html}{\texttt{gmt logo}}.}
\end{figure}
\begin{summarybox}[frametitle=\faInfoCircle{}\quad Informações sobre o projeto GMT]
  \begin{fa-ul}
    \faLink & Página principal: \url{https://www.generic-mapping-tools.org}
    \\
    \faGithub & Código: \url{https://github.com/GenericMappingTools}
    \\
    \faGavel & Licença: \href{https://opensource.org/licenses/LGPL-3.0}{GNU LGPL}
    \\
    \aiGoogleScholarSquare & 2389 citações no \href{https://scholar.google.com/citations?view\_op=view\_citation\&hl=en\&user=qfmPrUEAAAAJ\&citation\_for\_view=qfmPrUEAAAAJ:hkOj\_22Ku90C}{Google Scholar}\footnotemark{} (acessado em 13/04/2025)
  \end{fa-ul}
\end{summarybox}
\footnotetext{Citações ao trabalho \citet{Wessel2019}.}
\begin{summarybox}[frametitle=\faInfoCircle{}\quad Informações sobre o projeto PyGMT]
  \begin{fa-ul}
    \faLink & Página principal: \url{https://www.pygmt.org}
    \\
    \faGithub & Código: \url{https://github.com/GenericMappingTools/pygmt}
    \\
    \faGavel & Licença: \href{https://github.com/GenericMappingTools/pygmt/blob/main/LICENSE.txt}{BSD 3-clause}
    \\
    \aiGoogleScholarSquare & 167 citações no \href{https://scholar.google.com/citations?view\_op=view\_citation\&hl=en\&user=qfmPrUEAAAAJ\&citation\_for\_view=qfmPrUEAAAAJ:-\_dYPAW6P2MC}{Google Scholar}\footnotemark{} (acessado em 13/04/2025)
  \end{fa-ul}
\end{summarybox}
\footnotetext{Citações ao trabalho \citet{Uieda2022}.}
\begin{subsummarybox}[frametitle=\faFilePdf{}\quad Artigos publicados]
  \begin{paperlist}
    2019 &
      \Paul, \Joaquim, \Me, \Remko, \Florian, \Walter, \Dongdong.
      The Generic Mapping Tools, Version 6.
      \emph{Geochemistry, Geophysics, Geosystems}.
      \DOI{10.1029/2019GC008515}.
  \end{paperlist}
\end{subsummarybox}
\begin{subsummarybox}[frametitle=\faInfoCircle{}\quad Apresentações]
  \begin{paperlist}
    2019 &
      \Me, \Paul.
      PyGMT: Accessing the Generic Mapping Tools from Python,
      \emph{AGU Fall Meeting}.
      \DOI{10.6084/m9.figshare.11320280}
      \\
    2018 &
      \Me, \Paul.
      Building an object-oriented Python interface for the Generic Mapping Tools,
      \emph{Scipy}.
      \DOI{10.6084/m9.figshare.6814052}
      \YouTube{6wMtfZXfTRM}
      \\
    ~ &
      \Me, \Paul.
      Integrating the Generic Mapping Tools with the Scientific Python Ecosystem,
      \emph{AOGS $15^{th}$ Annual Meeting}.
      \DOI{10.6084/m9.figshare.6399944}
      \\
    2017 &
      \Me, \Paul.
      A modern Python interface for the Generic Mapping Tools,
      \emph{AGU Fall Meeting}.
      \DOI{10.6084/m9.figshare.5662411}
      \\
    ~  &
      \Me, \Paul.
      Bringing the Generic Mapping Tools to Python,
      \emph{Scipy}.
      \DOI{10.6084/m9.figshare.7635833}
      \YouTube{93M4How7R24}
      \end{paperlist}
\end{subsummarybox}

O \href{https://www.generic-mapping-tools.org}{Generic Mapping Tools} (GMT)
é um dos softwares livres mais utilizados na geofísica.
Ele foi criado na década de 1980 por dois alunos de doutorado do Lamont-Doherty
Earth Observatory, E.U.A.,
\href{https://www.star.nesdis.noaa.gov/star/Smith_WHF.php}{Walter H. F. Smith}
e \href{https://www.soest.hawaii.edu/pwessel/}{Paul Wessel}.
O GMT é um programa de linha de comando (i.e., sem interface gráfica) escrito
na linguagem C.
O programa oferece dezenas de comandos para processar e visualizar dados
geofísicos.
Meu envolvimento com o projeto começou em 2017 quando fui contratado pelo Paul
para criar uma ponte entre o GMT e a linguagem Python (seção~\ref{sec_hawaii}).
O resultado desse meu trabalho foi a criação do software
\href{https://www.pygmt.org}{PyGMT}.

Meu primeiro desafio para tornar o GMT acessível da linguagem Python foi
realizar a compilação do software de maneira compatível com a bibliotecas
científicas do Python (numpy, scipy, etc.).
Isso foi possível graças à plataforma
\href{https://conda-forge.org/}{Conda-Forge}, que automatiza a compilação e
distribuição de software, e a ajuda imensa do
\href{https://github.com/ocefpaf}{Filipe Fernandes}, um dos líderes do
Conda-Forge e um brilhante oceanógrafo e programador brasileiro.
Gastei meus primeiros seis meses de trabalho para superar esse desafio.
Em seguida, dei início ao desenvolvimento do código em Python que seria capaz
de executar rotinas da biblioteca em C do GMT.
Para isso, utilizei uma tecnologia chamada \textit{C foreign function
interface} (C FFI) que permite a interação de bibliotecas em C diretamente com
outras linguagens.
Construir essa interface foi um trabalho árduo mas que formou o núcleo que o
PyGMT usa para se comunicar com o GMT.
Meu investimento valeu a pena pois o PyGMT depende desse núcleo até hoje com
poucas modificações nos últimos anos.
Inicialmente, optei por concentrar meus esforços nessa parte do código que é
complexa e requer conhecimento profundo do GMT, bibliotecas em C e funções
avançadas em Python.
Também investi muito do meu tempo criando documentação, incluindo um guia para
desenvolvedores, e tornando o processo de desenvolvimento do PyGMT
automatizado e simples.
Tomei essas decisões para facilitar ao máximo o envolvimento futuro de novos
desenvolvedores voluntários no projeto, quebrando algumas das barreiras que
normalmente impossibilitam a participação de programadores novatos.
Como resultado disso, o projeto agora conta com a participação de oito outros
desenvolvedores e mais de 40 contribuidores esporádicos\footnote{Uma lista dos
principais desenvolvedores está disponível em \url{https://www.pygmt.org/latest/team.html}
e uma lista dos contribuidores está disponível em \url{https://github.com/GenericMappingTools/pygmt/graphs/contributors}}.
Sou muito grato à dedicação do
\href{https://github.com/seisman}{Dongdong Tian},
\href{https://weiji14.github.io/}{Wei Ji Leong} e
\href{https://github.com/maxrjones}{Max Jones}
que assumiram posições de liderança no projeto quando meu envolvimento diminuiu
em 2019 ao me mudar para Liverpool.
Tenho muito orgulho de dizer que o PyGMT continua crescendo e evoluindo sem
minha participação direta no seu desenvolvimento.
Considero o estabelecimento da comunidade que se formou em torno do PyGMT a
minha maior conquista relacionada a software livre.

Além do meu trabalho no PyGMT, também fui responsável pela transição do
desenvolvimento do GMT para a plataforma GitHub (motivada pela falha do
servidor utilizado anteriormente durante um encontro de desenvolvedores), a
criação da atual página do projeto \url{https://www.generic-mapping-tools.org},
a criação do fórum \url{https://forum.generic-mapping-tools.org}
e a modernização e automatização da compilação da página de documentação do
GMT\footnote{Disponível em \url{https://docs.generic-mapping-tools.org}}.
Também fui responsável pela escrita e coordenação, junto com o Paul,
de dois projetos financiados pela \href{https://www.nsf.gov/}{National Science
Foundation} (NSF)\footnote{Disponíveis em
\url{https://www.nsf.gov/awardsearch/showAward?AWD_ID=1829371} e
\url{https://www.nsf.gov/awardsearch/showAward?AWD_ID=1948602}}.
O objetivo principal desses projetos era estabelecer um plano para o futuro
do GMT sem tanto envolvimento direto do Paul, que estava perto da aposentadoria.
Com esse financiamento, pudemos atualizar a documentação do GMT e torná-la mais
acessível, promover eventos para recrutar desenvolvedores e treinar usuários
(seção~\ref{sec_workshops}), organizar encontros dos desenvolvedores e
contratar o Doutor \href{https://github.com/maxrjones}{Max Jones} para
trabalhar no GMT e PyGMT.
Como resultado, o time de desenvolvedores do GMT conta agora com mais três
pessoas, envolvidas não só na programação mas também na organização de eventos
e divulgação do projeto.

Tragicamente, o líder do projeto, Paul Wessel, faleceu em março 2024 após uma
longa batalha contra o câncer.
A perda do Paul foi um evento trágico para o projeto e para todos os
integrantes.
Paul foi um mentor incrível e um querido amigo.
Os próximos anos serão muito difíceis para o GMT e teremos muitos desafios. Com
certeza, o desenvolvimento do programa acontecerá em um ritmo muito menor de
agora em diante.
Porém a comunidade está dedicada a continuar o legado do Paul.
Somos todos gratos por termos tido a chance de encontrá-lo em sua casa na
Noruega para um último \textit{GMT Summit} em julho de 2023
(Figura~\ref{fig_gmt}).

\begin{figure}[tb!]
  \begin{center}
    \includegraphics[width=\textwidth]{images/gmt-summit-2023-07.jpg}
  \end{center}
  \caption{
      Foto dos participantes do último GMT Summit na Noruega em julho de 2023.
      Da esquerda para a direita: Joaquim Luis, Remko Scharroo, Paul Wessel,
      Kristof Koch, Walter Smith, Federico Esteban, Roger Davis, eu.
  }
  \label{fig_gmt}
\end{figure}

\subsection{xlandsat}

\begin{summarybox}[frametitle=\faInfoCircle{}\quad Informações sobre o projeto]
  \begin{fa-ul}
    \faLink & Página principal: \url{https://compgeolab.org/xlandsat}
    \\
    \faGithub & Código: \url{https://github.com/compgeolab/xlandsat}
    \\
    \faGavel & Licença: \href{https://github.com/compgeolab/xlandsat/blob/main/LICENSE.txt}{MIT}
  \end{fa-ul}
\end{summarybox}

Este é o mais recente software que foi criado no âmbito do
\href{https://www.compgeolab.org}{CompGeoLab}, tendo sido iniciado em Dezembro
de 2022.
O xlandsat é uma biblioteca feita para facilitar o processamento e visualização
de dados de sensoriamento remoto dos satélites
\href{https://en.wikipedia.org/wiki/Landsat_program}{Landsat 8 e 9} da
NASA e da USGS.
A biblioteca é capaz de ler os dados no formato do repositório
\href{https://earthexplorer.usgs.gov/}{EarthExplorer} e organizá-los em
estruturas de dados da biblioteca \href{https://xarray.dev/}{xarray}
\citep{Hoyer2017}, uma das ferramentas mais utilizadas para processamento de
dados geocientíficos.

A criação do xlandsat foi motivada pelas minhas aulas de sensoriamento
remoto na disciplina ``ENVS258 Environmental Geophysics'' da \UoL{}
(seção~\ref{sec_ensino_grad}).
Antes de ministrar essa disciplina, meu conhecimento de processamento de
imagens de satélite era mínimo.
Ao longo do preparo de meu material didático, aprendi muito sobre o assunto.
Criei uma coleção de funções escritas na linguagem Python para auxiliar meus
alunos a processarem os dados baixados do
\href{https://earthexplorer.usgs.gov/}{EarthExplorer} em seus relatórios.
Em 2022, decidi que estava na hora de organizar esse código em uma biblioteca
que poderia ser utilizada por meus alunos na disciplina em 2023.
Além disso, o sensoriamento remoto despertou meu interesse acadêmico e
eu necessitava de uma maneira fácil de explorar as possíveis aplicações e
limitações desses dados.
Por exemplo, utilizei o xlandsat para criar uma visualização da erupção de
Dezembro de 2022 do vulcão Mauna Loa, Havaí (figura~\ref{fig_maunaloa}).

\begin{figure}[tb]
  \begin{center}
    \includegraphics[width=\textwidth]{images/mauna-loa-landsat-2022-12-02.jpg}
  \end{center}
  \caption{
    Imagem da erupção de Dezembro de 2022 do vulcão Mauna Loa, Havaí, composta
    pelas bandas infravermelhas do satélite Landsat 9.
    O vulcão está cercado por nuvens (branco e azul claro). O fluxo de lava
    atual está na direção Sul/Norte em vermelho e verde. A cratera principal
    pode ser vista no centro da imagem em vermelho escuro.
    As manchas em marrom e preto são fluxos de lava de erupções anteriores.
    Também está visível a cratera Hale Ma'uma'u do vulcão Kīlauea no canto
    inferior direito da imagem.
    O código Python para reproduzir essa imagem está disponível em um artigo na
    minha página pessoal \url{https://www.leouieda.com/blog/mauna-loa.html} e
    também no repositório do GitHub
    \url{https://github.com/compgeolab/mauna-loa-landsat-2022}.
    Fonte da imagem: \citet[][CC0]{Uieda2022maunaloa}.
  }
  \label{fig_maunaloa}
\end{figure}


\section{Recursos educacionais abertos}
\label{sec_openedu}

\begin{subsummarybox}[frametitle=\faFilePdf{}\quad Artigos publicados em revistas]
  \begin{paperlist}
    2017 &
      \Me.
      Step-by-step NMO correction,
      \emph{The Leading Edge},
      \DOI{10.1190/tle36020179.1}.
      \GitHub{pinga-lab/nmo-tutorial}.
      \\
    2014 &
      \Me, \Bi, \Val.
      Geophysical tutorial: Euler deconvolution of potential-field data,
      \emph{The Leading Edge},
      \DOI{10.1190/tle33040448.1}.
      \GitHub{pinga-lab/paper-tle-euler-tutorial}.
  \end{paperlist}
\end{subsummarybox}
\begin{subsummarybox}[frametitle=\faBook{}\quad Recursos computacionais]
  \begin{paperlist}
    2025 &
      \Me, \Arthur, \Yago. Kit de sobrevivência digital para cientistas.
      \GitHub{compgeolab/kit}.
      \\
    2021 &
      \Me. A quick introduction to machine learning.
      \GitHub{leouieda/ml-intro}.
      \\
    2020 &
      \Me. Introduction to lithosphere dynamics.
      \GitHub{leouieda/lithosphere}.
      \\
    2020 &
      \Me. Introduction to remote sensing.
      \GitHub{leouieda/remote-sensing}.
      \\
    2015 &
      \Me. Matemática Especial 1: Introdução à computação e métodos numéricos.
      \GitHub{mat-esp/about}.
      \\
    2015 &
      \Me. Geofísica 2: Sismologia e métodos eletromagnéticos.
      \GitHub{leouieda/geofisica2}.
      \\
    2015 &
      \Me. Geofísica 1: Gravimetria e magnetometria.
      \GitHub{leouieda/geofisica1}.
  \end{paperlist}
\end{subsummarybox}
\begin{subsummarybox}[frametitle=\faYoutube{}\quad Apostilas]
  \begin{paperlist}
    2012 &
      \Bi, \Me. Tópicos de inversão em geofísica.
      \DOI{10.6084/m9.figshare.1192984}.
      \GitHub{pinga-lab/inverse-problems}.
  \end{paperlist}
\end{subsummarybox}
\begin{subsummarybox}[frametitle=\faYoutube{}\quad Vídeos]
  \begin{paperlist}
    2022 & A geophysical tour of mid-ocean ridges. \YouTube{NzJmRlJCNbQ}
      \\
    2022 & Anatomy of a PyGMT figure. \YouTube{96\_reU\_yh5I}
      \\
    2021 & Downloading Landsat 8 images from USGS EarthExplorer. \YouTube{Wn\_G4fvitV8}
      \\
    2021 & Searching on Google for openly licensed images. \YouTube{ISu51NB5Z28}
      \\
    2020 & From scattered data to gridded products using Verde. \YouTube{-xZdNdvzm3E}
  \end{paperlist}
\end{subsummarybox}

O termo ``recursos educacionais abertos'' (REA ou \textit{open educational
resources} em inglês) foi estabelecido pela UNESCO\footnote{Mais informações em
\url{https://www.unesco.org/en/open-educational-resourcess}} para se referir a
qualquer material destinado ao ensino e aprendizagem que
esteja no domínio publico ou sob direitos autorais regidos por uma licença
aberta que permita acesso gratuito, reutilização, adaptação e redistribuição do
material
(e.g., \href{https://creativecommons.org/licenses/by/4.0/}{Creative Commons Attribution}).
No Brasil, o uso obrigatório de REAs foi adotado pelo Sistema Universidade
Aberta do Brasil em 2016\footnote{Segundo a página
\url{https://www.gov.br/capes/pt-br/acesso-a-informacao/acoes-e-programas/educacao-a-distancia/universidade-aberta-do-brasil/recursos-educacionais-abertos/}
(acessada em 11/02/2023)}
e a CAPES criou o portal \url{https://educapes.capes.gov.br} para indexar REAs
produzidos por instituições brasileiras que oferecem cursos a distância.
Segundo a definição acima, produzo recursos educacionais abertos desde minha
primeira experiência de ensino em 2012 e a criação da apostila
``Tópicos de inversão em geofísica'' \citep{OliveiraJr2012}.
Todo o material que crio para uso nas minhas disciplinas e cursos de curta
duração estão disponíveis livremente com licenças
\href{https://creativecommons.org/licenses/by/4.0/}{Creative Commons Attribution}
ou
\href{https://opensource.org/licenses/BSD-3-Clause}{BSD}/\href{https://opensource.org/licenses/MIT}{MIT}
(para o código fonte).
Também sou o autor de dois tutoriais publicados na revista
\href{https://library.seg.org/journal/leedff}{The Leading Edge} que visam
explicar de maneira interativa conceitos básicos de geofísica.

Acredito que todo material educacional produzido por instituições públicas deve
ser disponibilizado livremente para benefício da população.
Além disso, compartilhar recursos educacionais entre professores e instituições
tem o potencial de elevar o ensino de todos os envolvidos.
A colaboração na produção de recursos possibilita a criação de material de
qualidade superior do que poderia ser atingida por uma única pessoa.
Essa cultura de colaboração em recursos abertos, como é feito no âmbito
de software livre, não é comum no ensino superior.
Por isso, organizei o evento
\href{https://hackmd.io/@leouieda/uk-geo-code-meetup}{Geo+Code} em Novembro de
2022 com meu financiamento do Software Sustainability Institute
(seção~\ref{sec_ssi}).
O principal objetivo do evento era juntar geocientistas do Reino Unido com um
interesse em ciência aberta e dar início a colaborações.
Durante o evento, demos início a criação de um livro aberto sobre geofísica
aplicada utilizando recursos computacionais e dados abertos.
O livro, ainda em estágio de planejamento, será desenvolvido no repositório
do GitHub \url{https://github.com/GeophysicsLibrary/applied-geophysics},
hospedado na organização \href{https://github.com/GeophysicsLibrary}{Geophysics Library}
que fundei em 2018 para agregar REAs voltados à geofísica.
No futuro, pretendo investir na criação de livros abertos na Geophysics
Library.
Pretendo utilizar para isso o material didático que desenvolvi para minhas
disciplinas da USP, \UoL{} e da UERJ.


%==============================================================================
\chapter{Linhas de Pesquisa}
\label{cap_pesquisa}

\begin{figure}[h]
  \HeroFigPad
  \begin{center}
    \includegraphics[width=\textwidth]{images/australia-ground-gravity-disturbance.jpg}
  \end{center}
  \caption{
    Compilação de dados terrestres de distúrbio da gravidade da Austrália.
    Distribuídos originalmente por \citet{Wynne2018}. Compilados e padronizados
    por \citet{Uieda2021} para facilitar o seu uso em diversas linhas de
    pesquisa.
  }
\end{figure}
\begin{summarybox}[frametitle=\faInfoCircle{}\quad Resumo das atividades]
  \begin{fa-ul}
    \faSearchDollar & Projetos financiados pelas agências: National Science
      Foundation (E.U.A.), Royal Society (Reino Unido) e Software Sustainability
      Institute (Reino Unido)\\
    \faFilePdf & 18 artigos publicados em revistas, 1 artigo aceito para
    publicação, 1 artigo em revisão como preprint, 11 trabalhos completos em
    anais de eventos\footnotemark[1] \\
    \faComment & 41 apresentações de trabalho, sendo 17 dessas convidadas\footnotemark[1] \\
    \aiGoogleScholarSquare & 3760 citações no
    \href{https://scholar.google.com/citations?user=qfmPrUEAAAAJ}{Google Scholar}
    e 2279 no \href{https://www.webofscience.com/wos/author/record/1766625}{Web of Science}
    (acessados em 13/04/2025)
  \end{fa-ul}
\end{summarybox}
\footnotetext[1]{O número de total trabalhos e apresentações pode ser diferente
das quantidades listadas abaixo. Alguns trabalhos e apresentações estão
listados em outras áreas de atuação (e.g., capítulo \ref{cap_cienciaaberta}) ou
pertencem a mais de uma linha de pesquisa.}


Ao longo de minha carreira, sempre busquei temas de pesquisa nos quais achava
que meu interesse em combinar a geofísica com a programação poderia ter um
maior impacto no avanço da ciência.
Creio que isso se reflete no número de citações que meus trabalhos geralmente
recebem, que é considerado alto para a área\footnote{Segundo análise da
  plataforma Dimensions. Por exemplo
  \url{https://badge.dimensions.ai/details/id/pub.1019631868} \citep{Uieda2012},
  \url{https://badge.dimensions.ai/details/id/pub.1064143907} \citep{Uieda2016} e
  \url{https://badge.dimensions.ai/details/id/pub.1059638400} \citep{Uieda2017}.
}.
Este capítulo é uma reflexão da minha produção científica do ponto de vista das
diferentes linhas de pesquisa que desenvolvi, apresentadas abaixo em ordem
cronológica com as últimas sendo as linhas mais recentemente abertas.

\section{Modelagem direta de campos gravitacionais em escala global}
\label{sec_modelagemdireta}

\begin{summarybox}[frametitle=\faInfoCircle{}\quad Resumo da linha de pesquisa]
  \begin{fa-ul}
    \faFilePdf & 3 artigos publicados \\
    \faFile & 1 trabalho completo em anais de eventos \\
    \faComment & 3 apresentações de trabalho \\
    \faUserGraduate & Alunos envolvidos: Santiago Soler (PhD), Mustafa Alordowny (BSc), Felipe Nascimento Hong (IC) \\
    \faGlobeAmericas & País dos colaboradores: Brasil, Argentina, China, Itália, Alemanha, Canadá
  \end{fa-ul}
\end{summarybox}
\begin{subsummarybox}[frametitle=\faFilePdf{}\quad Artigos publicados]
  \begin{paperlist}
    2019 & \Santiago, \Agustina, \Gimenez, \Me.
      Gravitational field calculation in spherical coordinates using variable
      densities in depth.
      \emph{Geophysical Journal International}.
      \DOI{10.1093/gji/ggz277}.
      \GitHub{pinga-lab/tesseroid-variable-density}.
      \Preprint{10.31223/osf.io/3548g}.
      \Data{10.6084/m9.figshare.8239622}.
      \\
    ~ & \Guangdong, \Bo, \Me, \JLiu, \MKaban, \LChen, \RGuo.
      Efficient 3D large-scale forward-modeling and inversion of gravitational fields in
      spherical coordinates with application to lunar mascons.
      \emph{Journal of Geophysical Research: Solid Earth}.
      \DOI{10.1029/2019jb017691}.
      \Preprint{10.31223/osf.io/dzf9j}.
      \Data{10.6084/m9.figshare.7300523}.
      \\
    2016 & \Me, \Val, \Carla.
      Tesseroids: Forward modeling gravitational fields in spherical coordinates,
      \emph{Geophysics}, \DOI{10.1190/geo2015-0204.1}.
      \GitHub{pinga-lab/paper-tesseroids}.
  \end{paperlist}
\end{subsummarybox}
\begin{subsummarybox}[frametitle=\faFile{}\quad Trabalhos completos em anais de eventos]
  \begin{paperlist}
    2011 & \Me, \Everton, \Carla, \Eder.
      Optimal forward calculation method of the Marussi tensor due to a geologic
      structure at GOCE height,
      \emph{Proceedings of the 4th International GOCE User Workshop}.
      \DOI{10.6084/m9.figshare.92624}.
      \GitHub{leouieda/goce2011}.
  \end{paperlist}
\end{subsummarybox}
\begin{subsummarybox}[frametitle=\faComment{}\quad Outras apresentações]
  \begin{paperlist}
    2010 & \Me, \Naomi, \Carla.
      Computation of the gravity gradient tensor due to topographic masses
      using tesseroids,
      \emph{AGU Meeting of the Americas},
      Foz do Iguaçu, Brazil.
      \DOI{10.6084/m9.figshare.156858}
      \\
    2008 & \Me, \Naomi.
      Utilização de tesseróides na modelagem de dados de gradiometria
      gravimétrica,
      \emph{XIII Simpósio de Iniciação Científica do IAG-USP},
      São Paulo, Brazil.
      \DOI{10.6084/m9.figshare.4779760}
  \end{paperlist}
\end{subsummarybox}

Esta foi minha primeira linha de pesquisa, iniciada durante meu trabalho de
conclusão de curso de Bacharelado em Geofísica na \USP{}
(seção~\ref{sec_usp}).
A motivação inicial para explorar essa linha foi o lançamento do satélite
GOCE, que efetuou medidas dos gradientes da gravidade em uma resolução espacial
sem precedentes.
Seria necessária uma ferramenta computacional que pudesse modelar as medições
que seriam feitas pelo GOCE para melhor compreender suas limitações e processar
os dados quando estivessem disponíveis.
Métodos numéricos para a solução das integrais de Newton do campo gravitacional
de um prisma esférico (tesseroide) já existiam
\citep{Heck2006,Asgharzadeh2007,WildPfeiffer2008} mas nenhum havia lidado com o
problema da acurácia variável com a distância entre os prismas e os pontos de
observação.
Isso me proporcionou a chance de trazer uma perspectiva diferente para a área.
Esta linha foi a minha introdução à pesquisa com colaborações internacionais.
Também foi a minha primeira experiência na orientação de alunos através da
minha coorientação do Santiago.
Mais recentemente, o aluno Felipe Nascimento Hong se juntou ao grupo em um
projeto de iniciação científica com o objetivo de utilizar novas soluções
analíticas para o campo gravitacional de um tesseroide \citep{Deng2023} para
realizar correção topográfica de dados gravimétricos terrestres.

\begin{fancyenum}{\faBullseye}{Objetivos}
  \item Desenvolver métodos numéricos para calcular o campo gravitacional e
    suas derivadas espaciais causados por prismas esféricos (tesseroides) de
    maneira computacionalmente eficiente e acurada.
  \item Utilizar os métodos desenvolvidos para modelar estruturas geológicas em
    escala continental e global.
  \item Disponibilizar ferramentas de software livre que implementam os métodos
    para o uso da comunidade científica.
\end{fancyenum}

\begin{fancyenum}{\faLightbulb}{Principais contribuições}
  \item Desenvolvimento de um algoritmo de discretização adaptativa dos
    tesseroides capaz de garantir um nível de acurácia alto para a integração
    numérica automaticamente \citep{Uieda2016}.
  \item Criação e disponibilização da ferramenta de software livre Tesseroids
    para realizar os cálculos de maneria eficiente \citep{Uieda2016}.
  \item Desenvolvimento de um método para permitir que a densidade dos
    tesseroides variasse radialmente de maneira genérica \citep{Soler2019}
    para uso na determinação do embasamento de grande bacias sedimentares
    (parte da tese de doutorado do aluno \SantiagoLink{}).
  \item Desenvolvimento de um método para acelerar o cálculo
    em até aproximadamente 50x em determinados casos. Este método poderia então
    ser utilizado na inversão 3D para determinar a distribuição de densidade em
    subsuperfície \citep[][em colaboração com pesquisadores da Central
    South University, China, e GFZ Potsdam, Alemanha]{Zhao2019} .
\end{fancyenum}

\begin{fancyenum}{\faRocket}{Impacto da pesquisa}
  \item O software Tesseroids é amplamente utilizado para processamento de
    dados de gravidade em escala global, o que é evidenciado pelo alto número
    de citações que recebe\footnote{Segundo
    \url{https://badge.dimensions.ai/details/id/pub.1064143907}}.
  \item O Tesseroids foi utilizado para gerar malhas regulares dos gradientes da
    gravidade gerados a partir dos dados do satélite GOCE \citep{Bouman2016}.
  \item Os avanços feitos no método de modelagem foram fundamentais para a
    criação de métodos de inversão \citep{Uieda2017,Zhao2019} e métodos para
    calcular campos magnéticos de tesseroides \citep{Baykiev2016}.
\end{fancyenum}



\section{Inversão 3D em métodos potenciais}
\label{sec_planting}

\begin{summarybox}[frametitle=\faInfoCircle{}\quad Resumo da linha de pesquisa]
  \begin{fa-ul}
    \faFilePdf & 4 artigos publicados \\
    \faFile & 7 trabalhos completos em anais de eventos \\
    \faComment & 10 apresentações de trabalho \\
    \faUserGraduate & Alunos envolvidos: Gabriel Aparecido das Chagas Silva (IC) \\
    \faGlobeAmericas & País dos colaboradores: Brasil, China, E.U.A.
  \end{fa-ul}
\end{summarybox}
\begin{subsummarybox}[frametitle=\faFilePdf{}\quad Artigos publicados]
  \begin{paperlist}
    2019 &
      \Guangdong, \Bo, \Me, \JLiu, \MKaban, \LChen, \RGuo.
      Efficient 3D large-scale forward-modeling and inversion of gravitational fields in
      spherical coordinates with application to lunar mascons.
      \emph{Journal of Geophysical Research: Solid Earth}.
      \DOI{10.1029/2019jb017691}.
      \Preprint{10.31223/osf.io/dzf9j}.
      \Data{10.6084/m9.figshare.7300523}.
      \\
    2016 &
      \Dio, \Me, \Val.
      How two gravity-gradient inversion methods can be used to reveal different
      geologic features of ore deposit - A case study from the Quadrilátero
      Ferrífero (Brazil),
      \emph{Journal of Applied Geophysics},
      \DOI{10.1016/j.jappgeo.2016.04.011}.
      \\
    2014 &
      \Dio, \Me, \Val.
      Imaging iron ore from the Quadrilátero Ferrífero (Brazil) using geophysical
      inversion and drill hole data,
      \emph{Ore Geology Reviews},
      \DOI{10.1016/j.oregeorev.2014.02.011}.
      \\
    2012 &
      \Me, \Val.
      Robust 3D gravity gradient inversion by planting anomalous densities,
      \emph{Geophysics},
      \DOI{10.1190/geo2011-0388.1}.
      \GitHub{pinga-lab/paper-planting-densities}.
      \Data{10.6084/m9.figshare.91574}.
  \end{paperlist}
\end{subsummarybox}
\begin{subsummarybox}[frametitle=\faFile{}\quad Trabalhos completos em anais de eventos]
  \begin{paperlist}
    2012 &
      \Me, \Val.
      Use of the ``shape-of-anomaly'' data misfit in 3D inversion by planting
      anomalous densities,
      \emph{SEG Technical Program Expanded Abstracts},
      \DOI{10.1190/segam2012-0383.1}.
      \GitHub{leouieda/seg2012}.
      \\
    ~ &
      \Dio, \Me, \YLi, \Val, \BragaVale, \Angeli, \Peres.
      Iron ore interpretation using gravity-gradient inversions in the Carajás, Brazil.
      \emph{SEG Technical Program Expanded Abstracts},
      \DOI{10.1190/segam2012-0525.1}.
      \\
    2011 &
      \Me, \Val.
      Robust 3D gravity gradient inversion by planting anomalous densities,
      \emph{SEG Technical Program Expanded Abstracts},
      \DOI{10.1190/1.3628201}.
      \GitHub{leouieda/seg2011}
      \\
    ~ &
      \Me, \Val.
      3D gravity inversion by planting anomalous densities.
      \emph{12th International Congress of the Brazilian Geophysical Society},
      \DOI{10.1190/sbgf2011-179}.
      \GitHub{leouieda/sbgf2011}
      \\
    ~ &
      \Me, \Val.
      3D gravity gradient inversion by planting density anomalies.
      \emph{73th EAGE Conference and Exhibition incorporating SPE EUROPEC},
      \DOI{10.3997/2214-4609.20149567}.
      \GitHub{leouieda/eage2011}
      \\
    ~ &
      \Dio, \Me, \Val, \BragaVale, \Gomes.
      In-depth imaging of an iron orebody from Quadrilatero Ferrifero using 3D
      gravity gradient inversion,
      \emph{SEG Technical Program Expanded Abstracts},
      \DOI{10.1190/1.3628219}.
      \\
    ~ &
      \Dio, \Val, \Me, \BragaVale.
      Inversão de Dados de Aerogradiometria Gravimétrica 3D-FTG Aplicada a
      Exploração Mineral na Região do Quadrilátero Ferrífero,
      \emph{12th International Congress of the Brazilian Geophysical Society},
      \DOI{10.1190/sbgf2011-243}.
  \end{paperlist}
\end{subsummarybox}
\begin{subsummarybox}[frametitle=\faComment{}\quad Outras apresentações]
  \begin{paperlist}
  2014 &
  \Me, \Val.
  Gravity inversion in spherical coordinates using tesseroids,
  \emph{EGU General Assembly}.
  \DOI{10.6084/m9.figshare.1155457}
  \\
  2013 &
  \Me, \Val.
  3D magnetic inversion by planting anomalous densities,
  \emph{AGU Meeting of the Americas},
  Cancun, Mexico.
  \DOI{10.6084/m9.figshare.703651}
  \\
  2012 &
  \Me, \Val.
  Rapid 3D inversion of gravity and gravity gradient data to test geologic
  hypotheses,
  \emph{International Symposium on Gravity, Geoid and Height Systems},
  Venice, Italy.
  \DOI{10.6084/m9.figshare.156859}
  \end{paperlist}
\end{subsummarybox}

Minha pesquisa nessa linha começou com meu mestrado no \ON{}
(seção~\ref{sec_on}).
O salto de 2D para 3D nos métodos de inversão que discretizam subsuperfície em
elementos geométricos, como prismas retangulares retos, causou um aumento
drástico nos recursos computacionais necessários para executá-los.
Meu foco inicial nessa linha de pesquisa foi tornar esse tipo de inversão
viável sem o uso de supercomputadores, podendo assim lidar com o aumento no
volume de dados causado pelo desenvolvimento da gradiometria gravimétrica.
Em meu primeiro artigo publicado \citep{Uieda2012}, descrevemos o
\textit{método de plantação} que era capaz de inverter dezenas de milhares de
observações com modelos da ordem de um milhão de prismas utilizando
computadores convencionais em uma fração do tempo que métodos semelhantes.
O método também proporciona ao intérprete a possibilidade de criar um esqueleto
do alvo, utilizando seu conhecimento geológico, com a inversão por plantação
preenchendo a forma do corpo em torno do esqueleto.

Minha segunda contribuição metodológica para a área foi com o trabalho de
\citet{Zhao2019}.
Inicialmente, atuei como revisor de uma versão anterior desse trabalho que não
chegou a ser publicada.
Como eu costumo assinar minhas revisões, fui convidado pelo aluno de doutorado
Guangdong Zhao da Central South University, China, para participar da autoria
do trabalho e ajudá-los a resolver algumas questões necessárias para submissão
do artigo.
Nesse trabalho, desenvolvemos um método de inversão linear de dados de
gravidade em uma aproximação esférica.
O método é eficaz pois se aproveita da simetria entre o modelo e a malha regular
de dados para reduzir o custo computacional em algumas ordens de grandeza.

Em 2024, o aluno \GabrielLink{} se juntou ao grupo em um projeto de iniciação
científica com bolsa PIBIC para trabalhar neste tema. Seu projeto, intitulado
``Refinamento de malha na inversão de dados gravimétricos pelo algoritmo de
plantação'' busca atualizar a implementação computacional do método de
\citet{Uieda2012} e utilizar técnicas de ``esqueletonização'' provenientes do
processamento de imagens para implementar uma discretização adaptativa como
parte do método de inversão.

\begin{fancyenum}{\faBullseye}{Objetivos}
   \item Construir modelos compactos capazes de recuperar a forma 3D de corpos
     geológicos a partir de dados geofísicos.
   \item Superar as limitações computacionais da inversão 3D em métodos
     potenciais devido ao elevado número de dados e elementos no modelo.
   \item Providenciar maior nível de controle ao intérprete para incorporar
     conhecimentos sobre a geologia do alvo.
\end{fancyenum}
\begin{fancyenum}{\faLightbulb}{Principais contribuições}
  \item Desenvolvimento de um método computacionalmente eficiente a inversão
    de dados de gradiente da gravidade capaz de suportar o dobro do número
    de dados que métodos semelhantes \citep{Uieda2012,Carlos2016}.
  \item Criação de um novo mecanismo que permite ao intérprete inserir
    informação geológica nos modelos geofísicos através das \textit{sementes}
    utilizadas em \citet{Uieda2012}.
  \item Desenvolvimento de um método eficiente para inversão 3D em uma
    aproximação esférica \citep{Zhao2019} (colaboração com pesquisadores da
    Central South University, China, e GFZ Potsdam, Alemanha).
\end{fancyenum}
\begin{fancyenum}{\faRocket}{Impacto da pesquisa}
  \item Possibilitou a comparação dos resultados com outro método de inversão
    já bem estabelecido, avançando a interpretação de formações contento
    minério de ferro no Quadrilátero Ferrífero \citep{Carlos2014,Carlos2016}.
  \item Um dos primeiros métodos capazes de inverter todas as componentes do
    tensor de gradientes da gravidade.
\end{fancyenum}


\section{Determinação da espessura crustal através de distúrbios da gravidade}
\label{sec_moho}

\begin{summarybox}[frametitle=\faInfoCircle{}\quad Resumo da linha de pesquisa]
  \begin{fa-ul}
    \faFilePdf & 1 artigo publicado \\
    \faUserGraduate & Alunos envolvidos: Aidan Hernaman (MSc) \\
    \faComment & 1 apresentação de trabalho
  \end{fa-ul}
\end{summarybox}
\begin{subsummarybox}[frametitle=\faFilePdf{}\quad Artigos publicados]
  \begin{paperlist}
    2017 &
      \Me, \Val.
      Fast non-linear gravity inversion in spherical coordinates with application
      to the South American Moho,
      \emph{Geophysical Journal International},
      \DOI{10.1093/gji/ggw390}.
      \Preprint{10.31223/osf.io/9ba4m}.
      \GitHub{pinga-lab/paper-moho-inversion-tesseroids}.
      \Data{10.6084/m9.figshare.3987267}.
  \end{paperlist}
\end{subsummarybox}
\begin{subsummarybox}[frametitle=\faInfoCircle{}\quad Apresentações]
  \begin{paperlist}
    2017 &
      \Me.
      Inverting gravity to map the Moho: A new method and the open source
      software that made it possible,
      \emph{Department of Geology and Geophysics, \UHM},
      Honolulu, USA.
      \DOI{10.6084/m9.figshare.4779766}
  \end{paperlist}
\end{subsummarybox}

Esta linha de pesquisa busca estimar a profundidade da
\href{https://en.wikipedia.org/wiki/Mohorovi%C4%8Di%C4%87_discontinuity}{Descontinuidade de Mohorovičić}
(Moho) através da inversão de distúrbios da gravidade.
Se considerarmos que a contribuição da variação de densidade na
crosta e no manto são negligenciáveis ou foram removidas, o que observamos no
distúrbio da gravidade corrigido da topografia é puramente o efeito da variação
na profundidade da Moho quando comparada com um nível de referência.
O método mais utilizado para essa estimava é o de \citet{Oldenburg1974}, que
utiliza a transformada rápida de Fourier (FFT) para estimar o relevo de uma
interface em uma aproximação plana da Terra.
Esse método foi utilizado em \citet{vanderMeijde2013} para estimar a espessura
da crosta na América do Sul.
Métodos que são adequados pra uma aproximação esférica possuem suas limitações.
\citet{Wieczorek1998}, que é uma generalização de \citet{Oldenburg1974} para
harmônicos esféricos, é mais adequado para aplicações globais.
\citet{Reguzzoni2013} é mais flexível mas requer recursos computacionais
elevados.

Para a última etapa do meu doutorado (seção~\ref{sec_doutorado}), me baseei no
trabalho de \citet{Silva2014} para desenvolver uma inversão não-linear para
a determinação da profundidade da Moho em uma aproximação esférica.
Meu método aprimora a generalização do método de \citet{Bott1960} feita por
\citet{Silva2014} introduzindo regularização de suavidade.
Para operar em uma aproximação esférica, a modelagem direta é feita utilizando
tesseroides (seção~\ref{sec_modelagemdireta}).
Estimativas da profundidade da Moho provenientes de dados sismológicos são
utilizados na inversão para determinar o contraste de densidade na Moho e o
nível de referência.

Esse trabalho se beneficiou muito da infraestrutura computacional que
desenvolvi ao longo da minha pós-graduação.
Os softwares Fatiando a Terra (seção~\ref{sec_fatiando}) e Tesseroids
(seção~\ref{sec_tesseroids}) foram fundamentais para a criação do método e
o impacto que esse trabalho teve (ver ``impacto da pesquisa'' abaixo).
Foi com esse trabalho que realmente tive evidência de que minha dedicação à
ciência aberta e ao uso de software livre na pesquisa
(seção~\ref{sec_software}) não seriam prejudiciais a minha produtividade
científica.

\begin{fancyenum}{\faBullseye}{Objetivos}
  \item Estimar a profundidade da Moho através da inversão de distúrbios da
    gravidade em uma aproximação esférica da Terra.
  \item Superar o custo computacional alto do procedimento de inversão
    não-linear.
  \item Incorporar estimativas sismológicas da profundidade da Moho na inversão.
\end{fancyenum}
\begin{fancyenum}{\faLightbulb}{Principais contribuições}
  \item Desenvolvi um fluxo de processamento baseado em softwares livres que
    está sendo utilizada pela comunidade científica.
  \item Criei um método eficiente de inversão não-linear em uma aproximação
    esférica capaz de incorporar estimativas sismológicas na solução.
  \item Produzi a primeira estimativa de alta resolução da profundidade da Moho
    para a América do Sul baseada em dados de gravidade em aproximação
    esférica.
\end{fancyenum}
\begin{fancyenum}{\faRocket}{Impacto da pesquisa}
  \item O método e o código associado ao trabalho \citet{Uieda2017}
    possibilitaram sua aplicação direta a outras regiões \citep[e.g.,][entre
    outros]{Chisenga2019, Sobh2020, KemgangGhomsi2021}.
  \item O código aberto foi utilizado como base para o método de
    \citet{Haas2020}, possivelmente acelerando seu desenvolvimento.
  \item O trabalho \citet{Uieda2017} possui um número de citações considerado
    alto para a área\footnote{Segundo
    \url{https://badge.dimensions.ai/details/id/pub.1059638400}}.
\end{fancyenum}


\section{Camada equivalente para processamento de dados gravimétricos e magnetométricos}
\label{sec_eql}

\begin{summarybox}[frametitle=\faInfoCircle{}\quad Resumo da linha de pesquisa]
  \begin{fa-ul}
    \faFilePdf & 2 artigos publicados \\
    \faComment & 1 apresentação de trabalho \\
    \faUserGraduate & Alunos envolvidos: Santiago Soler (PhD), India Uppal
    (PhD), Arthur Siqueira Macêdo (MSc), Eros Kerouak Cordeiro Pereira (MSc),
    Hamed Al-Salehi (BSc), Daniel Gilbert (BSc), Ellen Fernandes Marcos (BSc)
    \\
    \faGlobeAmericas & País dos colaboradores: Brasil, Argentina, Reino Unido, Canadá
  \end{fa-ul}
\end{summarybox}
\begin{subsummarybox}[frametitle=\faFilePdf{}\quad Artigos publicados]
  \begin{paperlist}
    2021 &
      \Santiago, \Me.
      Gradient-boosted equivalent sources.
      \emph{Geophysical Journal International}.
      \DOI{10.1093/gji/ggab297}.
      \GitHub{compgeolab/eql-gradient-boosted}.
      \Preprint{10.31223/X58G7C}.
      \Data{10.6084/m9.figshare.13604360}.
      \\
    2013 &
      \Bi, \Val, \Me.
      Polynomial equivalent layer,
      \emph{Geophysics},
      \DOI{10.1190/geo2012-0196.1}.
  \end{paperlist}
\end{subsummarybox}
\begin{subsummarybox}[frametitle=\faInfoCircle{}\quad Apresentações]
  \begin{paperlist}
    2020 &
      \Me, \Santiago.
      Evaluating the accuracy of equivalent-source predictions using
      cross-validation,
      \emph{EGU General Assembly}.
      \DOI{10.5194/egusphere-egu2020-15729}.
      \Data{10.6084/m9.figshare.12245372}
  \end{paperlist}
\end{subsummarybox}

A técnica da camada equivalente é baseada no fato de que o campo gravitacional
ou magnético de uma fonte 3D pode ser reproduzido em qualquer ponto fora da
fonte pelo campo de uma camada contínua 2D de propriedades físicas (densidade
ou magnetização) \citet{Dampney1969}.
Uma vez determinada essa distribuição 2D de propriedades físicas podemos
calcular o campo da fonte 3D real em qualquer ponto do espaço (e.g., em uma
malha regular ou a uma altitude maior).
Dessa forma, a camada equivalente se aproveita da ambiguidade inerente de
problemas inversos em métodos potenciais para processar e interpolar dados
irregularmente distribuídos.
Na prática, a camada contínua é discretizada em fontes pontuais chamadas de
fontes equivalentes.

Minha primeira contribuição nessa linha de pesquisa surgiu durante uma conversa
com meu amigo \VanderleiLink{} em um bar de São Cristóvão, Rio de Janeiro.
Na época, estávamos fazendo nosso doutorado no \ON{}
(seção~\ref{sec_on}) e dividíamos um apartamento.
Essas conversas eram frequentes e creio serem a fonte de meu entendimento de
diversos conceitos de métodos potenciais e problemas inversos em geofísica.
Durante essa conversa em particular, tentávamos contornar o maior empecilho
para a aplicação da camada equivalente: seu elevado custo computacional.
Nossa solução foi dividir a camada de fontes equivalentes em blocos e
representar a distribuição de propriedade física dentro de cada bloco por um
polinômio 2D de grau baixo.
Dessa forma, nosso problema inverso seria parametrizado pelos coeficientes dos
polinômios e não pelas propriedades físicas, resultando em uma redução no
número de parâmetros a serem estimados em algumas ordens de grandeza.
Chamamos nossa técnica de camada equivalente polinomial (PEL em inglês) que foi
publicada em \citet{OliveiraJr2013} e se tornou parte da tese de doutorado do
Vanderlei.

Os avanços obtidos com a PEL foram grandes mas ainda insuficientes quando o
número de dados observados ultrapassa centenas de milhares.
Buscamos superar essa limitação com o segundo trabalho da tese de doutorado
do \SantiagoLink{} (seção~\ref{sec_orientacao}), onde dividimos os dados e
fontes equivalentes em janelas sobrepostas e resolvemos o problema inverso de
maneira iterativa utilizando a técnica de \textit{gradient boosting}
\citep{Friedman2001}.
A técnica, batizada de \textit{gradient-boosted equivalent sources}, foi
publicada em \citet{Soler2021} onde demonstramos seu uso para interpolar
mais de 1,7 milhões de dados terrestres de distúrbio da gravidade utilizando
menos de 16 Gigabytes de memória RAM em aproximadamente uma hora de computação.

Atualmente, a aluna de doutorado \IndiaLink{} (seção~\ref{sec_orientacao})
está adaptando a gradient-boosted equivalent sources para dados magnéticos.
Seu objetivo é ajustar todos os dados aeromagnéticos da Antártica em conjunto
com dados de satélite para produzir uma única malha regular da amplitude do
campo magnético anômalo (ao invés da anomalia magnética de campo total) para
todo o continente.
O aluno de mestrado \ArthurLink{} está adaptando o método desenvolvido pela
India para coordenadas esféricas.
A aluna de trabalho de graduação \EllenLink{} desenvolveu um estudo
sobre a eficácia da validação cruzada aplicada ao método das fontes
equivalentes.
Por fim, o aluno \ErosLink{} está utilizando as fontes equivalentes para
integrar dados gravimétricos terrestres e de satélites para o sul e sudeste
brasileiro, além de desenvolvendo métodos para determinação do melhor
espaçamento de malhas regulares.

\begin{fancyenum}{\faBullseye}{Objetivos}
  \item Reduzir o custo computacional da aplicação da camada equivalente.
  \item Adaptar técnicas de aprendizagem de máquina para complementar nosso
    uso da camada equivalente, como a validação cruzada e \textit{gradient
    boosting}.
  \item Utilizar a camada equivalente para integrar dados de diferentes
    aquisições em diferentes escalas (terrestres, aéreos e de satélite),
    produzindo uma única malha uniforme em escala continental.
\end{fancyenum}
\begin{fancyenum}{\faLightbulb}{Principais contribuições}
  \item Criação de dois métodos computacionalmente eficientes para solução do
    problema inverso da camada equivalente \citep{OliveiraJr2013,Soler2021}.
  \item Disponibilização da gradient-boosted equivalent sources de
    \citet{Soler2021} no software
    \href{https://www.fatiando.org/harmonica/}{Harmonica} (implementada em
    grande parte pelo Santiago).
  \item Geração de uma malha regular de distúrbios da gravidade para todo o
    continente australiano a uma altitude uniforme \citep{Soler2021}.
\end{fancyenum}
\begin{fancyenum}{\faRocket}{Impacto da pesquisa}
  \item Os avanços obtidos em \citet{Soler2021} viabilizaram a aplicação das
      fontes equivalentes a conjuntos de milhões de dados. Isso possibilitou os
      projetos de outros membros do grupo e a linha de pesquisa na integração
      de dados aeromagnéticos da Antártica (seção~\ref{sec_antartica}).
\end{fancyenum}


\section{Deconvolução de Euler}
\label{sec_euler}

\begin{summarybox}[frametitle=\faInfoCircle{}\quad Resumo da linha de pesquisa]
  \begin{fa-ul}
    \faFilePdf & 3 artigos publicados \\
    \faFile & 1 trabalho completo em anais de eventos \\
    \faUserGraduate & Alunos envolvidos: India Uppal (PhD), Gelson Ferreira de Souza Junior (PhD), Lottie Cooper (BSc) \\
    \faGlobeAmericas & País dos colaboradores: Brasil, Reino Unido
  \end{fa-ul}
\end{summarybox}
\begin{subsummarybox}[frametitle=\faFilePdf{}\quad Artigos publicados]
  \begin{paperlist}
    2025 &
      \Me, \Gelson, \India, \Bi.
      Euler inversion: Locating sources of potential-field data through
      inversion of Euler's homogeneity equation,
      \emph{Geophysical Journal International},
      \DOI{10.1093/gji/ggaf114}.
      \GitHub{compgeolab/euler-inversion}.
      \\
    2014 &
      \Me, \Bi, \Val.
      Geophysical tutorial: Euler deconvolution of potential-field data,
      \emph{The Leading Edge},
      \DOI{10.1190/tle33040448.1}.
      \GitHub{pinga-lab/paper-tle-euler-tutorial}.
      \\
    2013 &
      \Figura, \Val, \Me, \Bi, \JB.
      Estimating the nature and the horizontal and vertical positions of 3D
      magnetic sources using Euler deconvolution,
      \emph{Geophysics},
      \DOI{10.1190/geo2012-0515.1}.
  \end{paperlist}
\end{subsummarybox}
\begin{subsummarybox}[frametitle=\faFile{}\quad Trabalhos completos em anais de eventos]
  \begin{paperlist}
    2014  &
      \Figura, \Val, \Me, \Bi, \JB.
      A Single Euler Solution Per Anomaly,
      \emph{76th EAGE Conference and Exhibition 2014},
      \DOI{10.3997/2214-4609.20140891}.
  \end{paperlist}
\end{subsummarybox}

A deconvolução de Euler foi introduzida por \citet{Thompson1982} e
\citet{Reid1990} como uma técnica rápida para a estimativa da profundidade das
fontes em métodos potenciais.
Sua simplicidade e rapidez fez com que fosse adotada amplamente. Essa
popularidade também resultou em abusos da técnica, com interpretações errôneas
de seus resultados e aplicações indevidas encontradas na literatura
\citep{Reid2014}.
Estava com isso em mente quando fui convidado pelo
\href{https://github.com/kwinkunks}{Matt Hall}, fundador do
\SwungLink{} (seção~\ref{sec_swung}), para escrever um tutorial de acesso
aberto na revista \href{https://library.seg.org/loi/leedff}{The Leading Edge}.
O Matt estava organizando uma série de tutoriais para explicar conceitos de
geofísica aplicada utilizando ferramentas de software livre.
Os tutoriais seriam de acesso aberto e também estariam disponíveis na
\href{https://wiki.seg.org/wiki/Main_Page}{SEG Wiki}\footnote{Por exemplo, o
tutorial \citet{Uieda2014} está disponível em
\url{https://wiki.seg.org/wiki/Euler_deconvolution_of_potential_field_data}}.
Decidi escrever sobre a deconvolução de Euler utilizando a implementação
disponível no \FatiandoLink{} para auxiliar na divulgação de boas práticas
ao utilizar a técnica.
Além disso, tive a oportunidade de participar do trabalho de mestrado do aluno
\href{https://www.pinga-lab.org/people/melo.html}{Felipe Ferreira Melo}.
Nosso objetivo era utilizar o método de \citet{Barbosa1999} para obter uma
única solução para a profundidade e para o índice estrutural por fonte das
anomalias.
O método que desenvolvemos foi publicado em \citet{Melo2013}.

Durante as aulas do Professor \SpirosLink{} na York University
(seção~\ref{sec_york}), aprendi sobre modelos matemáticos implícitos que são
utilizados na geodésia.
Esses modelos são utilizados quando não é possível separar os dados dos
parâmetros de um problema inverso.
Anos depois, nas aulas de métodos potenciais da minha orientadora
Professora \ValeriaLink{} no \ON{}, conectei a teoria que havia aprendido na
York com a deconvolução de Euler.
Isso me levou a desenvolver a ideia de utilizar a formulação matemática da
geodésia para inverter a equação da homogeneidade de Euler.
Esse projeto, embora promissor, foi pausado por conta de outros compromissos
que tive na época, principalmente meu novo cargo de Professor Assistente na
UERJ (seção~\ref{sec_uerj}).
Em 2024, retornei a esse trabalho junto ao \GelsonLink{}, \IndiaLink{},
e \VanderleiLink{}.
O método que desenvolvemos, chamado \textit{inversão de Euler}
\citep{Uieda2025}, se mostrou superior à deconvolução de Euler em quase todos
os aspectos: é mais robusto a ruídos e a fontes interferentes, é capaz de
estimar o índice estrutural das fontes de maneira robusta, além de ser tão
computacionalmente eficiente quanto o método original.

Através do projeto do aluno de doutorado \GelsonLink{}
(seção~\ref{sec_orientacao}), investigamos o uso da deconvolução de Euler para
a interpretação de dados de microscopia magnética (seção~\ref{sec_micromag}).
Na microscopia, buscamos estimar os momentos de dipolo de pequenos minrais com
estruturas de monodomínio ou pseudo-monodomínio.
Em ambos os casos, os sinais produzidos pelas fontes é aproximadamente dipolar
na escala que enxergam os microscópios atuais.
Como existe ambiguidade no problema inverso de estimar a amplitude do momento
e a profundidade do grão, necessitamos de informações \textit{a priori} sobre
a localização dos dipolos.
Em \citet{SouzaJunior2024} e \citet{SouzaJunior2025} utilizamos a deconvolução
de Euler como informação \textit{a priori} para a inversão do momento de dipolo
com grande sucesso.

\begin{fancyenum}{\faBullseye}{Objetivos}
  \item Popularizar o entendimento da técnica e as boas práticas em sua
    aplicação e na interpretação de seus resultados.
  \item Buscar técnicas novas para a aplicação da deconvolução de Euler,
    incluindo como diminuir o número de soluções falsas.
\end{fancyenum}
\begin{fancyenum}{\faLightbulb}{Principais contribuições}
  \item Desenvolvimento de um método para estimar o índice estrutural das
    fontes e obter uma única solução por fonte \citep{Melo2013}.
  \item Publicação de um tutorial de acesso livre explicando os conceitos do
    método e como interpretar seus resultados \citep{Uieda2014}.
\end{fancyenum}
\begin{fancyenum}{\faRocket}{Impacto da pesquisa}
  \item Os avanços obtidos na deconvolução de Euler foram utilizados
    como informação a priori em \citet{OliveiraJr2015} e na linha de pesquisa
    em microscopia magnética (seção~\ref{sec_micromag}) nos trabalhos de
    \citet{SouzaJunior2024} e \citet{SouzaJunior2025}.
\end{fancyenum}



\section{Interpolação de dados geofísicos}

\begin{summarybox}[frametitle=\faInfoCircle{}\quad Resumo da linha de pesquisa]
  \begin{fa-ul}
    \faFilePdf & 1 artigo publicado \\
    \faComment & 2 apresentações de trabalho \\
    \faUserGraduate & Alunos envolvidos: Majed Abura (BSc), Ali Alhazmi (BSc), Sarah Askevold (BSc) \\
    \faGlobeAmericas & País dos colaboradores: E.U.A.
  \end{fa-ul}
\end{summarybox}
\begin{subsummarybox}[frametitle=\faFilePdf{}\quad Artigos publicados]
  \begin{paperlist}
    2018 &
      \Me.
      Verde: Processing and gridding spatial data using Green's functions.
      \emph{Journal of Open Source Software}.
      \DOI{10.21105/joss.00957}.
      \GitHub{fatiando/verde}.
  \end{paperlist}
\end{subsummarybox}
\begin{subsummarybox}[frametitle=\faInfoCircle{}\quad Apresentações]
  \begin{paperlist}
    2018 &
      \Me, \Eric, \Paul, \David.
      Coupled Interpolation of Three-component GPS Velocities,
      \emph{AGU Fall Meeting}.
      \DOI{10.6084/m9.figshare.7440683}
      \\
    ~ &
      \Me, \David, \Paul.
      Joint Interpolation of 3-component GPS Velocities Constrained by
      Elasticity,
      \emph{AOGS $15^{th}$ Annual Meeting}.
      \DOI{10.6084/m9.figshare.6387467}
  \end{paperlist}
\end{subsummarybox}

Meu interesse nessa linha de pesquisa surgiu durante meu trabalho com o
\GMTLink{} (GMT; seção~\ref{sec_hawaii}).
O \PaulLink{}, principal desenvolvedor do GMT, e o Professor \SandwellLink{}
me convidaram para tentar resolver um problema que encontravam no código de
interpolação do GMT.
Os módulos
\href{https://docs.generic-mapping-tools.org/latest/greenspline.html}{\texttt{greenspline}}
e
\href{https://docs.generic-mapping-tools.org/latest/supplements/geodesy/gpsgridder.html}{\texttt{gpsgridder}}
do GMT utilizam métodos de interpolação baseados na solução de equações de
elasticidade com fontes pontuais (funções de Green).
Para realizar a interpolação, os programas realizam um ajuste de mínimos
quadrados linear aos dados para encontrar um fator de escala para cada fonte
pontual.
Em seguida, o valor predito pelas fontes pontuais é calculado em uma malha
regular.
Esse procedimento é o mesmo que utilizamos para a camada equivalente
(seção~\ref{sec_eql}), logo eu já tinha experiência prévia com o método.
Porém, o software não estava produzindo os resultados corretos quando os
dados eram ponderados pela sua incerteza.
A habilidade de ponderar os dados é a grande vantagem desses métodos
sobre outros que são mais eficientes, como o de \citet{Smith1990}
(implementado no módulo
\href{https://docs.generic-mapping-tools.org/latest/surface.html}{\texttt{surface}}
do GMT).

Para investigar o problema, resolvi implementar em Python os métodos de
\citet{Sandwell1987} e \citet{Sandwell2016} para conduzir experimentos
numéricos.
Esses métodos e outras funções que implementei durante essa investigação
formaram parte da biblioteca \href{https://www.fatiando.org/verde}{Verde}
que estava desenvolvendo para o \FatiandoLink{} (seção~\ref{sec_fatiando}).
Descobri que a fonte do erro estava no uso da solução exata de mínimos
quadrados no GMT, que matematicamente ignora a ponderação.
Por fim, contornamos esse efeito utilizando uma solução regularizada quando
a ponderação dos dados é necessária.
O resultado dessa investigação levou à criação de um software novo (Verde) que
possibilita maior flexibilidade para o processamento e interpolação de dados
geocientíficos do que o GMT.
Essa flexibilidade é necessária para reaproveitarmos técnicas de aprendizagem
de máquinas para a interpolação, que nada mais é que um problema de regressão
linear para previsão de dados.
Por exemplo, o Verde inclui métodos de validação cruzada específicos para dados
espacialmente auto-correlacionados \citep{Roberts2017} e métodos compatíveis
com a biblioteca \href{https://scikit-learn.org/stable/}{scikit-learn}
\citep{Pedregosa2011}.

Iniciei também uma investigação para adaptar o método de \citet{Sandwell2016}
para a interpolação simultânea das três componentes de velocidades de GPS.
O vínculo entre cada componente viria da teoria de deformação elástica, assim
como o caso de duas componentes de \citet{Sandwell2016}.
Esse métodos poderia então ser utilizado para interpolar de maneira conjunta
os dados de GPS (com 3 componentes mas pouca cobertura espacial) e de InSAR
(com 1 componente mas alta cobertura espacial).
Realizei duas apresentações em congressos sobre esse trabalho mas não fui capaz
de obter resultados satisfatórios.
Minha ida para Liverpool (seção~\ref{sec_liverpool}) interrompeu essa linha mas
espero continuá-la no futuro.

\begin{fancyenum}{\faBullseye}{Objetivos}
  \item Criar ferramentas de software livre capazes de interpolar dados
    geocientíficos de maneira eficiente.
  \item Desenvolver métodos para contornar o alto custo computacional da
    interpolação.
  \item Utilizar técnicas de aprendizagem de máquina para aprimorar os
    resultados da interpolação.
  \item Investigar a possibilidade da interpolação conjunta de dados de GPS e
    InSAR.
\end{fancyenum}
\begin{fancyenum}{\faLightbulb}{Principais contribuições}
  \item Criação do software Verde para interpolação em Python capaz de
    interagir com a biblioteca scikit-learn de aprendizagem de máquinas.
  \item Solução de um erro em dois módulos do GMT.
  \item Investigação inicial do uso da teoria da elasticidade para a
    interpolação conjunta de GPS e InSAR.
\end{fancyenum}
\begin{fancyenum}{\faRocket}{Impacto da pesquisa}
  \item O software Verde é utilizado regularmente pela comunidade científica
    e recebeu 49 citações ao trabalho \citet{Uieda2018}\footnote{Segundo
    a plataforma Google Scholar \url{https://scholar.google.com/citations?view_op=view_citation&hl=en&user=qfmPrUEAAAAJ&citation_for_view=qfmPrUEAAAAJ:ye4kPcJQO24C} (acessada em 13/04/2025)}
    e mais de 4000 downloads mensais\footnote{Segundo a página
    \url{https://pepy.tech/project/Verde} (acessada em 13/04/2025)}.
\end{fancyenum}


\section{Modelagem de dados de microscopia magnética}
\label{sec_micromag}

\begin{summarybox}[frametitle=\faInfoCircle{}\quad Resumo da linha de pesquisa]
  \begin{fa-ul}
    \faFilePdf & 2 artigo publicado, 1 artigo aceito para publicação, 1 artigo
    em revisão na forma de preprint\\
    \faUserGraduate & Alunos envolvidos: Gelson Ferreira de Souza Junior (PhD),
    Yago Moreira Castro (MSc)\\
    \faGlobeAmericas & País dos colaboradores: Brasil, E.U.A., Reino Unido \\
    \faSearchDollar & Financiamento: \href{https://royalsociety.org/}{Royal Society} (IES\textbackslash{}R3\textbackslash{}213141)
  \end{fa-ul}
\end{summarybox}
\begin{subsummarybox}[frametitle=\faFilePdf{}\quad Artigos publicados]
  \begin{paperlist}
    2025 &
      \Gelson, \Me, \Ricardo, \Roger, \Ualisson, \Yago.
      Robust directional analysis of magnetic microscopy images using
      non-linear inversion and iterative Euler deconvolution,
      \emph{EarthArXiv} (preprint em revisão na \textit{JGR: Solid Earth}),
      \DOI{10.31223/X5N42F}.
      \GitHub{compgeolab/micromag-interfering-sources}.
      \\
    2025 &
      \Ualisson, \Wyn, \Muxworthy, \Gelson, \Les, \Me, \Ricardo.
      Efficiency of thermoremanent magnetization acquisition in vortex-state
      particle assemblies,
      \emph{Geophysical Research Letters} (aceito para publicação),
      \DOI{10.22541/essoar.173870862.24424739/v1}.
      \Data{10.5281/zenodo.14051069}.
      \\
    2024 &
      \Gelson, \Me, \Ricardo, \Janine, \Roger.
      Full vector inversion of magnetic microscopy images using Euler deconvolution as prior information,
      \emph{Geochemistry, Geophysics, Geosystems},
      \DOI{10.1029/2023GC011082}.
      \GitHub{compgeolab/micromag-euler-dipole}.
      \\
    2015 &
      \Bi, \Dai, \Val, \Me.
      Estimation of the total magnetization direction of approximately spherical
      bodies,
      \emph{Nonlinear Processes in Geophysics},
      \DOI{10.5194/npg-22-215-2015}.
      \GitHub{pinga-lab/Total-magnetization-of-spherical-bodies}.
  \end{paperlist}
\end{subsummarybox}

Novos avanços em microscopia magnética, onde o campo magnético vertical de uma
lâmina pode ser medido em escala micrométrica, tem o potencial de permitir a
investigação de grãos individuais de minerais magnéticos.
Técnicas de paleomagnetismo tradicionais utilizam a medição da magnetização
total de uma amostra que contém minerais magnetizados em diferentes direções.
A separação dessas direções é um dos principais papéis das longas etapas de
desmagnetização realizadas em estudos paleomagnéticos.
Conseguir estimar a magnetização de minerais individuais através da microscopia
magnética permitirá a melhor separação das magnetizações e poderá levar a
estimativas de paleo-pólos geomagnéticos e de paleointensidades mais
confiáveis.

Meu trabalho nessa linha de pesquisa iniciou com a orientação do aluno
\GelsonLink{} (seção~\ref{sec_orientacao}) em 2021.
Seu projeto visa adaptar técnicas da geofísica aplicada para a microscopia
magnética, como a deconvolução de Euler, o método de \citet{OliveiraJr2015} e o
processamento de dados aeromagnéticos.
Nosso principal objetivo é realizer estudos de paleomagnetismo
e paleointensidade utilizando dados de microscopia magnética, um feito que
ainda não foi alcançado por outros grupos de pesquisa.
O Gelson já foi capaz de utilizar técnicas de processamento de imagens (que
aprendi com a minha disciplina de sensoriamento remoto;
seção~\ref{sec_ensino_grad}) e filtros, como a amplitude do gradiente total,
para isolar os campos magnéticos de minerais individuais nas imagens de
microscopia.
Com os campos separados, é possível determinar a posição 3D do mineral
utilizando uma única janela na deconvolução de Euler.
A posição é então utilizada como informação a priori em uma adaptação do método
de \citet{OliveiraJr2015} para estimar o momento de dipolo do mineral.
Publicamos seus resultados e o primeiro método semi-automático para a inversão
de centenas de fontes magnéticas em \citet{SouzaJunior2024}.
Percebemos em seguida que o método falhava em situações nas quais fontes de
sinal forte estavam sobrepostas a fontes de sinal fraco.
Isso impossibilitava a aplicação do método a rochas com altas concentrações de
minerais magnéticos, como rochas vulcânicas.
Aprimoramos o método introduzindo uma inversão não-linear para refinar
a solução e um processo de remoção do sinal progressivo do sinal das fontes
durante a inversão.
O resultado é um método mais robusto que é capaz de estimar corretamente
a magnetização em altas concentrações de minerais magnéticos.
Esses resultados foram publicados como um preprint \citep{SouzaJunior2025}
e está em revisão na revista \textit{JGR: Solid Earth}.

Em 2022, fui contemplado com um projeto financiado pela
\href{https://royalsociety.org/}{Royal Society} para dar início à essa
colaboração com o Gelson e seu orientador Ricardo I. F. Trindade.
O projeto combina minha experiência em processamento e inversão de dados
aeromagnéticos com o amplo conhecimento do Gelson e do Ricardo em
paleomagnetismo.
Através desse projeto, eu, o Gelson e o Ricardo pudemos realizar viagens entre
Liverpool e São Paulo para trabalharmos nos resultados dos dois artigos citados
acima.
Este projeto me possibilitou retornar ao paleomagnetismo 17 anos após terminar
minha primeira iniciação científica no assunto (seção~\ref{sec_ic_paleomag})
com o mesmo grupo de pesquisa da USP.

Em 2024, o aluno de mestrado \YagoLink{} se juntou ao grupo sob minha
orientação.
Seu projeto é desenvolver um software livre em Python que torne o trabalho
que temos feito na área de microscopia magnética acessível para a comunidade
do paleomagnetismo.
O software, chamado \href{https://www.fatiando.org/magali}{Magali}, é a mais
nova biblioteca adicionada ao projeto \FatiandoLink{}.
Esperamos ter uma primeira versão disponível em meados de 2025.

Uma questão importante que necessitava de resposta é: quantos grãos são
necessários para obtermos uma estimativa estatisticamente significante do campo
indutor?
A resposta dessa questão nos ajudaria a planejar a viabilidade de utilizar os
microscópios atuais para estudos paleomagnéticos.
Em \citet{Bellon2025}, utilizamos simulações micromagnéticas para responder
essa pergunta. Atualmente, o Gelson está trabalhando em uma confirmação
experimental desses resultados.

\begin{fancyenum}{\faBullseye}{Objetivos}
  \item Adaptar técnicas de geofísica aplicada para processar e interpretar
    dados de microscopia magnética.
  \item Obter estimativas da direção e intensidade da magnetização de amostras
      com base em dados de microscopia.
  \item Produzir uma ferramenta de software livre que implementa os métodos
    desenvolvidos para ajudar a difundi-los na comunidade científica.
\end{fancyenum}
\begin{fancyenum}{\faLightbulb}{Principais contribuições}
  \item Criação de uma técnica de inversão semi-automática capaz de estimar
      o momento de dipolo de centenas de minerais magnéticos em poucos minutos.
  \item Determinação do número de minerais pseudo-monodomínio necessários para
      obter uma direção de magnetização estável.
\end{fancyenum}
\begin{fancyenum}{\faRocket}{Impacto da pesquisa}
  \item O método desenvolvido está sendo utilizado pelos demais grupos
      tralhando em microscopia magnética para complementar ou substituir
      seus fluxos de processamento.
  \item O CompGeoLab é um dos poucos (cerca de 2 ou 3) grupos mundiais que
      estão à frente da pesquisa em inversão de microscopia magnética.
\end{fancyenum}



\section{Compilação de dados magnéticos Antárticos}
\label{sec_antartica}

\begin{summarybox}[frametitle=\faInfoCircle{}\quad Resumo da linha de pesquisa]
  \begin{fa-ul}
    \faUserGraduate & Alunos envolvidos: India Uppal (PhD), Arthur Siqueira
    Macêdo (MSc) \\
    \faGlobeAmericas & País dos colaboradores: Brasil, Reino Unido
  \end{fa-ul}
\end{summarybox}

O fluxo geotermal é um importante parâmetro que controla a cobertura de gelo
na Antártica.
Porém, é extremamente difícil obter medições diretas desse parâmetro por causa
do ambiente hostil no interior do continente antártico.
Uma das principais fontes de informação sobre o fluxo geotermal são dados
aeromagnéticos \citep{BurtonJohnson2020}.
Partindo do pressuposto de que o principal mineral magnético na crosta
terrestre é a magnetita, podemos inferir a que profundidade da crosta alcança a
temperatura de Curie desse mineral através da estimativa da espessura da parte
magnetizada da crosta.
Essa informação sobre a localização de uma isoterma em profundidade nos permite
estimar o fluxo geotermal na superfície.
A incerteza nessas estimativas é grande, envolvendo erros na compilação de
dados magnéticos e a limitação do método normalmente utilizado para obter as
estimativas de profundidade \citep{BurtonJohnson2020}.

Meu interesse por essa linha de pesquisa surgiu em 2020 quando me deparei com
os trabalhos feitos pelo grupo do Professor
\href{https://www.satellitengeophysik.uni-kiel.de/de/mitarbeiter/joerg_ebbing}{Jörg Ebbing},
particularmente o artigo \citet{Losing2020} da aluna
\href{https://www.satellitengeophysik.uni-kiel.de/de/mitarbeiter/mareen_loesing}{Mareen Lösing}.
Propus um projeto de doutorado no assunto, que seria financiado pela School of
Environmental Sciences da \UoL{}, para ter a oportunidade de
me aprofundar mais na área.
No final de 2021, a aluna \IndiaLink{} iniciou seu trabalho nesse projeto sob
minha orientação e sob coorientação dos Professores \VanderleiLink{} e
\href{https://www.liverpool.ac.uk/~holme/}{Richard Holme}
(seção~\ref{sec_orientacao}).
Desde então, a India deu início à compilação de um banco de dados com todos os
dados aeromagnéticos brutos e processados disponíveis.
Além disso, ela também estabeleceu um contato com pesquisadores da
British Antarctic Survey (\href{https://www.bas.ac.uk/profile/maxwe32/}{Maximilian Lowe}
e \href{https://www.bas.ac.uk/profile/tomj/}{Tom Jordan})
e com o subcomitê \href{https://www.scar-instant.org/index.php/research-themes/theme-2-solid-earth-ice-interactions/sc1-antarctic-geothermal-heat-flux}{SC1 Antarctic Geothermal Heat Flux}
da \href{https://www.scar.org/}{Scientific Committee on Antarctic Research} (SCAR).
Nosso objetivo é utilizar a técnica das fontes equivalentes para melhorar
a integração dos dados aeromagnéticos Antárticos e mesclá-los com dados de
satélite.
Para tanto, necessitaremos de novos métodos de fontes equivalentes eficientes
para dados magnéticos que possam lidar com a curvatura da Terra.
Também será necessário investigar cada aerolevantamento para determinar sua
confiabilidade e se há a necessidade de reprocessá-los.

Atualmente, a India já realizou uma adaptação do método de \citet{Soler2021}
para dados magnéticos.
Com isso, fomos capazes de modelar centenas de milhares de dados com
espaçamento variável entre as linhas de vôo.
Também fomos capazes de transformar os dados de anomalia magnética em dados de
amplitude do campo anômalo, que é menos sensível a variações na direção de
magnetização da crosta e pode auxiliar na determinação do fluxo geotermal.
Seus resultados serão submetidos como um preprint em breve.
Seus próximos passos serão na investigação dos diversos aerolevantamentos
individuais em colaboração com o pesquisador Tom Jordan da British Antarctic
Survey.
Além disso, o aluno \ArthurLink{} está adaptando o método da India para
coordenadas esféricas, o que será necessário para aplicá-lo a toda a compilação
de dados Antárticos.
Esperamos obter os primeiros resultados até o final de 2025.


\begin{fancyenum}{\faBullseye}{Objetivos}
  \item Reprocessar e harmonizar os dados aeromagnéticos da Antártica.
  \item Produzir uma malha regular de dados da amplitude do campo magnético
    uniforme para a Antártica, combinando dados de satélite e aeromagnéticos.
  \item Investigar o uso de métodos novos de separação regional-residual
    \citep[e.g.,][]{Florio2022} para remover o efeito de fontes rasas dos dados
    antárticos.
\end{fancyenum}


%==============================================================================
\chapter{Ensino, Orientações e Extensão}
\label{cap_ensino}

\begin{figure}[h]
  \HeroFigPad
  \begin{center}
    \includegraphics[width=\textwidth]{images/compgeolab-group-photo-2024-08-12.jpg}
  \end{center}
  \caption{
      Primeiro encontro presencial do \CompGeoLabLink{} no IAG em agosto de
      2024. Da esquerda para a direita: Yago, Santiago, Ellen, eu, India,
      Arthur, Gelson, Gabriel.
  }
\end{figure}
\begin{summarybox}[frametitle=\faChalkboardTeacher{}\quad Resumo das atividades]
  \begin{fa-ul}
    \faUserGraduate & Orientações concluídas: 12 de graduação, 1 de mestrado, 1
      coorientação de doutorado \\
    \faUser & Orientações em andamento: 3 de graduação, 2 de mestrado, 1 de
    doutorado, 1 coorientação de doutorado, 1 coorientação de mestrado \\
    \faChalkboardTeacher & 14 disciplinas de graduação e 3 de pós-graduação ministradas \\
    \faClock & 18 cursos de curta duração ministrados internacionalmente \\
    \faCheckSquare & Habilitação em pedagogia e técnicas de ensino aplicadas ao
      ensino superior \\
    \faTrophy & Paraninfo das turma de formandos em 2017 na UERJ e 2025 na USP \\
    \faLightbulb & Tópicos ensinados incluem: gravimetria, magnetometria,
    sismologia, sensoriamento remoto, métodos numéricos, programação em Python,
    métodos de campo em geofísica, introdução à geologia, problemas inversos,
    geofísica global e geodinâmica da litosfera.
  \end{fa-ul}
\end{summarybox}

As atividades de ensino e mentoria de alunos são onde encontro a maior
satisfação profissional.
Quase nenhuma outra atividade oferece a oportunidade de ter um impacto positivo
direto na vida de outras pessoas.
Minha abordagem para o ensino é muito influenciada pela minha experiência na
graduação (seção~\ref{sec_usp}) e minhas atividades de ciência aberta
(capítulo~\ref{cap_cienciaaberta}).
Utilizo a computação nas minhas aulas para empoderar os alunos com as
habilidades que necessitam para explorar conceitos e dados reais de maneira
independente.
Essa abordagem se mostrou particularmente eficaz em conjunto com uma sólida
base teórica, visualizações interativas e uma seleção de dados abertos
disponíveis para os alunos.
Este capítulo relata minhas atividades de ensino, mentoria e extensão,
incluindo minha abordagem pedagógica e meu papel na criação e ministração de
disciplinas de graduação.

\section{Orientações}
\label{sec_orientacao}

\subsection{Doutorado}

Minha primeira experiência como mentor de um jovem cientista foi a coorientação
do aluno \SantiagoLink{} junto com o
Professor Mario E. Giménez da Universidad Nacional de San Juan, Argentina,
entre 2016 e 2022.
O primeiro contato que tive com o Santiago foi através do software
Fatiando a Terra (seção~\ref{sec_fatiando}).
Em 2015, ele começou a se voluntariar com o projeto, implementando funções
para manejar dados em malhas regulares e o cálculo de espectros de potência.
Quando ele e o Mario me convidaram para coorientar sua tese de doutorado em
2016, fiquei feliz em aceitar continuar trabalhando com ele de maneira mais
regular.
Inicialmente, a proposta era que eu fosse seu orientador principal.
Mas como essa seria minha primeira orientação e seria feita inteiramente de
forma remota, achei mais prudente começar como coorientador.
O Santiago fez excelente progresso nas linha de pesquisa em modelagem com
tesseroides (seção~\ref{sec_modelagemdireta}) e camada equivalente
(seção~\ref{sec_eql}) durante seu doutorado.
Seu envolvimento no Fatiando a Terra aumentou com o tempo. Hoje em dia ele
ocupa uma posição de liderança no projeto, estabelecendo direções para o
desenvolvimento e atuando como mentor para novos membros da comunidade.
Ele é o criador e principal desenvolvedor de duas novas componentes do
Fatiando: \href{https://www.fatiando.org/harmonica}{Harmonica} e
\href{https://www.fatiando.org/choclo/}{Choclo}.
O Santiago também foi fundamental no estabelecimento do nosso grupo de pesquisa,
o \href{https://www.compgeolab.org/}{Computer-Oriented Geoscience Lab}
(CompGeoLab).
Nossas longas conversas sobre o papel da computação na ciência,
reprodutibilidade dos resultados numéricos e formas de fazer ciência aberta
foram as maiores influências que tive para estabelecer os princípios do
CompGeoLab, codificados em nosso manual
\url{https://www.compgeolab.org/manual}.
Em 2022, o Santiago defendeu sua tese de doutorado\footnote{Disponível em
\url{https://github.com/santisoler/phd-thesis}} com sucesso e continua
colaborando com o CompGeoLab regularmente.
Vale a pena apontar que, durante esses 6 anos de orientação, eu e o Santiago
nunca nos encontramos em pessoa por conta de diversos desencontros e a pandemia
de COVID.
Aprendemos juntos como criar uma relação próxima de trabalho inteiramente
online usando uma mistura de chamadas por vídeo semanais e mensagens regulares
pela plataforma Slack.
Atualmente, o Santiago está fazendo um pós-doutorado na University of British
Columbia (UBC), Canadá, sob supervisão da Professora
\href{https://lindseyjh.ca/}{Lindsey J. Heagy}.
Seu projeto inclui trabalhar no software livre
\href{https://simpeg.xyz/}{SimPEG}, desenvolvido pelo grupo da UBC, e facilitar
interações entre o SimPEG e o Fatiando a Terra.
A experiência de trabalhar com o Santigo foi um dos maiores privilégios da
minha carreira.
Santiago é um pesquisador brilhante, um engenheiro de software excepcional
e um verdadeiro amigo.

Iniciei minha segunda experiência de orientação a nível de doutorado em 2021
com a aluna \IndiaLink{} da \UoL{}, coorientada pelo Professor
\href{https://www.pinga-lab.org/people/oliveira-jr.html}{Vanderlei C. Oliveira Jr.}
do \ON{} e pelo Professor
\href{https://www.liverpool.ac.uk/~holme/}{Richard Holme} da University of
Liverpool.
A India foi nossa aluna do curso de Bacharelado em Geofísica de Liverpool e era
uma das melhores alunas de sua turma.
Ela foi selecionada pela School of Environmental Sciences para uma posição
dupla: meio período como aluna de doutorado e meio período empregada como
\textit{Graduate Teaching Assistant}, auxiliando no ensino de disciplinas de
graduação dos cursos de geociências e geografia física.
Acho importante realçar que a seleção para a vaga foi feita inteiramente
baseada no mérito da India e não levou em consideração o projeto de doutorado
ou os orientadores.
Seu projeto de doutorado iniciou uma nova linha de pesquisa para o CompGeoLab
onde buscamos aprimorar as compilações de dados de aeromagnéticos na Antártica
(seção~\ref{sec_antartica}).
Através desse projeto, nos envolvemos com a organização
\href{https://www.scar.org/}{Scientific Committee on Antarctic Research}
(SCAR), especificamente no grupo
\href{https://www.scar-instant.org/index.php/research-themes/theme-2-solid-earth-ice-interactions/sc1-antarctic-geothermal-heat-flux}{SC1
Antarctic Geothermal Heat Flux}.
A India também estabeleceu contato com pesquisadores da British Antarctic
Survey, que criaram a compilação de dados magnéticos antárticos ADMAP2
\citep{Golynsky2018}, para obter os dados brutos e consultá-los sobre os
detalhes do processamento que fizeram.
Após minha volta ao Brasil, o Richard passou a ser seu orientador principal
e eu seu coorientador.
A India é uma pesquisadora com atitude excepcionalmente profissional e
com iniciativa própria, além de ser dedicada e inteligente.

Também em 2021, fui convidado pelo Professor Ricardo I. F. Trindade a
coorientar o aluno de doutorado \GelsonLink{} da \USP{}.
Eu e o Ricardo havíamos conversado durante encontros no AGU Fall Meeting sobre
algumas ideias de adaptar técnicas de métodos potenciais, como a deconvolução
de Euler (seção~\ref{sec_euler}), a dados de microscopia magnética.
O Gelson se juntou ao CompGeoLab para dar início a essa nova linha de pesquisa
(seção~\ref{sec_micromag}), me trazendo de volta ao assunto da minha primeira
iniciação científica na USP (seção~\ref{sec_usp}).
Em apenas um ano, o Gelson já foi bem sucedido na adaptação da deconvolução de
Euler e o método de \citet{OliveiraJr2015}, junto com técnicas de processamento
de imagens que aprendi com minha disciplina de sensoriamento remoto em
Liverpool (seção~\ref{sec_ensino_grad}), para estimar o momento de dipolo
de grãos individuais de minerais ferromagnéticos.
Em 2022, fomos contemplados com um projeto da
\href{https://royalsociety.org/}{Royal Society} para realizar intercâmbios
entre a São Paulo e Liverpool para dar início à colaboração entre as duas
instituições.
Após minha volta ao Brasil, passei a ser o orientador principal do Gelson
e o Ricardo seu coorientador.
Orientar o Gelson tem sido um verdadeiro privilégio.
Ele é respeitoso, dedicado, inteligente, possui uma base sólida em geociências,
tem iniciativa própria e aprende rapidamente conceitos novos que estão fora de
sua zona de conforto.

\subsection{Mestrado}

Em 2020 comecei a orientar o projeto de mestrado do aluno Aidan Hernaman do
curso \textit{MESci in Geology and Geophysics} da \UoL{}.
Seu projeto seria determinar a incerteza envolvida na estimativa da
determinação da Moho através de dados de gravidade (seção~\ref{sec_moho}).
Nosso principal desafio era que a maior fonte de incerteza não são os ruídos do
dado gravimétrico, mas sim nos modelos crustais usados para corrigir os dados.
Nossa abordagem foi a realização de validação cruzada com os dados sismológicos
que utilizamos como vínculos.
O Aidan defendeu seu mestrado em 2021.
Infelizmente, os resultados obtidos não foram conclusivos e não fomos capazes
de publicá-los.
A orientação do Aidan foi muito difícil por conta do isolamento da pandemia de
COVID.
Eu tinha experiência com orientações a distância porém as condições de trabalho
do Aidan eram difíceis, com poucos recursos e a falta de acesso a uma rede de
apoio dos colegas de curso.
Mesmo assim, ele foi capaz de realizar seu projeto, escrever sua dissertação
e ser aprovado no mestrado, o que considero um grande feito dadas as
circunstâncias.

Já na USP em 2024, iniciei a orientação dos alunos \YagoLink{} e \ArthurLink{}.
Ambos são formados em geofísica pela Universidade de Brasília.
Yago está trabalhando na área de microscopia magnética e Arthur na área de
modelagem de dados magnéticos com fontes equivalentes.
Ambos chegaram ao IAG com muita energia e vontade de aprender, o que é o sonho
de todo orientador.
Até o momento, Yago e Arthur cumpriram os requisitos de disciplinas e estão
progredindo bem em seus projetos.
O Yago está estabelecendo um contato com a pesquisadora Melina Macouin do
\textit{Centre National de la Recherche Scientifique} da França, que está
desenvolvendo um novo tipo de microscópio magnético.
Com essa colaboração, o Yago passaria um tempo na França na segunda metade de
2025 para aprender o funcionamento do novo microscópio e mostrar para o grupo
do CNRS como funciona o software Magali.
Orientar o Yago e o Arthur tem sido um grande prazer.
Eles trouxeram uma energia extremamente positiva para o grupo e estão sempre
dispostos a ajudar seus colegas.
Ambos estão muito envolvidos com as atividades do IAG, organizando um ciclo de
seminários, realizando monitorias em disciplinas de graduação e como
representantes discentes na Comissão Coordenadora de Programa da Geofísica.

Também em 2024 dei início a coorientação do aluno de mestrado \ErosLink{} da
Universidade Federal do Paraná.
O Eros é orientado pela Professora
\href{https://lattes.cnpq.br/5310171606215286}{Alessandra de Barros e Silva Bongiolo}
e seu projeto é gerar um mapa integrado com todos os dados de gravidade
terrestre da região sul e sudeste do país.
Além disso, pretendemos integrar dados de satélite para preencher as lacunas
nos levantamentos terrestres.
O Eros é um aluno muito dedicado e com um talento natural para a programação,
habilidade que ele desenvolveu por conta própria ao longo de sua graduação em
geologia.
Ele demonstra muita iniciativa e perseverança quando surgem problemas ao longo
do caminho.
Pudemos recebê-lo no IAG em 2025 durante a XXVII Escola de Verão de Geofísica
do IAG e ele se integrou rapidamente com o resto do grupo de pesquisa.
Esperamos recebê-lo de volta no IAG para um doutorado no futuro próximo.


\subsection{Graduação}

Fui o orientador de 11 trabalhos de conclusão de curso de geofísica da \UoL{},
com alunos atuando em diversas linhas de pesquisa\footnote{Uma lista completa
dos alunos e seus projetos está disponível no meu Currículo Lattes:
\url{https://lattes.cnpq.br/\Lattes}}.
Aprendi com essas orientações como guiar alunos sem experiência prévia em
pesquisa pelas etapas iniciais de um projeto e como ajustar o nível dos
projetos propostos com o nível dos alunos no final de um curso de graduação no
Reino Unido.
Foi particularmente gratificante observar os alunos progredirem durante seus
projetos e produzirem trabalhos de conclusão com uma qualidade muito acima do
que eles achavam que seriam capazes.

Ellen

Gabriel


Paulo e Felipe.

Recrutando mulheres.

\section{Cursos de curta duração}
\label{sec_workshops}

\begin{subsummarybox}[frametitle=\faClock{}\quad Cursos e workshops ministrados online]
  \begin{paperlist}
    2022 &
      Crafting beautiful maps with PyGMT.
      \textit{EGU General Assembly}.
      \GitHub{GenericMappingTools/egu22pygmt}
      \\
    ~ &
      A geophysical tour of mid-ocean ridges.
      \textit{Transform 2022} (online).
      \GitHub{leouieda/transform2022}.
      \YouTube{NzJmRlJCNbQ}
      \\
    2021 &
      The Generic Mapping Tools for Geodesy.
      \textit{UNAVCO} (online).
      \GitHub{GenericMappingTools/2021-unavco-course}
      \\
    2020 &
      Let's build a geophysical inversion with Python.
      \textit{IRTG-2379 Graduate School: Modern Inverse Problems},
      \textit{RWTH Aachen University} (online).
      \GitHub{compgeolab/2020-aachen-inverse-problems}
      \\
    ~ &
      The Generic Mapping Tools for Geodesy.
      \textit{UNAVCO} (online).
      \GitHub{GenericMappingTools/2020-unavco-course}.
      \YouTube{EQgxDmCXvj4}
      \\
    ~  &
      From scattered data to gridded products using Verde.
      \textit{Transform 2020} (online).
      \GitHub{fatiando/transform2020}.
      \YouTube{-xZdNdvzm3E}
  \end{paperlist}
\end{subsummarybox}
\begin{subsummarybox}[frametitle=\faClock{}\quad Cursos e workshops ministrados presencialmente]
  \begin{paperlist}
    2025 &
      Kit de sobrevivência digital para cientistas.
      \textit{XXVII Escola de Verão de Geofísica do IAG - USP}.
      \GitHub{compgeolab/kit}
      \\
    2019 &
      Best Practices for Developing and Sustaining Your Open-Source Research Software.
      \textit{AGU Fall Meeting}.
      \GitHub{agu-ossi/2019-agu-oss}
      \\
    ~  &
      Become a Generic Mapping Tools Contributor Even If You Can't Code.
      \textit{AGU Fall Meeting}
      \\
    ~  &
      The Generic Mapping Tools for Geodesy.
      \textit{Scripps Institution of Oceanography} and \textit{UNAVCO}.
      \GitHub{GenericMappingTools/2019-unavco-course}.
      \YouTube{uPUt4\_kd6m8}
      \\
    ~  &
      Introduction to Python Workshop (Earth Sciences REU program).
      \textit{Department of Geology and Geophysics, \UHM}.
      \GitHub{leouieda/2019-06-reu-python}
      \\
    2018 &
      Best Practices for Modern Open-Source Research Codes.
      \textit{AGU Fall Meeting}.
      \GitHub{agu-ossi/2018-agu-oss}
      \\
    ~  &
      Git and GitHub: What are their uses? Are they worth the effort? Let's find out!
      \textit{ASPRS UHM Student Chapter, \UHM}
      \\
    2017 &
      Introduction to Python.
      \textit{Department of Geology and Geophysics, \UHM}.
      \GitHub{leouieda/python-hawaii-2017}
      \\
    2016 &
      Python for Geologists (SAGEO).
      \textit{Faculdade de Geologia, \UERJ}.
      \GitHub{leouieda/python-geologia-2016}
      \\
    ~  &
      Python como uma ferramenta numérica em Ciências da Terra: uma nova
      abordagem de programação.
      \textit{XVIII Escola de Verão de Geofísica do IAG-USP}.
      \GitHub{leouieda/verao2016}
      \\
    2014 &
      Tópicos de inversão em geofísica.
      \textit{III Semana de Geofísica da UnB}.
      \GitHub{pinga-lab/inversao-unb-2014}
      \\
    2012 &
      Tópicos de inversão em geofísica.
      \textit{XVI Escola de Verão de Geofísica do IAG-USP}.
      \GitHub{pinga-lab/inversao-iag-2012}
  \end{paperlist}
\end{subsummarybox}

Minha primeira experiência com o ensino foi através do curso ``Tópicos de
inversão em geofísica'' que ministrei em 2012 junto com meu amigo e então
colega de doutorado
\href{https://www.pinga-lab.org/people/oliveira-jr.html}{Vanderlei C. Oliveira Jr.}
na XVI Escola de Verão de Geofísica do IAG-USP.
Foi durante esse curso que percebi minha paixão pelo ensino e decidi seguir a
carreira acadêmica para poder combinar ensino, pesquisa e extensão.
Desde então, ministrei diversos cursos de curta duração e workshops em formato
online e presencial.
Esses cursos complementam o ensino tradicional em disciplinas de graduação e
pós-graduação, fornecendo a oportunidade de experimentar com tecnologias,
formatos de ensino e tópicos pouco tradicionais.

A maioria dos cursos que ministrei estão relacionados à programação. O formato
curto é adequado para uma introdução à conceitos básicos de programação ou
para abordar um assunto específico (e.g., como criar mapas com o
\href{https://www.pygmt.org}{PyGMT},
como interpolar dados com o \href{https://www.fatiando.org/verde}{Verde}
ou como criar testes unitários para seu software).
Por isso, acho as ``escolas de verão'' e ``semanas da geofísica'' organizadas
pelas universidades tão proveitosas.
Esses cursos também podem fornecer aos alunos um contato com especialistas de
todo o mundo.
Esse contato pode inclusive ser feito com um orçamento limitado
devido aos avanços recentes nas plataformas de vídeo conferência e a difusão de
atividades online causados pela pandemia de COVID.


\section{Disciplinas de graduação}
\label{sec_ensino_grad}

\begin{subsummarybox}[frametitle=\faGraduationCap{}\quad Disciplinas ministradas na \UERJ{}]
  \begin{courselist}
    2015--2016 &
      IME03-1366 Matemática Especial I.
      \newline
      \GitHub{mat-esp/about}
      \\
    2014--2016 &
      FGEL04-12422 Geofísica II.
      \newline
      \GitHub{leouieda/geofisica2}
      \\
    ~ &
      FGEL04-12421 Geofísica I.
      \newline
      \GitHub{leouieda/geofisica1}
      \\
    2015 &
      FGEL01-00805 Geologia Geral I.
  \end{courselist}
\end{subsummarybox}
\begin{subsummarybox}[frametitle=\faGraduationCap{}\quad Disciplinas ministradas na \UoL{}]
  \begin{courselist}
    2020--2023  &
      ENVS398: Global Geophysics and Geodynamics.
      \newline
      \GitHub{leouieda/lithosphere}
      \\
    ~ &
    ENVS258: Environmental Geophysics.
      \newline
      \GitHub{leouieda/remote-sensing}.
      \newline
      \GitHub{leouieda/gravity-processing}.
      \\
    ~ &
    ENVS386: Geophysical Data Modelling.
      \newline
      \GitHub{leouieda/ml-intro}.
      \\
    ~ &
      ENVS101/106: Study Skills and GIS (tutorial).
      \\
    2019--2021 &
      ENVS123: Introduction to Geoscience and Earth History.
      \\
    2019--2020  &
      ENVS362: Geophysics Field School.
  \end{courselist}
\end{subsummarybox}
\begin{subsummarybox}[frametitle=\faGraduationCap{}\quad Disciplinas ministradas na \USP{}]
  \begin{courselist}
    2024--2024  &
      AGG0204 - Computação para Geofísicos.
      \\
    2023--atual  &
      AGG0110 - Elementos de Geofísica.
      \\
    ~ &
      AGG0011 - Problemas Integrados em Ciências da Terra I.
      \\
    ~ &
      AGG0669 - Gravimetria e Magnetometria Aplicadas à Prospecção de Bens
      Minerais e Estruturas Crustais.
  \end{courselist}
\end{subsummarybox}

Na \UERJ{}, tive a oportunidade de criar o conteúdo de três disciplinas de
graduação: Matemática Especial I e Geofísica I e II.
A disciplina Matemática Especial I pertence ao curso de Bacharelado em
Oceanografia e cobria tópicos avançados de matemática.
Meu papel ao assumir essa disciplina era convertê-la em uma introdução à
programação em Python e ao cálculo numérico.
Decidi incluir no início da disciplina uma introdução ao software de controle
de versão \href{https://git-scm.com/}{git} e à plataforma
\href{https://github.com/}{GitHub}.
Assim, pude manejar a disciplina inteiramente pelo GitHub, com cada lição
sendo armazenada em um repositório da organização
\url{https://github.com/mat-esp}.
Durante as aulas práticas, os alunos se dividiam em grupos e cada grupo era
automaticamente fornecido com uma cópia do repositório da lição pela plataforma
\href{https://classroom.github.com/}{GitHub Classroom}.
Ao final da aula, os alunos submetiam suas soluções para a tarefa da lição
também pelo GitHub, onde recebiam as notas e correções.

As disciplinas de geofísica, cobrindo uma introdução aos métodos geofísicos,
são parte do curso de Bacharelado em Geologia e haviam acabado de serem
reformuladas quando assumi meu cargo na UERJ em 2014.
Logo, pude criar o conteúdo das disciplinas por conta própria e estabelecer
como gostaria que fossem estruturadas.
Optei por dividi-las entre aulas teóricas e aulas práticas computacionais.
Nas práticas, utilizei o software \href{https://jupyter.org/}{Jupyter} para
criar \textit{notebooks} que explicavam os conceitos abordados em aula
utilizando uma combinação de texto, equações, código pronto para demonstrar os
conceitos, tarefas para serem executadas através de visualizações interativas
(figura~\ref{fig_notebooksismica}) e perguntas para serem respondidas como
parte da avaliação somativa da disciplina.
Essa abordagem foi bem recebida pelos alunos.
Inclusive, fui escolhido como paraninfo da turma de formandos da Geologia em
2016 (ano de ingresso 2012).

\begin{figure}[t]
  \begin{center}
    \includegraphics[width=\textwidth]{images/seismic-waves-demo.jpg}
  \end{center}
  \caption{
    Exemplo de um notebook usado na minha disciplina ``Geofísica 2'' da UERJ
    para ensinar o conceito de ondas sísmicas, reflexão, refração e conversão
    de ondas P em ondas S ao interagir com uma interface geológica. O notebook
    contém instruções, teoria, perguntas e código pronto que os alunos
    podem executar e modificar para criar animações da propagação de ondas
    elásticas (utilizando o código de diferenças finitas do Fatiando a Terra).
    A figura no notebook é parte de uma animação da propagação de uma onda P
    que incide sobre uma interface, gerando ondas P e S refletidas e
    refratadas.
    As cores representam a soma do divergente e o rotacional do campo de
    deformações, mostrando as frentes de onda.
    Vetores indicam o deslocamento de cada ponto do modelo, com ondas P e S
    tendo deslocamento perpendicular e paralelo às frentes de onda,
    respectivamente.
  }
  \label{fig_notebooksismica}
\end{figure}

Na \UoL{}, participei de diversas disciplinas dos cursos de
Geofísica e Geologia, ministrando
programação em Python para Ciências da Terra em ``ENVS101
Study Skills and GIS'',
introdução à estrutura da Terra e isostasia em ``ENVS123 Introduction to
Geoscience and Earth History'',
inversão não-linear e aprendizagem de máquinas em ``ENVS386 Geophysical Data
Modelling''
e a matéria de campo do terceiro ano de geofísica
``ENVS362 Geophysics Field School''.
Continuei com a abordagem computacional que desenvolvi na UERJ, dessa vez
incluindo tarefas onde os alunos devem escrever parte do código.
Adotei a metodologia de
\href{https://pt.wikipedia.org/wiki/Aula_invertida}{aula invertida}, produzindo
vídeos explicando a base teórica para os alunos assistirem independentemente
e utilizando todo o tempo em sala de aula para atividades práticas com os
notebooks e discussões.

Criei a nova disciplina optativa ``ENVS398: Global Geophysics and Geodynamics''
junto com o Professor
\href{https://www.liverpool.ac.uk/environmental-sciences/staff/andrew-biggin/}{Andy Biggin}.
Utilizamos aulas gravadas para ensinar o conteúdo teórico.
As partes práticas da disciplina são dividas em duas partes.
Durante metade da disciplina, ministrada pelo Andy, os alunos aprendem sobre o
núcleo e o manto terrestre, discutindo artigos recentes da literatura durante
as aulas presenciais.
Durante a outra metade da disciplina, ministrada por mim, os alunos aprendem
sobre a geodinâmica da litosfera.
Nas aulas práticas, os alunos desenvolvem a implementação computacional dos
modelos abordados nas aulas teóricas e os comparam a dados reais.
Utilizei os notebooks com parte do seu código fornecido por mim para os alunos
construírem suas soluções em etapas gradativamente mais desafiadoras.
Também utilizei um conjunto global de dados de distúrbios da gravidade, fluxo de
calor geotermal, topografia e idade do assoalho oceânico para os alunos
interpretarem e processarem livremente.
Essa matéria foi consistentemente elogiada pelos alunos nos formulários de
avaliação semestrais das disciplinas.

Fui o responsável pela disciplina ``ENVS258 Environmental Geophysics'' onde
ensinava uma introdução ao sensoriamento remoto, processamento e aquisição de
dados de gravidade e com uma componente de campo, onde introduzimos aos alunos
os equipamentos que possuímos em Liverpool (magnetometria, GRP,
eletrorresistividade, GPS, EM-31 e refração sísmica).
Desenvolvi todo o material prático para a componente de sensoriamento remoto,
utilizando novamente os notebooks e dados abertos dos satélites Landsat
fornecidos na plataforma \href{https://earthexplorer.usgs.gov/}{EarthExplorer}
da USGS.
A avaliação somativa dessa componente era um relatório onde os alunos escolhem
um tema dentro do escopo da disciplina, fazem a pesquisa bibliográfica, baixam
os dados relevantes do EarthExplorer, processam os dados usando notebooks em
Python e geram suas visualizações e conclusões.
Para a grande maioria dos alunos, essa é sua primeira tarefa independente e o
seu primeiro contato com a pesquisa.
A qualidade e criatividade dos relatórios que os alunos produziam
frequentemente me surpreendia.
Em múltiplas ocasiões, incorporei o trabalho de alunos nas minhas aulas porque
eram simplesmente superiores aos exemplos que eu pude desenvolver\footnote{Por
exemplo, as práticas
\url{https://github.com/leouieda/remote-sensing/blob/main/practicals/practical2.ipynb}
e
\url{https://github.com/leouieda/remote-sensing/blob/main/practicals/practical4.ipynb}}.
A componente de sensoriamento remoto é sempre elogiada pelos alunos nas
avaliações do curso, o que me dá muito orgulho porque foi um tema que aprendi
quase inteiramente através de ministrar essa disciplina.


USP

\section{Disciplinas de pós-graduação}
\label{sec_ensino_pos}

\begin{subsummarybox}[frametitle=\faGraduationCap{}\quad Disciplinas ministradas na \USP{}]
  \begin{courselist}
    2025--atual  &
    AGG5740 - Teoria de Inversão em Geofísica.
      \\
    2024--atual &
    AGG5949 - Tópicos Gerais de Geofísica.
      \\
    ~ &
    AGG5954 - Modelagem e Interpretação de Dados de Gravimetria
    e Magnetometria.
  \end{courselist}
\end{subsummarybox}


\section{Atividades de Extensão}

Minhas atividades na área de extensão universitária são mais limitadas que
minha atuação em outras áreas.
Parte da razão é meu foco adicional em outras atividades, como minha atuação
em ciência aberta (capítulo~\ref{cap_cienciaaberta}).
Por conta da pandemia de COVID de 2020, muitas das atividades de extensão que
eram promovidas pela universidade estavam canceladas durante a maior parte da
minha estadia em Liverpool.
Em 2022, com a retomada das atividades presenciais, participei de três eventos
na \UoL{} chamados de \textit{Open Days}, durante os quais
os alunos de ensino médio visitam a universidade para conhecer mais sobre os
cursos oferecidos.
Na parte das Ciências da Terra, fizemos demonstrações sobre como a viscosidade
influencia o fluxo de lava, como o Ground Penetrating Radar (GPR) detecta
objetos em subsuperfície e como o conceito de isostasia explica a espessura
crustal em regiões montanhosas.
Além dos Open Days, fui voluntário no evento \textit{Hour of Code} para ensinar
programação para alunos de ensino fundamental da Salt Lake Elementary School em
Honolulu, E.U.A.
Também fui entrevistado nos podcasts de divulgação das geociências
\href{https://undersampledrad.io}{Undersampled Radio}
(episódio ``\href{https://undersampledrad.io/home/2016/7/open-sourcery}{Open
Sourcery}'' de 19/05/2016)
e \href{https://www.dontpanicgeocast.com/166}{Don't Panic Geocast}
(episódio ``\href{https://www.dontpanicgeocast.com/166}{You are headed to a warm and sunny place}''
de 27/04/2018).
Essas atividades são extremamente gratificantes e gostaria de dedicar mais
tempo a elas no futuro.
Em particular, gostaria de expandir uma atividade que desenvolvi para explicar
aquisições aeromagnéticas utilizando ímãs enterrados e o aplicativo de celular
\href{https://phyphox.org/}{Phyphox}, que dá acesso aos dados registrados pelos
magnetômetros dos celulares modernos.


%==============================================================================
\chapter{Conclusão}
\label{cap_conclusao}

Este memorial contém um relato dos últimos 19 anos, desde minha entrada no
curso de Bacharelado em Geofísica da \USP{} até minha
posição atual como Lecturer na \UoL{}.
Ao longo desse caminho, passei por 6 instituições em 4 países, fui autor de 16
artigos científicos e 10 softwares livres para ciência, ministrei 10
disciplinas diferentes de graduação e orientei e coorientei 16 alunos de
graduação, mestrado e doutorado.
Atuo em diversas linhas de pesquisa relacionadas a problemas inversos em
métodos potenciais, com aplicação na prospecção de recursos naturais e estudos
da crosta terrestre em escala continental.
Tenho uma forte convicção de que a ciência deve ser feita de forma
transparente, reprodutível, reutilizável e cooperativa.
Por isso, invisto uma grande parte do meu tempo e esforço em iniciativas que
apoiam a ciência aberta e o uso de software livre na ciência e na educação.
O ensino e a interação com os alunos são meus maiores motivadores
profissionais.
Não consigo me imaginar atualmente em uma carreira na qual não tenho a
oportunidade de ser professor e mentor.

Embora minha experiência em Liverpool tenha sido gratificante e possibilitado
meu crescimento profissional, descobri alguns aspectos da academia na
Inglaterra que a tornam menos atrativa para mim.
Por exemplo, a natureza extremamente comercial do ensino superior, a grande
disparidade de salários e o curso extremamente curto de Bacharelado de apenas
aproximadamente 50 dias de aulas por semestre ao longo de três anos (comparado
com aproximadamente 100 dias por semestre ao longo de quatro ou cinco anos nas
universidades brasileiras).
Este último fator resulta em alunos entrando na pós-graduação com menos
preparo, o que significa que projetos que eu julgaria a nível de doutorado no
Brasil são a nível de pós-doutorado para alunos do Reino Unido.
Também tive minha filha Yara em 2020 e, por conta das restrições em viagens
internacionais, não tive suporte familiar algum.
Percebo agora mais do que nunca o quanto a proximidade da família contribui
para o bem-estar e o balanço saudável entre o trabalho e a vida pessoal.
A longo prazo, todos esses fatores podem no futuro resultar em uma produção
científica e pedagógica que está abaixo das minhas perspectivas profissionais.

As circunstâncias familiares e profissionais que tenho agora são muito
diferentes do que eram cinco anos atrás quando decidi sair do meu emprego na
UERJ.
Aprendi muito com os últimos seis anos que passei em instituições
internacionais.
Pude estabelecer relações duradouras com pesquisadores excelentes e
experienciar como diferentes culturas abordam o ensino superior e o
financiamento para pesquisa.
Desde minha primeira experiência no exterior na York University
(seção~\ref{sec_york}), aprecio cada vez mais o que temos no Brasil e, em
particular, na \USP{}.
Obter os títulos de Bacharel, Mestre e Doutor sem acumular uma dívida
gigantesca é um privilégio que poucos tem nos E.U.A., Canadá e Inglaterra.
Devo muito ao país que me providenciou um ensino superior de qualidade
excepcional de forma gratuita e bolsas para realizar a pós-graduação.
Sinto que agora está na hora de retribuir meu país com a experiência que
adquiri no Brasil e no exterior em pesquisa, ensino, administração e extensão.

Caso tenha a oportunidade, pretendo buscar colaborações com os pesquisadores e
pesquisadoras do IAG e do IGc nas áreas de estrutura crustal da América do Sul,
de aplicações geoambientais (e.g., através do sensoriamento remoto), de
gravimetria e geodésia física, de desenvolvimento de software (e.g., através
dos aplicativos do grupo de sismologia e do envolvimento tradicional do IAG no
Generic Mapping Tools) e de paleomagnetismo.
Adoraria ter a chance de expandir minha atuação na extensão universitária,
especialmente na divulgação da geofísica através de visitas a escolas, nos
cursos de Geofísica para a Terceira Idade do IAG e na produção de recursos
educacionais abertos em português.
Também vejo que posso contribuir em diversos aspectos dos cursos de graduação
e pós-graduação.
Uma mudança no curso desde minha passagem pela USP que me chamou a atenção foi
a introdução das disciplinas ``Problemas Integrados em Ciência da Terra I e
II''.
Essas disciplinas visam fornecer uma ponte entre o ciclo básico e o conteúdo de
geofísica, um aspecto do qual sentia falta durante meu curso de graduação.
Os tópicos cobertos por essas disciplinas possibilitam o uso das técnicas
computacionais de ensino que utilizo em minhas aulas.
Possuo experiência prévia de ensino nos tópicos das disciplinas do concurso:
métodos potenciais, geofísica e geodinâmica global, estrutura interna da Terra
e sismologia.
Além das disciplinas especificadas no concurso, eu teria a capacidade de
propor disciplinas optativas e de pós-graduação nos tópicos de gravimetria,
engenharia de software científico (complementando a disciplina AGG0204),
métodos quantitativos de sensoriamento remoto, aprendizagem de máquina aplicada
a Ciência da Terra, entre outros.
Continuaria a oferecer cursos de curta duração presenciais e online,
particularmente os cursos da organização
\href{https://softwareunderground.org/}{Software Carpentry} da qual sou um
instrutor credenciado.
Gostaria também de propor o uso da observação por pares como uma técnica para
difundir boas práticas pedagógicas, porque acredito ter coisas a
contribuir assim como sei que tenho muito a aprender.
Em suma, seria uma honra ter a chance de fazer minha parte para formar os
alunos brasileiros e contribuir para a \USP{}.


%==============================================================================
\backmatter
\bibliographystyle{apalike-doi}
\bibliography{references}

\end{document}
